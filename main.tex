\documentclass[a4paper,10pt,reqno]{amsart}

\input{Exstuff}

\newtheorem{ex}{Exercise}[section]
\newenvironment{sol}
  {\renewcommand\qedsymbol{$\blacksquare$}\begin{proof}[Solution]}
  {\end{proof}}

  \setlength\parindent{0pt}

\usepackage{multicol}
\usepackage{adjustbox}



\title{Exercises Galois Theory and some solutions}

\author{Carsten Dietzel, Silvia Properzi}
%\date{February 2024}
\address[Carsten Dietzel]{Department of Mathematics and Data Science, Vrije Universiteit Brussel, Pleinlaan 2, 1050 Brussel, Belgium}
\email{Carsten.Dietzel@vub.be}


\address[Silvia Properzi]{Department of Mathematics and Data Science, Vrije Universiteit Brussel, Pleinlaan 2, 1050 Brussel, Belgium}
\email{Silvia.Properzi@vub.be}

\begin{document}

\maketitle
\section{Week 1}


\begin{ex}
\label{1.1}
    Let $\sigma : K\to L$ be a field homomorphism, prove that $\sigma$ is injective.
\end{ex}
\begin{sol}[\textit{using ring theory}]
    We know that $\ker\sigma$ is an ideal of $K$.
    But $K$ is a field so the only possibilities are that $\ker\sigma=\{0\}$ or $\ker\sigma=K$.
    As $\sigma(1)=1\neq 0$, we know that $1\not\in \ker\sigma$, hence $\ker\sigma=\{0\}$ which means that $\sigma$ is injective.
\end{sol}
\begin{sol}[\textit{by hands}]
    Let $a,b\in K$ such that $\sigma(a)=\sigma(b)$.
    Then 
    \[
    \sigma(a-b)=\sigma(a)-\sigma(b)=0.
    \]
    If we assume that $a\neq b$, then $a-b$ is a non-zero element in a field, hence it is invertible.
    Then 
    \[
    1=\sigma(1)=\sigma((a-b)(a-b)^{-1})=
    \sigma(a-b)\sigma((a-b)^{-1})=0,
    \]
    which is a contradiction.
    Therefore $a=b$ and so $\sigma$ is injective.
\end{sol}

\begin{ex}
\label{1.2}
    Let $K$ be a field, $K_0$ be its prime field and $\sigma:K\to K$ be a field homomorphism. Prove that $\sigma\in\Hom(K/K_0,K/K_0)$.
\end{ex}


\begin{ex}
\label{1.3}
Let $\Q[i]=\{a+ib\mid a,b\in \Q\}$, $\Q[\sqrt{2}]=\{a+\sqrt{2}b\mid a,b\in\Q\}$ and $\Q(i)=\{\frac{a+ib}{c+di}\mid a,b,c,d\in \Q\}$, $\Q(\sqrt{2})=\{\frac{a+\sqrt{2}b}{c+\sqrt{2}d}\mid a,b,c,d\in\Q\}$.

\begin{enumerate}[label=(\roman*)]
    \item Prove that $\Q(\sqrt{2})=\Q[\sqrt{2}]$ and $\Q(i)=\Q[i]$.
    \item Prove that $\Q(i)$ and $\Q(\sqrt{2})$ are not isomorphic.
\end{enumerate}
\end{ex}
\begin{sol}~
\begin{enumerate}[label=(\roman*)]
    \item Clearly for every field extension $L/K$ and 
    every $\alpha\in L$ we have that
    $K[\sqrt{2}]\subseteq K(\alpha)$.

    Vice versa take $\frac{a+\sqrt{2}b}{c+\sqrt{2}d}\in \Q(\sqrt{2})$,
    then we can write:
    \[
    \frac{a+\sqrt{2}b}{c+\sqrt{2}d}=\frac{(a+\sqrt{2}b)(c-\sqrt{2}d)}{(c+\sqrt{2}d)(c-\sqrt{2}d)}=
    \frac{ac-2bd+(bc-ad)\sqrt{2}}{c^2-2d^2}.
    \]
    Hence
    \[
    \frac{a+\sqrt{2}b}{c+\sqrt{2}d}=\frac{ac-2bd}{c^2-2d^2}+\frac{bc-ad}{c^2-2d^2}\sqrt{2}\in \Q[\sqrt{2}].
    \]
    In a similar way, given $\frac{a+ib}{c+id}\in \Q(i)$,
    we can write it as
    \[
    \frac{a+ib}{c+id}=\frac{ac+bd}{c^2+d^2}+\frac{bc-ad}{c^2+d^2}i\in \Q[i].
    \]
    \item Assume that $\Q(i)$ and $\Q(\sqrt{2})$ were isomorphic 
    and let $\varphi:\Q(i)\to \Q(\sqrt{2})$
    be a field isomorphism.
    Then 
    \[
    \varphi(i)^2=
    \varphi(i^2)=
    \varphi(-1)=
    -\varphi(1)=-1.
    \]
    But $\varphi(i)\in\Q(\sqrt{2})$ and,
    using the previous part of the exercise,
    $\Q(\sqrt{2})\subseteq \R$ where every square is positive, a contradiction.\qedhere
\end{enumerate}
    
\end{sol}

\begin{ex}
\label{1.4}~
    \begin{enumerate}[label=(\roman*)]
        \item Let $a=\sqrt{2}$ and $b=\sqrt[3]{3}$. Prove that $ab$ is algebraic over $\Q$.
        \item Show that $\sqrt{2} + i$ is algebraic over $\Q$ by finding a nonzero polynomial $f \in \Q[X]$ with $\deg(f) = 4$ such that $f(\sqrt{2} + i) = 0$. What are the other roots of $f$?
    \end{enumerate}
\end{ex}

\begin{sol}~
    \begin{enumerate}[label=(\roman*)]
        \item Observe that $ab^6=\sqrt{2}^6\sqrt[3]{3}^6=8\cdot 9=72$.
        Therefore $ab$ is a root of the polynomial $X^6-72\in\Q[X]$.
        \item let $\alpha=\sqrt{2} + i$. 
        Then $\alpha^2=2+2\sqrt{2}i-1=1+2\sqrt{2}i$ and $\alpha^2-1=2\sqrt{2}i$.
        Squaring both sides of the last equality we have that
        \[
         \alpha^4-2\alpha^2+1=(\alpha^2-1)^2=(2\sqrt{2}i)^2=-8.
        \]
        Therefore the polynomial 
        \[
        f(X)=X^4-2X^2+1+8=X^4-2X^2+9\in \Q[X]
        \]
        is such that $f(\alpha)=0$.

        Looking for other roots of $f$ means to find all $x\in C$
        such that $f(x)=0$, i.e. $x^4-2x^2+9=0$. 
        Going back to how we construct $f$, this equation 
        can also be written as $(x^2-1)^2=(2\sqrt{2}i)^2$.
        Therefore $x^2-1=2\sqrt{2}i$ or $x^2-1=-2\sqrt{2}i$.
        So 
        \[
        x^2=2\sqrt{2}i+1=\alpha^2\qquad\text{ or } \qquad x^2=-2\sqrt{2}i+1=\overline{\alpha^2}=\overline{\alpha}^2,
        \]
        where $\overline{\alpha}$ indicates the complex conjugate of 
        $\alpha$.
        Thus $x=\pm \alpha$ or $x=\pm \overline{\alpha}$ and the 4 roots of $f$ are
        \[
        x_1=\alpha=\sqrt{2} + i, x_2=-\alpha=-\sqrt{2} - i,
        \]
        \[
        x_3=\overline{\alpha}=\sqrt{2} - i, x_4=-\overline{\alpha}=-\sqrt{2} + i.
        \]
    \end{enumerate}
\end{sol}

\begin{ex}
\label{1.5}
Let $p$ be a prime number. Denote by $\binom{n}{k}$ the binomial coefficient ``$n$ over $k$".
    \begin{enumerate}[label=(\roman*)]
    \item Prove that $p$ divides $\binom{p}{k}$ for $1 \leq k \leq p-1$.
    \item Let $K$ be a field of characteristic $p$. Show that the map $\Phi: K \to K$; $x \mapsto x^p$ is a field endomorphism. This map is called the \emph{Frobenius endomorphism} of $K$.
\end{enumerate}
\end{ex}

\begin{sol}~

    \begin{enumerate}[label=(\roman*)]
        \item Note that $p \nmid i$ for $1 \leq i \leq p-1$, therefore $p \nmid 1 \cdot 2 \ldots \cdot k = k!$ for $k \leq p-1$. If $k \geq 1$, then $p \nmid (p-k)!$ for the same reason.

        We have $\binom{p}{k} = \frac{p!}{k!(p-k)!}$ which can also be written as
        \[
        p! = \binom{p}{k} \cdot k! \cdot (p-k)!
        \]
        If $1 \leq k \leq p-1$, then $p|p!$, so $p$ has to divide at least one factor on the right side. As $p \nmid k!,(p-k)!$, it follows that $p | \binom{p}{k}$.

        \item Clearly, $\Phi(1) = 1^p = 1$. As $K$ is commutative under multiplication, we see for arbitrary $x,y \in K$ that
        \[
        \Phi(xy) = (xy)^p = x^py^p = \Phi(x) \Phi(y).
        \]
        Furthermore,
        \begin{align*}
            \Phi(x+y) & = (x+y)^p \\
            & = \sum_{k=0}^p \binom{p}{k}x^py^{p-k} \\
            & = x^p + y^p + \sum_{k=1}^{p-1} \underbrace{\binom{p}{k}}_{=0}x^ky^{p-k} \\
            & = x^p + y^p = \Phi(x) + \Phi(y).
        \end{align*}
        Note that $\binom{p}{k} = 0$ for $1 \leq k \leq p-1$ follows from the divisibility $p | \binom{p}{k}$ proven in the previous part.

        Therefore, $\Phi$ is a field homomorphism.
    \end{enumerate}
\end{sol}

\begin{ex}
\label{1.6}
    Let $L/K$ be a field extension and let $M$ be a subring of $L$ that contains $K$. Suppose that $\dim_KM < \infty$.
    \begin{enumerate}[label=(\roman*)]
        \item Prove that for any $\alpha \in M$, there is a nonzero $f \in K[X]$ with $f(\alpha) = 0$.

        \noindent \textit{Hint:} The elements $\alpha^n$ ($n \geq 0$) are linearly dependent.
        
        \item Prove that $M$ is a field.
        
        \noindent \textit{Hint:} For any $0 \neq \alpha \in K$, let $f \in \Q[X]$ be a nonzero polynomial with $f(\alpha) = 0$ whose degree is as small as possible. If $f = \sum_{i=0}^{\infty} a_iX^i$, prove that $a_0 \neq 0$ and use this to construct an inverse of $\alpha$ that lies in $M$.
    \end{enumerate}
\end{ex}

\begin{sol}
    \begin{enumerate}[label=(\roman*)]
        \item Let $n = \dim_KM$, then $\alpha^0,\alpha^1 \ldots, \alpha^n$ are linearly dependent over $K$, so there are $a_i \in K$ ($0 \leq i \leq n$), not all $a_i = 0$, such that
        \[
        \sum_{i=0}^n a_i \alpha^i = 0.
        \]
        Using those $a_i$, define the polynomial $f = \sum_{i=0}^n a_iX^i$. By choice of the $a_i$'s, we have $f \neq 0$ and
        \[
        f(\alpha) = \sum_{i=0}^n a_i \alpha^i = 0.
        \]
        \item Let $\alpha \neq 0$. We have to show that $\alpha^{-1} \in K$. Choose $0 \neq f \in K[X]$ with $f(\alpha) = 0$ such that $n= \deg f$ is as small as possible. Write $f = \sum_{i=0}^na_iX^i$, then, by minimality of $n$, $a_n \neq 0$. We show that $a_0 \neq 0$: suppose otherwise, then
        \begin{align*}
            f & = \sum_{i=1}^na_iX^i  
             = X \cdot \sum_{i=0}^{n-1}a_{i+1}X^i
             =: X \cdot g \\
            \Rightarrow 0 & = f(\alpha) = \alpha \cdot g(\alpha)\\
            \overset{\alpha \neq 0}{\Rightarrow} g(\alpha) & = 0.
        \end{align*}
        But then $g(\alpha) = 0$, and $\deg g < \deg f$, contradicting the minimality of $n$!

        Therefore, $a_0 \neq 0$, and we can rewrite
        \begin{align*}
            \sum_{i=0}^n a_i \alpha^i & = 0 \\
           \Rightarrow \sum_{i=1}^n a_i \alpha^i & = -a_0 \\
           \Rightarrow \alpha \cdot  \sum_{i=0}^{n-1} a_{i+1} \alpha^i & = -a_0 \\
           \Rightarrow \alpha^{-1} = -a_0^{-1} \cdot \sum_{i=0}^{n-1} a_{i+1} \alpha^i & \in M.
        \end{align*}
        The inclusion in $M$ follows from the fact that $M$ contains $K$ and $\alpha$ and is closed under multiplication and addition.
    \end{enumerate}
\end{sol}

\begin{ex}
\label{1.7}
Let $K$ be field.
\begin{enumerate}[label=(\roman*)]
    \item Prove that $K[X]$ is a PID.
    \item Let $I\neq \{0\}$ be an ideal of $K[X]$, then there exists a unique monic polynomial that generates $I$ as an ideal.
\end{enumerate}
    
\end{ex}
\newpage
\section{Week 2}

\begin{ex}
\label{2.1}

    Let $L/K$ be a field extension and let $\alpha, \beta\in L$ such that 
    \[
    [K(\alpha):K]=[K(\beta):K]=2.
    \]
    Assume that the characteristic of $K$ is not 2.
    \begin{enumerate}[label=(\roman*)]
    \item Prove that there is an $\alpha' \in L$ such that $K(\alpha') = K(\alpha)$ and $\alpha'^2 \in K$.
    \item Assume that $\alpha,\beta \in L$ satisfy $\alpha^2,\beta^2 \in K$. Prove that $K(\alpha)=K(\beta)$ if and only if $\frac{\alpha^2}{\beta^2}$ is a square in $K$.
    \item Prove that there is a bijective map 
    \[
    K^\times/(K^\times)^2\longrightarrow \{L\mid L/K \text{ is a field extension with } [L:K] \leq 2\}.
    \]
\end{enumerate}    
\end{ex}
\begin{sol}~
    \begin{enumerate}[label=(\roman*)]
        \item Since $[K(\alpha):K]=2$, $\alpha\not\in K$ and the 3 
        elements $1,\alpha,\alpha^2$ are linearly dependent over $K$.
        Thus there exist $a_0,a_1,a_2\in K$ not all zero such that 
        \[
        a_0+a_1\alpha+a_2\alpha^2=0.
        \]
        If $a_2=0$, then $a_0+a_1\alpha=0$, hence either $a_1=0$ or 
        $\alpha=-a_0/a_1$. 
        If $a_1=0$, the also $a_0=0$. But this is not possible 
        because $(a_0,a_1,a_2)\neq (0,0,0)$.
        On the other hand, if $\alpha=-a_0/a_1$, then $\alpha\in K$, a contradiction.

        So we have that $a_2\neq 0$ and we can divide by $a_2$,
        obtaining
        \[        b+a\alpha+\alpha^2=0, \text{ i.e. } \alpha^2+a\alpha=-b,
        \]
        where $a=a_2^{-1}a_1\in K$ and $b=a_2^{-1}a_0\in K$. 
        Since we assumed that $K$ has not characteristic two,
        we can also complete the square:
        \[
        (\alpha+a/2)^2=\alpha^2+a\alpha+a^2/4=-b+a^2/4\in K.
        \]
        Therefore $\alpha'=\alpha+a/2$
        is such that $\alpha'^2=-b+a^2/4\in K$.
        Moreover $\alpha'=\alpha+a/2\in K(\alpha)$, so $K(\alpha')\subseteq K(\alpha)$ and $\alpha=\alpha'-a/2\in K(\alpha')$ so $K(\alpha)\subseteq K(\alpha.)$. Therefore $K(\alpha)=K(\alpha')$ and $\alpha'^2\in K$.
        \item Assume that $\frac{\alpha^2}{\beta^2}$ is a square in $K$,
        i.e. there is a $k\in K$ such that
        $\frac{\alpha^2}{\beta^2}=k^2$.
        (Note that $k\neq 0$ otherwise $\alpha=0\in K$
        and $[K(\alpha):K]=[K:K]=1$.)

        Then $\alpha=k^2\beta^2$, hence $\alpha=\pm k\beta\in K(\beta)$. So $K(\alpha)\subseteq K(\beta)$.
        On the other hand, $\beta^2=\frac{\alpha^2}{k^2}$, hence $\beta=\pm \frac{\alpha}{k}\in K(\alpha)$. So $K(\beta)\subseteq K(\alpha)$.
        Having proved both inclusions we deduce that 
        $K(\alpha)=K(\beta)$.

        Vice versa, assume that 
        $K(\alpha)=K(\beta)$.
        Knowing that $[K(\alpha):K]=2$, 
        we have that $\{1,\alpha\}$ 
        is a generating set of the $K$-vector space $K(\alpha)=K(\beta)$.
        Therefore there exist $a,b\in K$ such that
        $\beta=a+b\alpha$.
        Squaring both sides of the equality, we get
        $\beta^2=a^2+2ab\alpha+\alpha^2$.
        So
        $2ab\alpha=\beta^2-a^2-\alpha^2$ is a sum of elements in $K$.
        Hence $2ab\alpha\in K$, but $\alpha\not\in K$.
        Therefore the only possibility is that $2ab=0$, i.e. (since we are not in characteristic 2)
        $ab=0$. But $b\neq 0$, otherwise $\beta=a\in K$, which is not possible (otherwise
        and $[K(\beta):K]=[K:K]=1$).
        Thus $a=0$, i.e. $\beta=b\alpha$ and so $\frac{\beta^2}{\alpha^2}=a^2$ is a square in $K$.
        \item Let $L/K$ be a field extension with $[L:K]=2$. 
        Take $\alpha\in L\setminus K$, then
        \[
        1<[K(\alpha):K]\leq [L:K]=2.
        \]
        Hence $L=K(\alpha)$.
        Moreover, the first part of this exercise allows us to choose $\alpha$ such that $\alpha^2\in K$.
        Now we can define the following map
        \[\psi:\{L\mid L/K \text{ is a field extension with } [L:K] \leq 2\}\to K^\times/(K^\times)^2\]
        as $\psi(K)=[1]$ and for
        $[L:K]=2$ as $\psi(L)=[\alpha^2]\in K^\times/(K^\times)^2$, for $\alpha\in L\setminus K$ (so $L=K(\alpha)$) such that $\alpha^2\in K$.     
        
        This map is well-defined:
        take $L/K$ of degree 2 and
        $\alpha,\beta\in L\setminus K$ such that $\alpha^2\in K$.
        Then, by the remark made at the beginning, $L=K(\alpha)=K(\beta)$ 
        As shown in the second part of this exercise, this is equivalent to $\frac{\alpha^2}{\beta^2}\in (K^\times)^2$, so $[\alpha^2]=[\beta^2]\in K^\times/(K^\times)^2$.
        (Note also that $[\alpha^2]=[1]$  if and only if $\alpha^2 =k^2\in K^2$,
        so $\alpha=\pm k\in K$ and $K(\alpha)=L$.)

        Let now prove that $\psi$ is injective.
        Assume that we have 
        two extension $L/K$ and 
        $L'/K$ of degree $\leq 2$, such that
        $[\alpha^2]=\psi(L)=\psi(L')=[\beta^2]$.
        
        If $[\beta^2]=[\alpha^2]=[1]$,
        then $L=K(\alpha)=K=K(\beta)=L'$.
        
        Otherwise, $\beta^2/\alpha^2\in (K^\times)^2$,
        so, by the previous part of this exercise,
        $L=K(\alpha)=K(\beta)=L'$.     

        Finally, for the subjectivity, let $x\in K^\times$.
        If $x\in (K^\times)^2$, then $x=\alpha^2$ for some $\alpha\in K^\times$
        and so $L=K(\alpha)$ is an extension of $K$ of degree $\leq 2$ 
        and $\psi(L)=[\alpha]$.

        If $x\not\in (K^\times)^2$, then we can find and extension $L=K(\alpha)$ such that
        $\alpha^2=x\in K^\times$ and so $\psi(L)=[x]\in K^\times/(K^\times)^2$.
        To construct this extension consider the polynomial $f(X)=X^2-x\in K[X]$.
        It has to be irreducible, otherwise $X^2-x=(aX+b)(cX+d)=acX^2+(ad+bc)X+bd$, for some
        $a,c\in K^\times$ and $b,d\in K$.
        So $1=ac$, $0=ad+bc$ and $-x=bd$, i.e. $c=a^{-1}$, $d=-a^{-1}bc=bc^2$ and $-x=bd=b^2c^2\in (K^\times)^2$, a contradiction to the assumption $x\not\in (K^\times)^2$.
        Then we can consider the field $L=K[X]/(X^2-x)$ that contains $K$
        (it is a field because the ideal $(X^2-x)$ is maximal,
        since it is generated by an irreducible polynomial).
        Defining $\alpha$ as the class of $X$ in $L$ 
        we get that $L=K(\alpha)$ and $\alpha^2=x\in K$.\qedhere
    \end{enumerate}    
\end{sol}

\begin{ex}\label{2.2}
    Let $E/K$ be a field extension and 
    $a$ and $b$ be algebraic over $K$.

    \begin{enumerate}
        \item Assume that $[K(a):K] = m$, $[K(b):K] = n$. Prove that $K[a,b] \subseteq E$ is generated, as a vector space over $K$, by the elements $a^ib^j$ ($1 \leq i \leq m$, $1 \leq j \leq n$).
        \item Prove that $a+b$ and $ab$ are algebraic over $K$. Can you estimate the quantities $[K(a+b):K]$ and $[K(ab):K]$?
        \item Find a polynomial $f \in \Q[X]$ such that $\deg(f) \leq 6$ and $f(\sqrt[3]{3} + \sqrt{5})$.
    \end{enumerate}
\end{ex}
\begin{sol}~

\begin{enumerate}
    \item We know that 
    $\{b^j\mid 1\leq j\leq n\}$ is a 
    $K$-basis of $K(b)$, so it also generates
    $K[a,b]=(K[a])[b]=(K(a))[b]$ as a $K(a)$-vector space.
    Now let $x\in K[a,b]$, then we can write it as
    \[
    x=
    \sum_{j=1}^n \alpha_j b^j,
    \]
    for some $\alpha_j\in K(a)$.
    Moreover $\{a^i\mid 1\leq i\leq m\}$ is a 
    $K$-basis of $K(a)$, hence for every $j$,
    \[
    \alpha_j=\sum_{i=1}^m k_{i,j} a^i,
    \]
    for some $k_{i,j}\in \Q$.
    Putting everything together we get
    \[
    x=
    \sum_{j=1}^n \alpha_j b^j=
    \sum_{j=1}^n \sum_{i=1}^m k_{i,j} a^i b^j,
    \]
    so $\{a^ib^j\mid 1 \leq i \leq m, 1 \leq j \leq n\}$
    generates $K[a,b]$ as a $K$ vector space.
    \item We know that $K(a)/K$ is finite and that $K(b)/K$ is finite, so also $K(a)(b)/K$ is finite.
    Therefore, considering the tower of extension 
    $K\subseteq K(a)\subseteq K(a)(b)=K(a,b)$,
    we get that $K(a,b)/K$ is a finite extension, hence 
    also algebraic.
    Therefore $a+b,ab\in K(a,b)$ are algebraic over $K$.
    \item Let $g(X)=X^3-3\in\Q[X]$. 
    Then $g(\sqrt[3]{3})=0$, hence we can use the 
    ``conjugation'' trick and consider the polynomial 
    $f(X)=g(X-\sqrt{5})g(X+\sqrt{5})$, which has 
    degree 6 and $f(\sqrt[3]{3} + \sqrt{5})=0$.
    \begin{align*}
        f(X)=&g(X-\sqrt{5})g(X+\sqrt{5})=
        ((X-\sqrt{5})^3-3)((X+\sqrt{5})^3-3)=\\
        &(X-\sqrt{5})^3(X+\sqrt{5})^3-3((X-\sqrt{5})^3+(X+\sqrt{5})^3)+9
        %&((X-\sqrt{5})(X+\sqrt{5}))^3-
        %3((X-\sqrt{5}+X+\sqrt{5})((X-\sqrt{5})^2
        %-(X-\sqrt{5})(X+\sqrt{5})+(X+\sqrt{5})^2)+9=\\
        %&((X-\sqrt{5})(X+\sqrt{5}))^3-
        %3((X-\sqrt{5}+X+\sqrt{5})((X-\sqrt{5})^2
        %-(X-\sqrt{5})(X+\sqrt{5})+(X+\sqrt{5})^2)+9=
    \end{align*}
    But we can compute 
    \begin{align*}
        (X-&\sqrt{5})^3+(X+\sqrt{5})^3=\\
        &(X-\sqrt{5}+X+\sqrt{5})((X-\sqrt{5})^2
        -(X-\sqrt{5})(X+\sqrt{5})+(X+\sqrt{5})^2)=\\
        &2X((X-\sqrt{5})(X-\sqrt{5}-X-\sqrt{5})+X^2+2\sqrt{5}X+5)=\\
        &2X(-2\sqrt{5}(X-\sqrt{5})+X^2+2\sqrt{5}X+5)=\\
        &2X(10+X^2+5)=2X^3+30X.
    \end{align*}
    Therefore 
    \begin{align*}
        f(X)=&
        (X^2-5)^3-3(2X^3+30X)+9=\\
        &X^6-15X^4+75X^2-125-6X^3-90X+9=\\
        &X^6-15X^4-6X^3+75X^2-90X-116.\qedhere
    \end{align*}
\end{enumerate}   
\end{sol}

The following lemma can be used, without proof, in the following exercise.

\begin{lem*}[Gauss' Lemma]
    Let $A$ be a unique factorization domain and $K$ be its fraction field.
    A non-constant polynomial $f\in A[X]$ is irreducible if and only if is primitive and is irreducible in $K[X]$.
\end{lem*}

\begin{ex} [Eisenstein's irreducibility criterion]\label{2.3}
    Let $A$ be a unique factorization domain and $K$ be its fraction field.
    Let $f=\sum_{i=0}^n a_iX^i\in K[X]$ be a polynomial of degree $n>0$. 
    Assume that there exists a prime element $p\in A$ such that
    $p\mid a_i$ for all $i\in\{0,1,\dots,n-1\}$, $p\nmid a_n$ and
    $p^2\nmid a_0$. Then $f$ is irreducible in $K[X]$. 
\end{ex}



\begin{ex}
\label{2.4}
Let $\zeta\in\C$ be a primitive cubic root of one.
Set $E=\Q[\sqrt[3]{2}]$, $F=\Q(\zeta)$ and $L=\Q[i]$.
\begin{enumerate}[label=(\roman*)]
    \item Prove that $[E:\Q]=3$ and $[F:\Q]=2$ and compute the minimal polynomial of $\zeta$ over $\Q$ and over $L$.
    \item Prove that $EF=\Q(\sqrt[3]{2},\zeta)$.
    \item Compute $[EF:\Q]$ and $[E\cap F:\Q]$.
\end{enumerate}  
\end{ex}
\begin{sol}
    See \cref{3.4}
\end{sol}

\begin{ex}\label{2.5}
    Let $E/K$ be a field extension and let $L/K$ and $M/K$ be subextensions.
    \begin{enumerate}[label=(\roman*)]
    \item Prove that $[LM : K ] \cdot [L \cap M : K] \leq [L:K] \cdot [M:K]$.
    \item Can you find examples where $[LM : K ] \cdot [L \cap M : K] < [L:K] \cdot [M:K]$?
    
    \noindent \textit{Hint:} Use two different roots of the polynomial $X^3 -2$.
    \end{enumerate}  
\end{ex}
\begin{sol}
    See \cref{3.5}
\end{sol}

\begin{ex}
\label{2.6}
    Let $E/K$ be a field extension and let $f \in K[X]$ be a polynomial such that $f$ factorizes in $E[X]$ as $f = \prod_{i=1}^n(x - \alpha_i)$. Prove by induction that $[K(\alpha_1,\alpha_2, \ldots, \alpha_n):K] \leq n!$.
\end{ex}
\begin{sol}
    See \cref{3.6}
\end{sol}

\newpage

\section{Week 3}


\begin{ex}
\label{3.1}
    
    For every polynomial $p(X)=\sum_{i=0}^n a_iX^i\in K[X]$ of degree $n$,
    define its \emph{reciprocal polynomial} as
    \[
    \widehat{p}(X)=\sum_{i=0}^n a_{n-i}X^i.
    \]
    Let $p(X),q(X)\in K[X]$ be polynomials of degree $n$ and $m$ respectively such that $p(0)\neq 0$ and $q(0)\neq 0$. Prove that
    \begin{enumerate}[label=(\roman*)]
    \item $\widehat{p}(X)=X^np(1/X)$ in $K(X)$,
    \item $\widehat{ \widehat{p}}(X)=p(X)$,
    \item $\widehat{pq}(X)=\widehat{p}(X)\widehat{q}(X)$,
    \item $\widehat{p}(X)$ is irreducible if and only if $p(X)$ is irreducible.
    \end{enumerate}
\end{ex}

\begin{sol}
Suppose that $p(X)=\sum_{i=0}^n a_iX^i$ has degree $n$ and $q(X)=\sum_{i=0}^m b_iX^i$ has degree $m$.
    \begin{enumerate}[label=(\roman*)]
    \item 
    $X^np(1/X)=X^n\sum_{i=0}^n a_i(1/X)^i=\sum_{i=0}^n a_iX^{n-i}=\sum_{j=0}^n a_{n-j}X^j=\widehat{p}(X)$
    \item Since $p(0)\neq 0$, $a_0\neq 0$, therefore $\deg(\widehat{p})=\deg(p)=n$. Hence, using the previous part of this exercise,
    \[
    \widehat{\widehat{p}}(X)=X^n\widehat{p}(1/X)=X^n(1/X)^np(1/(1/X))=p(X).
    \]    
    \item Since $p(0)\neq 0$ and $q(0)\neq 0$ we deduce that also $pq(0)\neq 0$.
    Hence we can apply the previous results for $p,q$ and $pq$. 
    So, using also that $pq(1/X)=p(1/X)q(1/X)$ and that $\deg{pq}=n+m$, we have that
    \[
    \widehat{pq}(X)=X^{n+m}pq(1/X)=X^np(1/X)X^mq(1/X)=\widehat{p}(X)\widehat{q}(X).
    \]
    \item Suppose $\widehat{p}(X)$ is irreducible and let $p=q_1q_2$ for
    $q_1(X),q_2(X)\in K[X]$.
    Note that $0\neq p(0)=q_1(0)q_2(0)$ implies that both $q_1(0)\neq 0$ and $q_2(0)\neq 0$.
    Considering now the reciprocal polynomials and using the previous properties,
    $\widehat{p}(X)=\widehat{q_1}(X)\widehat{q_2}(X)$.
    Since $\widehat{p}(X)$ is irreducible, there is $i\in\{1,2\}$ such that $\widehat{q_i}$ 
    is a constant, i.e. $\deg{q_i}=\deg{\widehat{q_i}}=0$.
    Thus, $q_i$ is a constant too, and therefore $p(X)$ is irreducible.

    Vice versa, assume that $p(X)$ is irreducible. Since we know that $p(X)=\widehat{\widehat{p}}$, by the property just proved, we obtain that $\widehat{p}$ 
    is also irreducible.\qedhere
    \end{enumerate}
\end{sol}



\begin{ex}
\label{3.2}
    Let $E/K$ be a field extension and $x\in E$ be an algebraic element and let $f=f(x,K)$ is the minimal polynomial of $x$ over $K$ of degree $\deg(f)=n$.
    \begin{enumerate}[label=(\roman*)]
    \item Prove that $[K(x):K]=n$.
    \item Prove that $\frac{1}{f(0)}\widehat{f}$ is the minimal polynomial of $1/x$ over $K$.
    \item Write $f(X)=\sum_{i=0}^na_iX^i=p(X^2)+Xd(X^2)$, where 
    \[
    p(X)=\sum_{j=0}^{\lfloor n/2\rfloor } a_{2j}X^{j}\text { and }
    d(X)=\sum_{j=0}^{\lfloor n/2\rfloor } a_{2j+1}X^j.
    \]
    Let $g(X)=p(X)^2-Xd(X)^2$ and prove that
    \begin{itemize}
        \item if $d(x^2)=0$, then the minimal polynomial of $x^2$ over $K$ is $p(X)$,
        \item if $d(x^2)\neq 0$, then the minimal polynomial of $x^2$ over $K$ is $(-1)^ng(X)$.
    \end{itemize}
    \noindent
    \end{enumerate}
\end{ex}
\begin{sol}~
    \begin{enumerate}[label=(\roman*)]
    \item We will prove that $B=\{1,x,\dots, x^{n-1}\}$ is a basis of $K(x)$ as a $K$ vector space which implies that $[K(x):K]=\dim_K(K(x))=|B|=n$.
    \begin{itemize}
        \item[-] \textit{$B$ is a generating set for $K(x)$ as a $K$ vector space.}
        
        To prove this recall that $K(x)=K[x]$, since $x$ is algebraic over $K$.
        
        (Similar proof to $1)\Rightarrow 2)$ of Theorem 2.7 of the lecture notes.)

        Let $z\in K(x)=K[x]$, say $z=h(x)$ for some $h\in K[X]$. 
        Divide $h$ by $f$ to obtain polynomials $q,r\in K[X]$ 
        such that $h=fq+r$, where $r=0$ or $\deg r<\deg f=n$. This implies that
	\[
		z=h(x)=f(x)q(x)+r(x)=r(x).
	\]
	Moreover, we can write
        $r=\sum_{i=0}^{n-1}c_iX^i$ for some $c_0,\dots,c_{n-1}\in K$. 
        Thus $z=\sum_{i=0}^{n-1}c_ix^i\in \langle 1,x,\dots,x^{n-1}\rangle$
        and hence $K[x]$ is generated by $\{1,x,\dots,x^{n-1}\}$ as a $K$-vector space.
        \item[-] \textit{$B$ is linearly independent over $K$.}

        If $B$ is linearly dependent over $K$ then there exists a linear combination 
        $0=\sum_{i=0}^{n-1}c_ix^i$ over $K$, with not all $c_i$ equal to 0.
        Then the polynomial $h(X)=\sum_{i=0}^{n-1}c_iX^i$ is in $K[X]\setminus\{0\}$
        and has $x$ as a root.
        So 
        \[
        n-1=\deg(h)\leq \deg(f)=n,
        \]
     a contradiction.    
    \end{itemize}
    \item Fist of all we note that $f(0)\neq 0$. Otherwise we can write $f(X)=Xg(X)$ for some $g(X)\in K[X]$, but $f$ is monic and irreducible in $K[X]$, hence $g(X)=1$ and $f(X)=X$.
    Evaluating $f$ in $x$ we obtain $0=f(x)=x$, a contradiction with the hypothesis $x\neq 0$.

    Since $f(0)\neq 0$, we can use the previous exercise and obtain that $\widehat{f}$ is also irreducible and $\widehat{f}=X^nf(1/X)$.
    Hence
    \[
    \frac{1}{f(0)}\widehat{f}\left(1/x\right)=\frac{1}{f(0)}\left(1/x\right)^nf\left(\frac{1}{1/x}\right)=\frac{1}{f(0)x^n}f(x)=0.
    \]
    So we have that $\frac{1}{f(0)}\widehat{f}$ is an irreducible polynomial in $K[x]$ 
    with $1/x$ as a root.
    To prove that $\frac{1}{f(0)}\widehat{f}$ is 
    the minimal polynomial of $1/x$ over $K$ it remains to prove that it is monic.
    Looking at the definition of $\widehat{f}$ we see that its leading coefficient
    is the constant term of $f$, i.e. $f(0)$.
    Therefore the leading coefficient of $\frac{1}{f(0)}\widehat{f}$ is $\frac{1}{f(0)}f(0)=1$, hence $\frac{1}{f(0)}\widehat{f}$ is monic.
    \item 
    First of all note that 
    \begin{equation}
           0=f(x)=p(x^2)+xd(x^2), \label{eq: f(x)=0}
    \end{equation} therefore
    \[
    g(x^2)=p(x^2)^2-x^2d(x^2)^2=(p(x^2)+xd(x^2))(p(x^2)-xd(x^2))=0.
    \]
    So $x^2$ is a root of $g$ and the degree of $g$ is 
    \[
   \begin{cases}
        2\deg(p) &\text{ if }n\text{ is even}\\
        1+2\deg(d)&\text{ if }n\text{ is odd}
    \end{cases} =\begin{cases}
        2\left\lfloor \frac{n}{2}\right\rfloor &\text{ if }n\text{ is even}\\
        1+2\left\lfloor \frac{n}{2}\right\rfloor&\text{ if }n\text{ is odd}
    \end{cases} = n.
    \]
    Moreover, by the first point of this exercise and the fact that the degree is multiplicative
    \[
    n=\deg(f)=[K(x):K]=[K(x):K(x^2)][K(x^2):K].
    \]
    Hence, to compute $[K(x^2):K]$ (which is also the degree of $f(x^2,K)$,
    the minimal polynomial of $x^2$ over $K$), we need to know $[K(x):K(x^2)]$.
    Observe that $X^2-x^2$ is a polynomial in $K(x^2)[X]$ which has $x$ has a root. Thus 
    \[
    [K(x^2):K]=\deg(x,K(x^2))\leq \deg(X^2-x^2)=2
    \]
    Therefore 
    \begin{equation}\label{eq: deg of x^2}
        [K(x^2):K]=\frac{[K(x):K]}{[K(x):K(x^2)]}\in\left\{n,\frac{n}{2}\right\}.
    \end{equation}
    \begin{itemize}
        \item If $d(x^2)=0$, by Equation \eqref{eq: f(x)=0}, also $p(x^2)=0$ and $\deg(p)=\lfloor \frac{n}{2}\rfloor< n$.
        So 
        \[
        [K(x^2):K]=\deg(f(x^2,K))\leq \deg(p)< n,
        \]
        thus, by \eqref{eq: deg of x^2}, $[K(x^2):K]=\frac{n}{2}$, which implies that $n$ has to be even, $p(X)$ monic and 
        \[
        [K(x^2):K]=\frac{n}{2}=\left\lfloor \frac{n}{2}\right\rfloor=\deg(p).
        \]
        Therefore $p(X)$ is a monic polynomial in $K[X]$
        which as $x^2$ as a root and of degree $[K(x^2):K]=\deg(f(x^2,K))$,
        hence it is $\deg(f(x^2,K))$,
        the minimal polynomial of $x^2$ over $K$.
        \item If $d(x^2)\neq0$, then, by Equation \eqref{eq: f(x)=0}, 
        \[
        x=-\frac{p(x^2)}{d(x^2)}\in K(x^2).
        \]
        Therefore $K(x)\subseteq K(x^2)\subseteq K(x)$ and so $K(x^2)=K(x)$,
        which means $[K(x):K(x^2)]=1$ and, by \eqref{eq: deg of x^2} ,
        \[
        [K(x^2):K]=\frac{[K(x):K]}{[K(x):K(x^2)]}=[K(x):K]=n=\deg(g).
        \]
        Moreover the leading coefficient of $g(X)$ is
        \[
        \begin{cases}
        a_n &\text{ if }n\text{ is even}\\
        -a_n&\text{ if }n\text{ is odd}
        \end{cases}=(-1)^na_n=(-1)^n .
        \]
        Therefore we have the monic polynomial  
        $(-1)^ng(X)\in K[X]$ that vanishes in $x^2$,
        of degree $n=[K(x^2):K]=\deg(f(x^2,K))$. Hence $(-1)^ng(X)=f(x^2,K)$.\qedhere
    \end{itemize}
    \end{enumerate}
\end{sol}



The following lemma can be used, without proof, in the following exercise.

\begin{lem*}[Gauss' Lemma]
    Let $A$ be a unique factorization domain and $K$ be its fraction field.
    A non-constant polynomial $f\in A[X]$ is irreducible if and only if it is primitive and irreducible in $K[X]$.
\end{lem*}

\begin{ex} [Eisenstein's irreducibility criterion]
\label{3.3}
    Let $f=\sum_{i=0}^n a_iX^i\in \Z[X]$ be a polynomial of degree $n>0$. 
    Assume that there exists a prime $p$ such that
    $p\mid a_i$ for all $i\in\{0,1,\dots,n-1\}$, $p\nmid a_n$ and
    $p^2\nmid a_0$. Then $f$ is irreducible in $\Q[X]$.

    \vspace{.5cm}
    
    \textit{More general version}
    
    Let $A$ be a unique factorization domain and $K$ be its fraction field.
    Let $f=\sum_{i=0}^n a_iX^i\in A[X]$ be a polynomial of degree $n>0$. 
    Assume that there exists a prime element $p\in A$ such that
    $p\mid a_i$ for all $i\in\{0,1,\dots,n-1\}$, $p\nmid a_n$ and
    $p^2\nmid a_0$. Then $f$ is irreducible in $K[X]$. 
\end{ex}
\begin{sol}
    Suppose that $f$ is reducible in $K[X]$.
    Then $g=c^{-1}f$, where $c$ is the content of $f$,
    would be reducible and primitive.
    Hence, by Gauss' Lemma, $g$ is also 
    reducible in $A[X]$.
    So $c^{-1}f=g=hl$, for some non-constant polynomials $h,l\in A[X]$.
    Now consider $\pi:A\to A/(p)$, $a\mapsto \overline{a}$ the natural projection.
    We know that $\overline{a_i}=0$ for all 
    $i\in\{0,1,\dots, n-1\}$ and $\overline{a_n}\neq 0$.
    Therefore 
    \[
    \overline{\pi}(ch)\overline{\pi}(l)=\overline{c}\:\overline{\pi}(h)\overline{\pi}(l)=\overline{\pi}(f)=\overline{a_n}X^n\in  A/(p)[X].
    \]
    But $A/(p)[X]$ is a UFD so the only possibility
    is that $\overline{\pi}(ch)=\overline{d}X^t$ and $\overline{\pi}(l)=\overline{f}X^s$, for
    some $f,d\in A/(p)\setminus\{\overline{0}\}$
    and $t,s\in\{1,\dots, n-1\}$.
    In particular, $\overline{\pi}(ch)$ and $\overline{\pi}(l)$ have both constant term 
    equal to 0.
    Hence $p$ divides $ch(0)$ and $l(0)$ in $A$.
    Therefore $p^2$ divides $ch(0)l(0)=f(0)$,
    a contradiction.
\end{sol}




\begin{ex}
\label{3.4}
Let $\zeta\in\C$ be a primitive cubic root of one.
Set $E=\Q[\sqrt[3]{2}]$, $F=\Q(\zeta)$ and $L=\Q[i]$.
\begin{enumerate}[label=(\roman*)]
    \item Prove that $[E:\Q]=3$ and $[F:\Q]=2$ and compute the minimal polynomial of $\zeta$ over $\Q$ and over $L$.
    \item Prove that $EF=\Q(\sqrt[3]{2},\zeta)$.
    \item Compute $[EF:\Q]$ and $[E\cap F:\Q]$.
\end{enumerate}  
\end{ex}
\begin{sol}~
 \begin{enumerate}[label=(\roman*)]
    \item We know that $[E:\Q]$ is the same as
    the degree of the minimal polynomial of $\sqrt[3]{2}$
    over $\Q$.
    Clearly $\sqrt[3]{2}$ is a root of $X^3-2\in\Q[X]$.
    Moreover, by Eisenstein's criterion (\cref{3.3}) with
    $p=2$, we get that $X^3-2$ is irreducible in $\Q[X]$.
    Hence $f(\sqrt[3]{2},\Q)=X^3-2$ and $[E:\Q]=3$.

    We also know that $\xi$ is a 
    (non-rational) root of $X^3-1=(X-1)(X^2+X+1)$.
    But $X^2+X+1$ has no rational roots and it has degree
    2, so it is irreducible in $\Q[X]$ and it is
    the minimal polynomial of $\xi$ over $\Q$.
    Therefore 
    $$[F:\Q]=\deg(f(\xi,\Q))\deg(X^2+X+1)=2.$$
    \item By definition $EF=\Q(E\cup F)$, which clearly 
    $EF$ contains $\Q, \sqrt[3]{2}$ and $ \xi$, so 
    $EF\supseteq \Q(\sqrt[3]{2},\xi)$.
    Moreover, every element $x$ in $EF=\Q(E\cup F)$ 
    is a $\Q$-linear combination
    \[
    x=\sum_{i=0}^n a_i e_i +\sum_{j=0}^m b_j f_j,
    \]
    where $a_i,b_j\in\Q$, $e_i\in E$ and $f_j\in F$
    for all $i,j$.
    In addition, $E$ is a $\Q$-vector space with basis
    $\{1,\sqrt[3]{2},\sqrt[3]{4}\}$, so for every $i$,
    \[
    e_i=e_{i,1}+e_{i,2}\sqrt[3]{2}+e_{i,3}\sqrt[3]{4},
    \]
    for some $e_{i,1}, e_{i,2}, e_{i,3}\in\Q$.    
    On the other hand, $F$ is $\Q$-vector space with 
    basis $1, \xi, \xi^2$, so for every $j$,
    \[
    f_j=f_{j,1}+f_{j,2}\xi,
    \]
    for some $f_{j,1}, f_{j,2}\in\Q$.
    Finally, 
    
    $$x=\sum_{i=0}^n a_i \big(e_{i,1}+e_{i,2}\sqrt[3]{2}+e_{i,3}\sqrt[3]{4}\big) +\sum_{j=0}^m b_j \big(f_{j,1}+f_{j,2}\xi\big)\in \Q(\sqrt[3]{2},\xi).$$
    Thus we also have the other inclusion $EF\subseteq \Q(\sqrt[3]{2},\xi)$.
    \item We are in the following situation:
    \
    \begin{center}
        \begin{tikzcd}
	& {EF=\Q(\sqrt[3]{2},\xi)}\arrow[rd,no head]\\
	{E=\Q(\sqrt[3]{2})}\arrow[ru,no head,"\leq 2"]&& {F=\Q(\xi)}\arrow[ldd,no head,"2"]  \\
	& {E\cap F}\arrow[lu,no head] \arrow[ru,no head,"\leq 2"]\\
	& {\Q}\arrow[luu,no head,"3"]\arrow[u,no head]
 \end{tikzcd}
    \end{center}
    so on the one hand we know that
    \[
    [EF:\Q]=[EF:E][E:\Q]=[EF:E]2
    \]
    and on the other hand, we know that
    \[
    [EF:\Q]=[EF:F][F:\Q]=[EF:F]3\leq 2\cdot 3=6.
    \]
    Therefore $2$ and $3$ divide $[EF:\Q]\leq 6$.
    Hence the only possibility is that $[EF:\Q]=6$.

    %The fact that $[EF:\Q]=6$ also implies that 
    %$[EF:F]=3$ and $[EF:E]=2$.
    The intersection $E\cap F$ is contained in $E$,
    which is contained in $\R$, while $\xi\notin\R$.
    Therefore $[F:E\cap F]>1$, since $\xi\in F\setminus E\cap F$  but we also know that $[F:E\cap F]\leq [F:\Q]=2$.
    Hence $[F:E\cap F]=2$, which means that $E\cap F=\Q$
    and so $[E\cap F:\Q]=1$.\qedhere    
\end{enumerate}
\end{sol}

\begin{ex}
\label{3.5}
    Let $E/K$ be a field extension and let $L/K$ and $M/K$ be subextensions.
    \begin{enumerate}[label=(\roman*)]
    \item Prove that $[LM : K ] \cdot [L \cap M : K] \leq [L:K] \cdot [M:K]$.
    \item Can you find examples where $[LM : K ] \cdot [L \cap M : K] < [L:K] \cdot [M:K]$?
    
    \noindent \textit{Hint:} Use two different roots of the polynomial $X^3 -2$.
    \end{enumerate}  
\end{ex}

\begin{sol}
    \begin{enumerate}[label=(\roman*)]
    \item Considering the extensions $L/(L \cap M)$, $M/(L \cap M)$, \cref{2.2} implies that
    \[
    [LM:L \cap M] \leq [L:L \cap M] \cdot [M:L \cap M].
    \]
    Multiplying by $[L \cap M:K]^2$, we get
    \begin{align*}
        [LM:L \cap M] \cdot [L \cap M:K] \cdot [L \cap M:K] & \leq [L:L \cap M] \cdot [L \cap M:K] \cdot [M:L \cap M] \cdot [L \cap M:K] \\
        \Rightarrow [LM:K] \cdot [L \cap M:K] & \leq [L:K] \cdot [M:K].
    \end{align*}
    \item We consider $\alpha = \sqrt[3]{2}$ and $\beta = \omega \sqrt[3]{2}$, where $\omega \neq 1$ is a primitive root of unity. $\alpha$ and $\beta$ are different roots of the irreducible polynomial $X^3-2$. As $f(\alpha,\Q) = f(\beta,\Q) = 3$, it follows that $[\Q(\alpha):\Q] = [\Q(\beta):\Q] = 3$.

    Note that
    \[
    E := \Q(\alpha)\cdot \Q(\beta) = \Q(\alpha,\beta) = \Q(\omega, \sqrt[3]{2}).
    \]
    Expressing $\omega = -\frac{1}{2} \pm \frac{i}{2}\sqrt{3}$, we see that $[\Q(\omega):\Q] = 2$. On the other hand, $[\Q(\sqrt[3]{2}):\Q] = 3$. As $\Q(\sqrt[3]{2}),\Q(\sqrt[3]{2}) \subseteq E$, it follows that $2,3$ divide $[E:\Q]$ which implies that $6$ divides $[E:\Q]$. On the other hand, as
    \[
    [E:\Q] \leq [\Q(\omega):\Q] \cdot [\Q(\sqrt[3]{2}):\Q] = 2 \cdot 3 = 6,
    \]
    we infer that $[E:\Q] = 6$. Now considering $\Q(\alpha),\Q(\beta)$, we get $\Q(\alpha) \cap \Q(\beta) = \Q$ as $\Q(\alpha) \neq \Q(\beta)$ and $[\Q(\alpha) \cap \Q(\beta):\Q]$ must be a proper divisor of $[\Q(\alpha):\Q] = 3$, and therefore be $1$.

    Putting everything together, we get:
    \[
    [\Q(\alpha)\Q(\beta):\Q] \cdot [\Q(\alpha) \cap \Q(\beta): \Q] = 6 \cdot 1 < 3 \cdot 3 = [\Q(\alpha):\Q] \cdot [\Q(\beta):\Q].\qedhere
    \]
    \end{enumerate}
\end{sol}

\begin{ex}
\label{3.6}
    Let $E/K$ be a field extension and let $f \in K[X]$ be a polynomial such that $f$ factorizes in $E[X]$ as $f = \prod_{i=1}^n(x - \alpha_i)$. Prove by induction that $[K(\alpha_1,\alpha_2, \ldots, \alpha_n):K] \leq n!$.
\end{ex}

\begin{sol}

    For sake of simplicity, we assume that $f$ is monic.
    
    If $n = 1$, then $f$ is a  $X - \alpha$ with $\alpha \in K$. In this case, $K(\alpha) = K$ which implies
    \[
    [K(\alpha):K] = [K:K] = 1 = 1!.
    \]
    Suppose now that the statement has been proven true over each field $K^{\prime}$ and each polynomial $g \in K^{\prime}[X]$ with $\deg g = n$. Let now be $f \in K[X]$ with
    \[
    f = \prod_{i=1}^{n+1 (X - \alpha_i)}
    \]
    and consider the extension $K(\alpha_{n+1})/K$. As $f(\alpha_{n+1},K)$ divides $f$, we see that
    \[
    [K(\alpha_{n+1}):K] = \deg f(\alpha_{n+1},K) \leq \deg f = n+1.
    \]
    Furthermore,
    \[
    g = \prod_{i=1}^{n} (X-\alpha_i) = \frac{f}{X-\alpha_{n+1}} \in K(\alpha_{n+1})[X],
    \]
    because both numerator as denominator are in $K(\alpha_{n+1})[X]$. As $\deg g = n$, the induction hypothesis tells us that
    \[
    [K(\alpha_1,\alpha_2, \ldots, \alpha_{n+1}):K(\alpha_{n+1})] = [K(\alpha_{n+1}(\alpha_1,\alpha_2, \ldots, \alpha_n)):K(\alpha_{n+1})] \leq n!.
    \]
    And it follows that
    \begin{align*}
    [K(\alpha_1,\alpha_2, \ldots, \alpha_{n+1}):K] & = [K(\alpha_1,\alpha_2, \ldots, \alpha_{n+1}):K(\alpha_{n+1})] \cdot [K(\alpha_{n+1}):K] \\
    & \leq n! \cdot (n+1) = (n+1)!.
    \end{align*}
    This implies that the statement is also true in case that $\deg f = n+1$. By the induction principle, the statement is proven in general.
\end{sol}

\newpage
\section{Week 4}


\begin{ex}
\label{4.1}
    Let $K$ be a field. For a polynomial $f = \sum_{k=0}^na_kX^k \in K[X]$, we define the \emph{derivative} by
    \[
    f' = \sum_{k=1}^n ka_kX^{k-1}.
    \]
    \begin{enumerate}[label=(\roman*)]
        \item Let $\alpha \in K$ and $f,g \in K[X]$. Prove the following properties of the derivative
        \begin{enumerate}
            \item $(f + g)' = f' + g'$,
            \item $(\alpha \cdot f)' = \alpha \cdot f'$,
            \item $(f\cdot g)' = f'\cdot g + f \cdot g'$.
        \end{enumerate}
        \item Let $f \in K[X]$ be a polynomial that factorizes as $f = \prod_{i=1}^n (X - \alpha_i)$. Prove that the roots $\alpha_1, \ldots, \alpha_n$ are pairwise different if and only if $\gcd(f,f') = 1$.
        \item Let $C$ be an algebraic closure of $K$ and let $f \in K[X]$ be a polynomial with $\deg(f) \geq 1$ that is irreducible over $K$. Prove that $f$ has repeated roots in $C$ if and only if $f' = 0$.  In particular, show that 
        having such a polynomial $f$
        implies that $\mathrm{char}(K) = p > 0$ and $f(X) = g(X^p)$ for some irreducible polynomial $g \in K[X]$.
    \end{enumerate}
\end{ex}

\begin{sol}~
 \begin{enumerate}[label=(\roman*)]
        \item For sake of simplicity, write $f = \sum_{k=0}^{\infty} a_k X^k$ and $g = \sum_{k=0}^{\infty} b_kX^k$. Then
        %\begin{enumerate}
            %\item 
            \begin{align*}
            (f + g)' & 
            =  \left(\sum_{k=0}^{\infty} (a_k + b_k) X^k \right)' 
            = \sum_{k=1}^{\infty} k (a_k+b_k)X^{k-1} \\
            & = \sum_{k=1}^{\infty} ka_kX^{k-1} + \sum_{k=1}^{\infty} kb_kX^{k-1} = f' + g'.
        \end{align*}
        Hence we have proved (a).
        We can now prove also (b) as:
        %\item
        \[
        (\alpha \cdot f)'
        = \left(\sum_{k=0}^{\infty} \alpha a_k X^k \right)
        = \sum_{k=1}^{\infty} k \alpha a_kX^{k-1} 
        = \alpha \cdot \sum_{k=1}^{\infty} k a_kX^{k-1}
        = \alpha \cdot f'.
        \]
        % \begin{align*}
        %     (\alpha \cdot f)' & = \left(\sum_{k=0}^{\infty} \alpha a_k X^k \right) \\
        %     & = \sum_{k=1}^{\infty} k \alpha a_kX^{k-1} \\
        %     & = \alpha \cdot \sum_{k=1}^{\infty} k a_kX^{k-1} \\
        %     & = \alpha \cdot f'.
        % \end{align*}
        %\item 
        To prove (c), we first check the equality for $f = X^k$ and  $g = X^l$:
        \[
        (f \cdot g)' = (X^{k+l})' = (k+l)X^{k+l-1} = kX^{k-1}X^l + X^klX^{l-1} = f' \cdot g + f \cdot g'.
        \]
        Using the already-established $K$-linearity (i.e. (a) and (b) of this exercise), we can now calculate
        \begin{align*}
            (f \cdot g)' & 
            = \left( \sum_{k,l \geq 0} a_kb_l X^{k+l} \right)' 
            = \sum_{k,l \geq 0} a_kb_l (X^{k+l})'\\
            &= \sum_{k,l \geq 0} a_kb_l ((X^k)'X^l + X^k(X^l)') \\
            & = \sum_{k,l \geq 0} a_kb_l (X^k)'X^l  + \sum_{k,l \geq 0} a_kb_l X^k(X^l)' \\
            & = \left( \sum_{k=0}^{\infty} a_kX^k \right) \cdot \left( \sum_{l=0}^{\infty} b_l(X^l)' \right) + \left( \sum_{k=0}^{\infty} a_k(X^k)' \right) \cdot \left( \sum_{l=0}^{\infty} b_lX^l \right) \\
            & = f' \cdot g + f \cdot g'.
        \end{align*}
        %\end{enumerate}

        \item Let $\alpha_i$ be one of the roots of $f$. Write $f = (X - \alpha_i) \cdot g$.
        By the (i) of this exercise,
        \[
        f' = (X - \alpha_i)' \cdot g + (X - \alpha_i) \cdot g' = g + (X - \alpha_i) \cdot g'.
        \]
        Therefore, $f'(\alpha_i) = g(\alpha_i) = \prod_{j \neq i} (\alpha_i - \alpha_j)$. This implies that $f'(\alpha_i) = 0$ if and only if $\alpha_i$ is a repeated root of $f$ which is the case if and only if $(X - \alpha_i)$ is a common divisor of $f$ and $f'$.

        As the linear factors $(X-\alpha_i)$ are prime elements of $K[X]$, it follows that $f$ and $f'$ have common divisors if and only if $f$ has repeated roots.

        \item Let $f$ be irreducible in $K[X]$ with repeated roots in $C[X]$. By the previous exercise, $\gcd(f,f') \neq 1$. As $f$ is irreducible and $\gcd(f,f') \in K[X]$, this implies $\gcd(f,f') = f$. Therefore, $f$ divides $f'$. As $\deg(f') < \deg(f)$, this implies $f' = 0$.

        In case that $f' = 0$, we write $f = \sum_{k=0}^{\infty}a_kX^k$ and consider that
        \[
        f' = \sum_{k=1}^{\infty} ka_kX^{k-1} = 0.
        \]
        Therefore, for all $k \geq 0$, we have $k = 0$ or $a_k = 0$. As there is at least one $k \geq 1$ with $a_k$ we conclude that $k = 0$ holds in $K$ for some nonzero $k$. This implies that  $\mathrm{char}(K) = p > 0$ and that $a_k = 0$ whenever $p \nmid k$. We can therefore write
        \[
        f = \sum_{l = 0}^{\infty} a_{pl}X^{pl} = g(X^p)
        \]
        with $g = \sum_{l=0}^{\infty} a_{pl}X^l$. For a decomposition $g = g_1g_2$, we  also get a decomposition $f(X) = g(X^p) = g_1(X^p)g_2(X^p)$ which implies that either $g_1(X^p)$ or $g_2(X^p)$ is in $K$, which amounts to saying that $g_1$ or $g_2$ is in $K$. Therefore, $g$ has to be irreducible.\qedhere
    \end{enumerate}
    
\end{sol}

\begin{ex}
\label{4.2}
    Let $L$ be a finite field.
    \begin{enumerate}[label=(\roman*)]
    \item Show that $L$ is not algebraically closed.
    
    \noindent \textit{Hint:} Consider the polynomial $f(X)=1+\prod_{l\in L} (X-l)\in L[X]$.
    \item Show that $L$ contains a 
    subfield $K$ isomorphic to $\Z/p$
    (its ring of integers) and that $|L|=p^m$, where $m=[L:K]$.
    \end{enumerate}
    Assume now that $K=\Z/p$ and let $f(X)=X^{p^m}-X\in K[X]$.
    Let $C$ be an algebraic closure of $K$ and set 
    $L=\{\alpha\in C\mid f(\alpha)=0\}$.
    Prove that
    \begin{enumerate}
        \item[(iii)] $|L|=p^m$

        \noindent\textit{Hint:} Use the previous exercise.
        \item[(iv)] Recall that, since $K$ has characteristic $p$, $\Phi: K \to K$; $x \mapsto x^p$ is a field endomorphism (the Frobenius endomorphism).
        Prove that $L$ is a field and $K\subseteq L\subseteq C$.
    \end{enumerate}
\end{ex}
\begin{sol}~
\begin{enumerate}[label=(\roman*)]
    \item The polynomial $f(X)\in L[X]$ doesn't have roots in $L$.
    In fact consider any $a\in L$, then 
    \[
    f(a)=1+\prod_{l\in L} (a-l)=1+(a-a)\prod_{l\in L\setminus \{a\}} (X-l)=1+0=1\neq 0.
    \]
    \item Let $K$ be the ring of integers of $L$. 
    Since $L$ is finite, $K$ is finite too. 
    Hence $K$ is isomorphic to $\Z/p$ for some prime $p$.
    $L$ is a vector space of dimension $m$ over $K\cong\Z/p$.
    Let $\{x_1,x_2,\dots,x_m\}$ is a 
    basis of $L$ over $K$,
    then every element of $L$ can be 
    written in a unique way as a linear 
    combination $\sum_{i=1}^m a_ix_i$, 
    with $a_i\in K$.
    So the number of elements of $L$ is
    equal to the number of tuples
    $(a_1,\dots,a_m)\in K^m$.
    Hence $|L|=|K|^m=p^m$.
    \item If $f=X^{p^m}-X$, then $f'=p^mX^{p^m-1}-1=-1$. 
    So, $ \gcd(f,f')=1$ and, by the previous exercise, we can conclude 
    that $f$ has $p^m$ pairwise different roots in $C$, i.e. $|L|=p^m$.
    \item Using the Frobenius endomorphism $\Phi $ of $K$, we can see that
    \[
    L=\{\alpha\in C\mid \alpha^{p^m}=\alpha\}=\{\alpha\in C\mid \Phi^m(\alpha)=\alpha\}.
    \]
    Since $\Phi$ is an endomorphism, $\Phi^m$ is also an endomorphism.
    Then for all $\alpha,\beta\in L$,
    \[
    \Phi^m(\alpha+\beta)=\Phi^m(\alpha)+\Phi^m(\beta)=\alpha+\beta
    \]
    \[
    \Phi^m(-\alpha)=-\Phi(\alpha)=-\alpha
    \]
    \[
    \Phi^m(\alpha\beta)=\Phi^m(\alpha)\Phi^m(\beta)=\alpha\beta.
    \]
    \[
    \Phi^m(\alpha^{-1})=\big(\Phi^m(\alpha)\big)^{-1}=\alpha^{-1}.
    \]
    So $\alpha+\beta,\-\alpha,\alpha\beta,\alpha^{-1}\in L$, i.e.
    $L$ is a field.\qedhere
\end{enumerate}
\end{sol}

\begin{ex}
\label{4.3}
Let $C/K$ be an algebraic field extension.
Show that the following are equivalent:
\begin{enumerate}[label=(\roman*)]
    \item $C$ is an algebraic closure of $K$.
    \item For every algebraic extension $L/K$ there is an extension homomorphism $\varphi\in \Hom(L/K,C/K)$.
\end{enumerate}  
\noindent \textit{Hint:} For (i)$\Rightarrow$(ii) use Proposition 3.6 in the notes.
\end{ex}





\begin{ex}
\label{4.4}
    Let $f\in K[X]$ be a polynomial of degree $n$.
    Let $C$ be an algebraic closure of $K$ and $A=\{\alpha_1,\dots\alpha_k\}\subseteq C$ be the distinct roots of $f$ in $C.$
    We know that $E=K(\alpha_1,\dots,\alpha_k)$ is the decomposition field of $f$ over $K$.
    \begin{enumerate}[label=(\roman*)]
    \item Prove that $[E:K]\leq n! $.
    \item Prove that there is an injective homomorphism $\Gal(E/K)\longrightarrow \Sym_{A}\cong \Sym_k$ 
    
    \noindent\textit{Hint: } Prove that $\sigma(A)=A$ for every $\sigma \in \Gal(E/K)$.
    \item For $K=\Z/3$ and $f=X^3-X-1$, compute $[E:K]$ and $\Gal(E/K)$.
    (Observe that in this case $[E:K]<n!$.)
    \end{enumerate}
\end{ex}
\begin{sol}
    See \cref{5.2}
\end{sol}



\begin{ex}
\label{4.5}
    Let $f=X^4-5X^2+5\in \Q[X]$ and $E$ be a decomposition field of $f$ over $\Q$. 
    Prove that $[E:\Q]=4$. 

    \textit{Hint:}
    Given $\alpha,\beta\in \C$
    two solutions of $f$ such that $\beta\neq-\alpha$,
    compute $\alpha\beta$.
    Prove also that $E=\Q(\alpha)$.
\end{ex}
\begin{sol}
    See \cref{5.3}
\end{sol}



\newpage

\section{Week 5}


\begin{ex}
\label{5.1}
    Let $\xi\in\C$ be a primitive cubic root of one.
    Prove that the extension $\Q(\sqrt[3]{2})/\Q$ is not normal and that $\Q(\sqrt[3]{2},\xi)/\Q$ is normal.
\end{ex}
\begin{sol}
Let $G=\Gal(\Q(\sqrt[3]{2},\Q))$.
By Proposition 4.10 of the lecture notes we know that
$y\in O_G(\sqrt[3]{2})$ if and only if $y$ and 
$\sqrt[3]{2}$ have the same minimal polynomial over $\Q$.
So we need to compute the minimal polynomial of $\sqrt[3]{2}$ over $\Q$.

First of all, note that $\sqrt[3]{2}$ is a root of
the polynomial $f(X)=X^3-2$.
%But using Eisenstein criterion with $p=3$ we see that
%$f$ is irreducible over $\Q$.
%Hence $f$ is the minimal polynomial of $\sqrt[3]{2}$ over $\Q$.

The roots of $f$ are $\sqrt[3]{2},\sqrt[3]{2}\xi$ and $\sqrt[3]{2}\xi^2$ which are all not in $\Q$.
So, being of degree 3 and not having rational roots, $f$ is irreducible over $\Q$.

Thus, by Proposition 4.10, 
\[
O_G\big(\sqrt[3]{2}\big)=\{\sqrt[3]{2},\sqrt[3]{2}\xi,\sqrt[3]{2},\xi^2\}.
\]
This implies that there exists $\sigma\in \Hom(\C/\Q,\C/\Q)$ such that $\sigma(\sqrt[3]{2})=\sqrt[3]{2}\xi$.
But $\sqrt[3]{2}\xi\not \in \Q(\sqrt[3]{2})$
because $\sqrt[3]{2}\xi\in\C\setminus\R$ while 
$\sqrt[3]{2}\in \R$. 
Hence $\sigma(\Q(\sqrt[3]{2}))\not\subseteq \Q(\sqrt[3]{2})$ and so $\Q(\sqrt[3]{2})/\Q$ is not a normal extension.

To prove that $\Q(\sqrt[3]{2},\xi)$ is a normal extension 
we use Proposition 5.10, so it is enough to prove
that $\Q(\sqrt[3]{2},\xi)$ is the decomposition field of $f$.
We know that the decomposition field $E$ of $f$ over $\Q$ is
$\Q$ extended with the roots of $f$, i.e.
$E=\Q(\sqrt[3]{2},\sqrt[3]{2}\xi,\sqrt[3]{2}\xi^2)$.
But it's easy to see that actually 
\[
\Q(\sqrt[3]{2},\xi)=\Q(\sqrt[3]{2},\sqrt[3]{2}\xi,\sqrt[3]{2}\xi^2)=E.
\]
The inclusion $\subseteq$ is because $\sqrt[3]{2},  \xi=\frac{\sqrt[3]{2}\xi}{\sqrt[3]{2}}\in E$. 
Vice versa $\supseteq$ is due to the fact that 
the roots of $f$ are products of $\sqrt[3]{2}$ and $\xi$, elements in $\Q(\sqrt[3]{2},\xi)$.
\end{sol}

\begin{ex}
\label{5.2}
    Let $f\in K[X]$ be a polynomial of degree $n$.
    Let $C$ be an algebraic closure of $K$ and $A=\{\alpha_1,\dots\alpha_k\}\subseteq C$ be the distinct roots of $f$ in $C.$
    We know that $E=K(\alpha_1,\dots,\alpha_k)$ is the decomposition field of $f$ over $K$.
    \begin{enumerate}[label=(\roman*)]
    \item Prove that $[E:K]\leq n! $.
    \item Prove that there is an injective homomorphism $\Gal(E/K)\longrightarrow \Sym_{A}\cong \Sym_k$ 
    
    \noindent\textit{Hint: } Prove that $\sigma(A)=A$ for every $\sigma \in \Gal(E/K)$.
    \item For $K=\Z/3$ and $f=X^3-X-1$, compute $[E:K]$ and $\Gal(E/K)$.
    (Observe that in this case $[E:K]<n!$.)
    \end{enumerate}
\end{ex}
\begin{sol}~

 \begin{enumerate}[label=(\roman*)]
    \item If $n = 0$, the polynomial $f$ is a nonzero constant. Therefore, $A = \emptyset$ and $E = K$. In this case, $[E:K] = [K:K] = 1 = 0!$.

    Suppose that we have proven that $[F:L] \leq n!$ whenever $F$ is the decomposition field of a polynomial $g \in L[X]$ with $\deg(g) = n$.
    
    We assume now that $f \in K[X]$ has $\deg(f) = n+1$. Denote the decomposition field of $f$ over $K$ by $E$. Let $\alpha$ be a root of $f$.

    As $f(\alpha) = 0$ we know that $f(\alpha,K)| f$. Therefore,
   \[
    [K(\alpha):K] = \deg(f(\alpha,K)) \leq \deg(f) = n+1.
    \]
    As $\alpha \in K(\alpha)$, we conclude that $g = \frac{f}{X-\alpha} \in K(\alpha)[X]$. Furthermore, $E$ is the decomposition field of $g$ over $K$: if $A$ is the set of roots of $g$, then $A \cup \{\alpha \}$ is the set of roots of $f$. Therefore, $E = K(A \cup \{ \alpha \}) = K(\alpha)(A)$.
    
     As $\deg(g) = n$, we can apply the inductive hypothesis and infer that
     \[
     [E:K] = [E:K(\alpha)] \cdot [K(\alpha):K] \leq (n+1) \cdot n! = (n+1)!.
    \]
     \item Let $\alpha \in A$ and $\sigma \in \Gal(E/K)$, then
    \[
    f(\sigma(\alpha)) = \overline{\sigma}(f) (\sigma(\alpha)) = \sigma(f(\alpha)) = \sigma(0) = 0.
     \]
     Therefore, $\sigma(\alpha) \in A$. As a consequence, $\sigma(A) = A$ for all $\sigma \in \Gal(E/K)$.
     Therefore the restriction
     \begin{align*}
     \gamma: \Gal(E/K) & \to \Sym_A \\
     \sigma & \mapsto \sigma |_A
     \end{align*}
    is well-defined and, as the restriction of a group action, indeed a homomorphism. As $E$ is generated by $A$ over $K$, an automorphism $\sigma \in \Gal(E/K)$ is uniquely determined by its action on $A$. We conclude that $\gamma$ is injective.
     \item Let $\alpha$ be a root of $f$. As $f = X(X-1)(X+1) -1$, the fact that $\mathrm{char}(K) = 3$ implies that $\alpha + k$ is a root of $f$ for any $k \in \Z/3$. Looking at the degree, this implies that these are in fact all roots of $f$.

     Note that $f(k) = -1$ for all $k \in K$ which implies that $f$ has no roots in $K$. A reducible polynomial of degree $3$ over $K[X]$ always has roots in $K$, therefore $f$ has to be irreducible.

     It follows that $f = f(\alpha,K)$ and $[K(\alpha):K] = 3$. As all roots of $f$  are $\alpha + k \in K(\alpha)$ for $k \in K$, we conclude that $E = K(\alpha)$. Therefore $[E:K] = 3$.

    By the same argument as in the proof of Theorem 4.10, there is for each $k \in K$ a unique $\phi \in \Hom(E/K,E/K)$ with $\phi(\alpha) = \phi(\alpha+k)$. This shows that $\Gal(E/K) \cong \Z/3$.\qedhere
     \end{enumerate}
 \end{sol}

\begin{ex}
\label{5.3}
    Let $f=X^4-5X^2+5\in \Q[X]$ and $E$ be a decomposition field of $f$ over $\Q$. 
    Prove that $[E:\Q]=4$. 

    \textit{Hint:}
    Given $\alpha,\beta\in \C$
    two solutions of $f$ such that $\beta\neq-\alpha$,
    compute $\alpha\beta$.
    Prove also that $E=\Q(\alpha)$.
\end{ex}
\begin{sol}
Note that, since $f$ is an even polynomial
if $\alpha\in \C$ is a root of $f$,
then also $-\alpha$ is a root of $f$.
 Hence, given two roots $\alpha,\beta\in \C$
 such that $\beta\neq-\alpha$,
we have that $E=\Q(\alpha,-\alpha,\beta,-\beta)$.
 But $-\alpha,-\beta\in \Q(\alpha,\beta)\subseteq E$
 and so 
\[
 E=\Q(\alpha,-\alpha,\beta,-\beta)\subseteq \Q(\alpha,\beta)\subseteq\Q(\alpha,-\alpha,\beta,-\beta)=E,
 \]
 which means that $E=\Q(\alpha,\beta)$.
 Moreover we can decompose $f$ in $\C[X]$ as 
 \[
 (X-\alpha)(X+\alpha)(X-\beta)(X+\beta)=(X^2-\alpha^2)(X^2-\beta^2)=X^4-(\alpha^2+\beta^2)X^2+\alpha^2\beta^2.
 \]
This implies in particular that $\alpha^2\beta^2=5$, hence $\beta=\pm \frac{\sqrt{5}}{\alpha}\in \Q(\alpha,\sqrt{5})$.
Therefore $E=\Q(\alpha,\beta)\subseteq \Q(\alpha,\sqrt{5})$.
On the other hand $\sqrt{5}=\pm\alpha\beta\in\Q(\alpha,\beta)$,
 hence $\Q(\alpha,\sqrt{5})\subseteq\Q(\alpha,\beta)=E$.
 So we can conclude that $E=\Q(\alpha,\sqrt{5})$.
Using the multiplicative of the degree of finite extension we get that
 \[
 [E:\Q]=[E:\Q(\alpha)][\Q(\alpha):\Q].
\]
But $[\Q(\alpha):\Q]$ is equal to the degree of the minimal
polynomial of $\alpha$ over $\Q$.
 Using Eisenstein criterion with $p=5$, we have that $f$ is irreducible (and monic), so it is the minimal polynomial of $\alpha$ over $\Q$.
Thus $[\Q(\alpha):\Q]=\deg f=4$.
It remains to compute $[E:\Q(\alpha)]=[\Q(\alpha,\sqrt{5}):\Q(\alpha)]$.
 We have the following situation:
\begin{multicols}{2}
\adjustbox{scale=.98,center}{%
\begin{tikzcd}
	& {E=\Q(\alpha,\sqrt{5})}\arrow[rd,no head]\\
	{\Q(\alpha)}\arrow[ru,no head,"\leq 2"]&& {\Q(\sqrt{5})}\arrow[ld,no head,"2"]  \\
	& {\Q}\arrow[lu,no head,"4"] 
 \end{tikzcd}
 }
 \vfill\null
 \columnbreak
 Observe that $\Q(\alpha,\sqrt{5})$ is equal to the composite of $\Q(\alpha)$ and $\Q(\sqrt{5})$.
 We can use the property of composite extension,
$[LF:L]\leq[F:K]$, to deduce that 
 \[
 [\Q(\alpha,\sqrt{5}):\Q(\alpha)]\leq [\Q(\sqrt{5}):\Q]=2.
 \]
 The last equality is because $X^2-5$ is 
 the minimal polynomial of $\sqrt{5}$ over $\Q$,
 as it is monic has $\sqrt{5}$ as a root 
 and it's irreducible due to Eisenstein's criterion.
 \end{multicols}
 Finally, we want to understand whether $[\Q(\alpha,\sqrt{5}):\Q(\alpha)]$ is 1 or 2.
 For this, we need to understand the relation between $\alpha$ and $\sqrt{5}$.
Note that $\alpha^4-5\alpha^2+5=0$, so we can solve the equation for
$\alpha^2$ as it is a root of $X^2-5X+5$, i.e.
 \[
\alpha^2=\frac{5\pm \sqrt{25-20}}{2}=\frac{5\pm \sqrt{5}}{2},
 \]
 hence $\sqrt{5}=\pm (2\alpha^2-5)\in\Q(\alpha)$.
 So $\Q(\alpha,\sqrt{5})\subseteq\Q(\alpha)\subseteq \Q(\alpha,\sqrt{5})$, 
which means that $E=\Q(\alpha)$ and
$[E:\Q]=[\Q(\alpha):\Q]=4$.
\end{sol}

\begin{ex}
\label{5.4}
    Let $\alpha=\sqrt[4]{7}+\sqrt{2}\in\C$
    \begin{enumerate}[label=(\roman*)]
        \item Prove that $\sqrt{2}\in \Q(\alpha)$.

        \noindent\textit{Hint:} Use that $(\alpha-\sqrt{2})^4=7$ and compute $\sqrt{2}$ depending on $\alpha$.
        \item Prove that $\Q(\alpha)=\Q(\sqrt{2},\sqrt[4]{7})$.
        \item Prove that $\sqrt{2} \not\in \Q(\sqrt[4]{7})$.

        \noindent\textit{Hint:} If $\sqrt{2} \in \Q(\sqrt[4]{7})$, then the extension $\Q(\sqrt[4]{7})/\Q(\sqrt{2})$ would be a quadratic extension. Therefore, there were $\beta,\gamma \in \Q(\sqrt{2})$ such that
        \[
        \sqrt[4]{7}^2 + \beta \sqrt[4]{7} + \gamma = 0.
        \]
        Produce a contradiction by showing that this would imply $\sqrt{7} \in \Q(\sqrt{2})$.
        \item Compute $[\Q(\alpha):\Q]$
        \item Prove that $\Q(\alpha)/\Q$ is not a normal extension.
        
        \noindent \textit{Hint:} $\sqrt[4]{7}$.
    \end{enumerate}
\end{ex}
\begin{sol}~

    \begin{enumerate}[label=(\roman*)]
        \item Note that $(\alpha - \sqrt{2})^4 -7 = 0$. By expanding the left side, we get
        \begin{align*}
            0 &= \alpha^4 - 4\sqrt{2}\alpha^3 + 12 \alpha^2 - 8\sqrt{2}\alpha - 3 \\
            & = (\alpha^4 + 12\alpha^2 - 3) -  (4 \alpha^3 + 8 \alpha )\sqrt{2}. \\
            \rightarrow \quad \sqrt{2} & = \frac{\alpha^4 + 12\alpha^2 - 3}{4 \alpha^3 + 8 \alpha} \in \Q(\sqrt{2}).
        \end{align*}
        We are left with checking that $4 \alpha^3 + 8 \alpha \neq 0$. But this would only be possible for $\alpha \in \{ 0, \pm i\sqrt{2} \}$ which is not the case.
        \item From the definition, it is immediate that $\alpha \in \Q(\sqrt{2} , \sqrt[4]{7})$, therefore $\Q(\alpha) \subseteq \Q(\sqrt{2} , \sqrt[4]{7})$.

        On the other hand, the first part of the exercise shows that $\sqrt{2} \in \Q(\alpha)$. As $\sqrt[4]{7} = \alpha - \sqrt{2} \in \Q(\alpha)$, we also see that $\sqrt[4]{7} \in \Q(\alpha)$. It follows that $\Q(\sqrt{2} , \sqrt[4]{7}) \subseteq \Q(\alpha)$.

        We conclude that $\Q(\sqrt{2} , \sqrt[4]{7}) = \Q(\alpha)$.

        \item Suppose that $\sqrt{2} \in \Q(\sqrt[4]{7})$. As $[\Q(\sqrt[4]{7}):\Q] = 4$ and $[\Q(\sqrt{2}):\Q] = 2$ this would imply that $[\Q(\sqrt[4]{7}):\Q(\sqrt{2})] = 2$. Let therefore $f(\sqrt[4]{7},\Q(\sqrt{2})) = X^2 + \beta X + \gamma$ with $\beta, \gamma \in \Q(\sqrt{2})$. 
        
        Evaluating $f$ at $X = \sqrt[4]{7}$ would lead to $\sqrt{7} + \beta \sqrt[4]{7} + \gamma = 0 $,
        hence
        \[
        \beta^2 \sqrt{7}  = (- \sqrt{7} - \gamma)^2= 7 + 2\gamma \sqrt{7}  + \gamma^2 
        \]
        and so
        \[
        \sqrt{7}  = \frac{\gamma^2 + 7}{\beta^2 - 2 \gamma} \in \Q(\sqrt{2}).
        \]
        This is a contradiction, as soon as we have justified the last step by excluding $\gamma = \frac{\beta^2}{2}$. But $X^2 + \beta X + \frac{\beta^2}{2} = 0$ holds only for $X =\frac{\beta}{2} (-1 \pm i)$ which is clearly not in $\Q(\sqrt{2})$.

        \item Recall that $\Q(\alpha) = \Q(\sqrt{2},\sqrt[4]{7})$. As $\sqrt{2} \not\in \Q(\sqrt[4]{7})$, we see that $[\Q(\alpha):\Q(\sqrt[4]{7})] > 1$. On the other hand, $\sqrt{2}$ is a zero of $X^2 - 2 \in \Q(\sqrt[4]{7})[X]$, therefore $[\Q(\alpha):\Q(\sqrt[4]{7})] \leq 2$, which proves that $[\Q(\alpha):\Q(\sqrt[4]{7})] = 2$.

        Therefore,
        \[
        [\Q(\alpha):\Q] = [\Q(\alpha):\Q(\sqrt[4]{7})] \cdot [\Q(\sqrt[4]{7}): \Q] = 2 \cdot 4 = 8.
        \]
        \item We know that $\sqrt[4]{7} \in \Q(\alpha)$ which has minimal polynomial $f(\sqrt[4]{7},\Q) = X^4-7$. One root of this polynomial is $i\sqrt[4]{7} \not\in \Q(\alpha)$, therefore $\Q(\alpha)/\Q$ is not normal.\qedhere
    \end{enumerate}
\end{sol}

 \newpage

\section{Week 6}

\begin{ex}
\label{6.1}
    Let $F$ be a field, we denote by $F^\times$ the group $F\setminus\{0\}$ with the field multiplication. Every finite subgroup of $F^\times$ is a cyclic group.
    
    Prove this statement using the following steps:
    \begin{enumerate}[label=(\roman*)]
        \item Let $G$ be a finite subgroup of $F^\times$. Being an abelian finite group we have that $G\cong \bigoplus_{i=1}^k \Z/{p_i^{n_i}}$,
        where $p_i$ are not necessarily distinct primes.
        Take $m=\lcm(p_i^{n_i}\mid i\in\{1,\dots,k\})$.
        Prove that $x^m=1$ for all $x\in G$.
        \item Prove that $m=\prod_{i=1}^k p_i^{n_i}$.
        
        \noindent\textit{Hint: }Consider the polynomial $X^m-1$. How many roots does it have?
        \item Prove that $G$ is cyclic.
        
        \noindent\textit{Hint: }Show that $p_i\neq p_j$ for all $i\neq j$.
    \end{enumerate}
\end{ex}
\begin{sol}
We can assume that $G=\bigoplus_{i=1}^k \Z/{p_i^{n_i}}$.
\begin{enumerate}[label=(\roman*)]
    \item For every $x\in G$ we can write it as $x=(x_1,\dots, x_k)$, where $x_i\in \Z/{p_i}^{n_i}$ for every $i\in\{1,\dots,k\}$..
    So $x_i^{p_i^{n_i}}=1$ for every $i\in\{1,\dots,k\}$.
    Since $m$ is also a multiple of $p_i^{n_i}$, we get that $x_i^m=1$ for every $i\in\{1,\dots,k\}$.
    Therefore $x^m=(x_1,\dots, x_k)^m=1_G$.
    \item The polynomial $X^m-1\in K[X]$ has degree $m$, so it can have at most $m$ roots.
    However, the previous point of this exercise shows that every element of $G$ is a root.
    Thus $\prod_{i=1}^k p_i^{n_i}=|G|\leq m$.
    On the other hand, clearly $m=\lcm(p_i^{n_i}\mid i\in\{1,\dots,k\})\leq \prod_{i=1}^k p_i^{n_i}$.
    Therefore $m=\prod_{i=1}^k p_i^{n_i}$.
    \item Since we proved that $\lcm(p_i^{n_i}\mid i\in\{1,\dots,k\})=\prod_{i=1}^k p_i^{n_i}$, we can deduce that
    $p_i^{n_i}$ and $p_j^{n_j}$ have to be coprime whenever $i\neq j$.
    Therefore $p_i\neq p_j$ for all $i\neq j$.
    Moreover, since $G\cong\bigoplus_{i=1}^k \Z/{p_i^{n_i}}$, we can deduce, using the Chinese remainder theorem,
    that $G\cong\Z/m$.\qedhere
\end{enumerate}
\end{sol}

\begin{ex}
\label{6.2}
   If $C$ is an algebraic closure of $K$, $x\in C$ and $G=\Gal(C/K)$. Prove that the following statements are equivalent:
   \begin{enumerate}[label=(\roman*)]
        \item $x$ is separable over $K$.
        \item Every $y\in O_G(x)$ is separable over $K$.
        \item $\gamma(K(x)/K)=[K(x):K]=\deg(f(x,K))$.
    \end{enumerate}
\end{ex}
\begin{sol}
By Proposition 4.11 of the lecture notes, we know that 
$$f(x,K)=\prod_{y\in O_G(x)} (X-y)^m$$
for some $m$.
Moreover, by Proposition 4.10 of the lecture notes,
$f(y,K)=f(x,K)$ for all $y\in O_G(x)$.
    \begin{description}[font=\normalfont]
        \item[(i)$\iff$(ii)] $x$ is separable over $K$ if and only if $x$ is a simple root of $f$, i.e. $m=1$.
        Thus $x$ is separable over $K$ if and only if $y$
        is separable over $K$ for all $y\in O_G(x)$.
        \item[(i)$\iff$(iii)] We know already that $[K(x):K]=\deg(f(x,K))$.
        By Proposition 5.16 of the lecture notes, we also have that $\gamma(K(x)/K)=|O_G(x)|$.
        Thus 
        $$[K(x):K]=\deg(f(x,K))=m|O_G(x)|=\gamma(K(x)/K).$$
        But $x$ is separable over $K$ if and only if $m=1$, i.e.
        \[
        [K(x):K]=\deg(f(x,K))=|O_G(x)|=\gamma(K(x)/K).\qedhere
        \]
    \end{description}
\end{sol}

Recall the following property.

\begin{pro}
    Let $K$ be a field and $C$ an algebraic closure of $K$. If a polynomial $f \in K[X]$ is irreducible and has repeated roots in $C$, then $K$ has positive characteristic $p > 0$ and  $f = g(X^p)$ for an irreducible polynomial $g \in K[X]$.
\end{pro}

\begin{ex}
\label{6.3}
    Let $K$ be a field of positive characteristic $p$ and let $K(t)$ be the field of rational functions over $K$. Prove that the extension $K(t)/K(t^p)$ is inseparable.

    \noindent\textit{Hint:} Use the Eisenstein criterion to prove that the polynomial $X^p-t^p$ is irreducible in $K(t^p)[X]$.
\end{ex}
\begin{sol}
    It is clear that $K(t^p)(t) = K(t)$ and that $t$ is a root of the polynomial $f = X^p - t^p \in K(t^p)[X]$. We have to prove that $f = f(t,K(t^p))$ by showing that it is irreducible in $K(t^p)[X]$.

    $K(t^p)$ is the field of fractions of the factorization domain $K[t^p]$ which we can identify with the ring $K[Y]$ by substituting $Y = t^p$. We apply Eisenstein with the prime $Y \in K[Y]$ to the monic polynomial $f = X^p - Y$: $Y | a_i$ for $i < p$ and $Y^2 \nmid Y = a_0$, hence $f$ is irreducible in $K[Y,X]$ and, thus, also in $K(Y)[X]=K(t^p)[X]$.

    Note that $f = X^p - t^p = (X - t)^p$ which implies that $t$ is a repeated root of $f = f(t, K(t^p))$. This proves that $K(t)/K(t^p)$ is an inseparable extension. 
\end{sol}

\begin{ex}
\label{6.4}
    Let $K$ be a field. Prove the equivalence of the following two statements:

    \begin{enumerate}
        \item Each algebraic extension of $K$ is separable.
        \item \textit{Either} $K$ has characteristic $0$, \textit{or} $K$ has positive characteristic $p > 0$ and the Frobenius endomorphism $\Phi_K: K \to K$; $\Phi_K(x) = x^p$ is bijective.
    \end{enumerate}

    \noindent\textit{Hint:} The first statement is equivalent to the statement that each irreducible polynomial in $K[X]$ doesn't have repeated roots in an algebraic closure. Use the proposition to reduce the problem to the case that $K$ has positive characteristic $p$.

    In this case, use the fact that $(K[X])^p = \Phi_K(K)[X^p]$ to show that if $\Phi_K(K) = K$ then there are no irreducible polynomials of the form $g(X^p)$ in $K[X]$. On the other hand, if $\Phi_K(K) \neq K$, let $b \in K \setminus \Phi_K(K)$ and prove that the polynomial $X^p - b$ is irreducible in $K[X]$.
    
\end{ex}

\begin{sol}
We have to show that $K$ can only have inseparable extensions if and only if $K$ has positive characteristic $p$ and $\Phi_K$ isn't surjective.

The existence of inseparable extensions of $K$ implies that there is an element $\alpha \in L$ for some extension $L/K$ such that $f(\alpha,K)$ has repeated roots. We know that this is only possible if $K$ has positive characteristic $p$ and $f(\alpha,K) = g(X^p)$ for some irreducible polynomial $g \in K[X]$. Suppose that $\Phi_K$ is bijective and write
\[
g = \sum_{i=0}^n a_iX^i
\]
for some $a_i \in K$. Then
\[
f = g(X^p) = \sum_{i=0}^n a_iX^{i\cdot p} = \sum_{i=0}^n (\Phi_K^{-1}(a_i))^pX^{i\cdot p} = \left( \underbrace{\Phi_K^{-1}(a_i)}_{=b_i}X^i \right)^p.
\]
As all $b_i \in K$, this shows that $f$ cannot be irreducible. Therefore, we have proven that inseparable extensions of $K$ can only exist if $\Phi_K$ isn't bijective.

Suppose now that $K$ has positive characteristic $p$ and $\Phi_K$ isn't bijective. As a homomorphism of fields, $\Phi_K$ is always injective, therefore it is not surjective. Suppose that $a \not\in \im\ \Phi_K$. This means that $a$ is not a $p$-th power. Let $\alpha$ be a root of $f$ in an algebraic closure $\overline{K}/K$. We claim that the extension $K(\alpha)/K$ is inseparable. As $\alpha^p = a$, we see that
\[
f(\alpha, K) | X^p - a = X^p - \alpha^p = (X - \alpha)^p \Rightarrow f(\alpha,K) = (X- \alpha)^k \textnormal{ for some } 2 \leq k \leq p.
\]
Note that $\alpha \not\in K$ as $a$ is not a $p$-th power in $K$, so indeed $k \geq 2$. But this implies that $\alpha$ is a repeated root of $f(\alpha,K)$, so $L/K$ is inseparable.
\end{sol}



\newpage

\section{Week 7}


\begin{ex}
\label{7.1}
    Let $K$ be a field of characteristic different from 2. 
    \begin{enumerate}
        \item  Let $E/K$ be an extension of degree $[E:K]=2$, prove that $E/K$ is Galois.
        \item Find a counterexample in characteristic 2.
        
        \noindent \textit{Hint: }Consider the field $E=\Z/2(X)$. 
    \end{enumerate}
\end{ex}
\begin{sol}~
\begin{enumerate}
    \item First of all, since the extension is finite we know that 
$E/K$ is also algebraic.

Let now $\alpha \in E\setminus K$.
Then 
$$1<\deg(f(\alpha,K))=[K(\alpha)]\leq [E:K]=2,$$
so $\deg(f(\alpha,K))=2$.
In particular, since $\alpha\in E$ is a root of $f(\alpha,K)$, 
a polynomial of degree 2,
we deduce that $f(\alpha,K)$
decomposes linearly in $E[X]$.
Hence $E$ is a decomposition field
of $f=f(\alpha,K)$.
Moreover, we can write $f=X^2+aX+b$ for some $a,b\in K$.
Hence $f'=2X+a$ is non-zero because $K$ has characteristics different from 2,
thus $f$ is a separable polynomial (by the derivative criterion).
Another way to prove the separability of $f$
is that otherwise $K$ would have characteristic $p>0$
and $f\neq g(X^p)$ for some $g\in K[X]$,
a contradiction with $\deg(f)=2\neq p$.
Therefore $E$ is a decomposition field of $f(\alpha,K)$, which is 
a separable polynomial, i.e. $E/K$ is Galois.
\item Consider $E=\Z/2(X)$ and $K=\Z/2(X^2)$.

We can re-write $E$ also as $E=K(X)$, so $[E:K]=\deg(f(X,K))$.
Since the polynomial $Y^2-X^2\in K[Y]$
has $X$ as a root in $E$, we know that $[E:K]\leq 2$.
Moreover $\{1,X\}$ is a generating set of $E$ as a $K$-vector space.
On the other hand, $X\notin K$, otherwise $X=\frac{f(X^2)}{g(X^2)}$, for some polynomials $f,g \neq 0$ with coefficients in $\Z/2$.
Thus $X$ would satisfy $Xg(X^2)=f(X^2)$,
but the degree as polynomials in $X$ would be odd for
$\deg(Xg(X^2))$ and even for $f(X^2)$, a contradiction.
Therefore $[E:K]>1$, i.e. $[E:K]=2$.
Moreover $\deg(f(X,K))[E:K]=2$, so $f(X,K)=Y^2-X^2$.

Finally, we prove that $E/K$ is not a Galois extension, proving that it is not separable.
In fact, the element $X\in E$ is not separable as its minimal polynomial is
$f(X,K)=Y^2-X^2=(Y-X)^2\in E[Y]$,
so $X$ is a multiple root of it. \qedhere
\end{enumerate}
\end{sol}

\begin{ex}
\label{7.2}
    Let $E/K$ be a separable extension and $L/K$ be the normal closure of $E$ in an algebraic closure $C$ of $K$.
    Prove that $L/K$ is a Galois extension.
\end{ex}
\begin{sol}
Recall that, if $E=K(S)$ for a subset $S\subseteq E$ and $C$ is an algebraic closure of $K$,
the definition of $L$
is $L=K(T)$,
where 
$$T=\{y\in C\mid y\text{ is a root of }f(x,K) \text{ for }x\in S\}$$
We know already that $L/K$ is normal.
    Since for an extension being Galois is equivalent to being normal and separable, we are left to prove that $L/K$ is separable.
Moreover, to prove that $L/K$ is separable it is enough to prove that 
every element in $T$ is separable over $K$.
So let $y\in T$, i.e. $y$ is a root of $f(x,K)$
for some $x\in S$.
But then $f(y,K)=f(x,K)$.
We also know that 
$f(x,K)=\prod_{y\in O_G(x)}(X-y)^m$,
but $x\in E$, so it is separable, hence $f(x,K)$ is also separable and therefore $y$ is separable.    
\end{sol}



\begin{ex}
\label{7.3}
Let $K$ be a field of characteristic $p > 0$ and let $L$ be an algebraic extension.
\begin{enumerate}[label=(\roman*)]
        \item Prove that for every $\alpha \in L$, there is an $n \in \Z_{\geq 0}$ such that $\alpha^{p^n}$ is separable.

        \noindent \textit{Hint}: Write $f(\alpha,K) = g(X^{p^n})$ with $n$ as large as possible.
        \item Recall that a field extension $L/K$ is \emph{purely inseparable} if and only if 
        %each element $\alpha \in L$ is not separable over $K$.
        %in the notes it's written as 
        $f(\alpha,K)$ is not separable for all $\alpha \in L\setminus K$.
        Prove that assuming that $L/K$ is finite, then it is purely inseparable if and only if $L^{p^{\infty}} = \bigcap_{n \geq 0} L^{p^n} \subseteq K$.
        \item Let $M = K L^{p^{\infty}}$ and assume that $L/K$ is finite. Prove that $M/K$ is separable and $L/M$ is purely inseparable.
    \end{enumerate}
\end{ex}
\begin{sol}~

    \begin{enumerate}[label=(\roman*)]
        \item Let $\alpha\in L$, and let $f=f(\alpha,K)$ be its minimal polynomial.
        %If $\alpha$ is separable over $K$ then we have the claim 
        %with $n=1$.
        We can write $f(\alpha,K) = g(X^{p^n})$ with $n$ as large as possible.
        Moreover, $g$ is irreducible (otherwise $f(\alpha,K)$ would be reducible, a contradiction) and, for the maximality of $n$,
        it is not of the form $h(X^p)$ for $h\in K[X]$.
        Therefore $g\in K[X]$ is a separable polynomial.
        But $g(\alpha^{p^n})=0$, so $g=f(\alpha^{p^n},K)$
        and $\alpha^{p^n}$ is a separable element over $K$.
        \item Let $L/K$ be a finite purely inseparable extension.
        Observe that the $n$ found in the previous part of the exercise is bounded by $[L:K]$.
        Thus there exists $n\in \N$ such that 
        $\alpha^{p^n}$ is separable for all $\alpha\in L$, hence $\alpha^{p^n}\in K$ 
        (since $L/K$ is purely inseparable).
        Thus $L^{p^n}\subseteq K$ and so $L^{p^\infty}\subseteq K$.

        Vice versa assume that $L^{p^\infty}\subseteq K$.
        We want to prove that $L/K$ is purely inseparable.
        Let $\alpha\in L\setminus K$.
        Then there is an $n\in \N$ such that $\beta=\alpha^{p^n}\in L^{p^\infty}\subseteq K$.
        So $\alpha$ is a root of $X^{p^n}-\beta\in K[X]$,
        hence $f(\alpha,K)$ divides $X^{p^n}-\beta$ in $K[X]$.
        But $X^{p^n}-\beta=(X-\alpha)^{p^n}$ in $C[X]$, for $C$ an algebraic closure of $K$.
        Thus $f(\alpha,K)=(X-\alpha)^d$ for some $d\in \N$ and since $\alpha\notin K$ we have that $d>1$.
        Therefore $f(\alpha,K)$ is not separable over $K$.
        \item Since $L/K$ is finite,
        there exists $n\in\N$ such that 
        all elements in $L^{p^n}$ are separable over $K$.
        Therefore all $\alpha\in L^{p^\infty}$ are separable over $K$, hence $M/K$ is separable.

        We know that $L^{p^\infty}\subseteq M$, so by
        the previous part of the exercise,
        we know that $L/M$ is purely inseparable.\qedhere
    \end{enumerate}
\end{sol}

\begin{ex}
\label{7.4}~

    \begin{enumerate}[label=(\roman*)]
    \item Let $L/K$ be a finite Galois extension and let $M/K$ be a subextension.
    For $\sigma \in \Gal(L/K)$ prove that $\Gal({}^{\sigma}M/K) = {}^{\sigma}\Gal(M/K)$, where the latter upperscript-$\sigma$ denotes conjugation by $\sigma$ in $\Gal(L/K)$.
    
    \item Let $E/K$ be a finite Galois extension 
    and $F_1,\dots,F_n$ fields 
    such that $K\subseteq F_i\subseteq E$ for 
    all $i\in\{1,\dots,n\}$. For every 
    $i$ let $S_i=\Gal(E/F_i)$. Then
    \[
    \Gal\left(E/\bigcap_{i=1}^nF_i\right)=\left\langle\bigcup_{i=1}^nS_i\right\rangle,
    \quad
    \Gal\left(E/\prod_{i=1}^nF_i\right)=\bigcap_{i=1}^nS_i.
    \]

    \noindent \textit{Hint:} Determine the subfield fixed by $\left\langle\bigcup_{i=1}^nS_i\right\rangle$ and the subgroup of $\Gal(E/K)$ that fixes $\prod_{i=1}^nF_i$.
    \end{enumerate}
\end{ex}
\begin{sol}
    See \cref{8.1}.
\end{sol}

\begin{comment}
\begin{ex}
    Let $L$ be a finite field of characteristic $p > 0$. We know that $|L|= p^n$ for some $n \geq 1$. We furthermore know that $L$ contains $K = \Z/p$ as a subfield. 
    \begin{enumerate}[label=(\roman*)]
        \item Use the fact that $L^{\times}$ is cyclic to prove that $L$ is a decomposition field of the polynomial $X^{p^n}-X$ over $K$.
        \item Prove that $\Gal(L/K) = \left\langle \Phi_L \right\rangle \cong C_n$ where $\Phi_L$ is the Frobenius endomorphism $x \mapsto x^p$.
        \item Show that $L/K$ is Galois and use the Galois correspondence to describe the possible sizes of the subfields $M \subseteq L$. How many are there?
    \end{enumerate}
\end{ex}
\end{comment}

% I'd consider leaving this out as finite fields are dealt with in the next week.

\begin{ex}
\label{7.5}
For infinite extensions, the Galois correspondence is not always true (more precisely $\beta$ is not always injective).

Let $p_1,p_2,\dots$ be the ordered list of prime numbers, let
$S_k=\{\sqrt{p_i}\mid 1\leq i\leq k\}$ and 
$S=\cup_{k\in\N} S_k$.
Consider the field 
extensions $E_k=\Q(S_k)$ 
and $E=\Q(S)$,
let $G=\Gal(E/\Q)$ and for every $\sigma\in G$ 
denote $m(\sigma)=\{x\in P\mid \sigma(x)\neq x\}$.

Let $H=\{\sigma \in G\mid m(\sigma)\text{ is finite}\}$.
\begin{enumerate}[label=(\roman*)]
        \item Prove by induction on $k$ that 
        $\sqrt{m}\notin E_k$
        for every $m\in \N$ such that $\gcd(m,p_i)=1$ for all $1\leq i\leq k$.
        \item $H$ is a proper subgroup of $G$.
        \item ${}^HE=\Q$.
    \end{enumerate}
\end{ex}
\begin{sol}
    See \cref{8.2}.
\end{sol}

\newpage

\section{Week 8}

\begin{ex}
\label{8.1}~

    \begin{enumerate}[label=(\roman*)]
    \item Let $L/K$ be a finite Galois extension and let $M/K$ be a subextension.
    For $\sigma \in \Gal(L/K)$ prove that $\Gal(\sigma(M)/K) = \sigma\Gal(M/K)\sigma^{-1}$.
    
    \item Let $E/K$ be a finite Galois extension 
    and $F_1,\dots,F_n$ fields 
    such that $K\subseteq F_i\subseteq E$ for 
    all $i\in\{1,\dots,n\}$. For every 
    $i$ let $S_i=\Gal(E/F_i)$. Then
    \[
    \Gal\left(E/\bigcap_{i=1}^nF_i\right)=\left\langle\bigcup_{i=1}^nS_i\right\rangle,
    \quad
    \Gal\left(E/\prod_{i=1}^nF_i\right)=\bigcap_{i=1}^nS_i.
    \]

    \noindent \textit{Hint:} Determine the subfield fixed by $\left\langle\bigcup_{i=1}^nS_i\right\rangle$ and the subgroup of $\Gal(E/K)$ that fixes $\prod_{i=1}^nF_i$.
    \end{enumerate}
\end{ex}
\begin{sol}
~
\begin{enumerate}[label=(\roman*)]
\item By definition an element in $\Gal(\sigma(M)/K)$
is a field automorphism $\varphi:\sigma(M)\to \sigma(M)$ such that $\varphi|_K=id$.
Given $\psi\in \Gal(M/K)$, then 
$\sigma\psi\sigma^{-1}$ is bijective and
$\sigma\psi\sigma^{-1}(\sigma(M))=\sigma(M)$,
so $\sigma\psi\sigma^{-1}\in \Gal(\sigma(M)/K)$.
On the other hand, if $\varphi\in \Gal(\sigma(M)/K)$, then $\varphi=\sigma\psi\sigma^{-1}$, where $\psi=\sigma^{-1}\varphi\sigma\in \Gal(M/K)$.
\item Let $S=\left\langle\bigcup_{i=1}^nS_i\right\rangle$,
we want to prove that $\bigcap_{i=1}^nF_i={}^S L$.
\begin{itemize}
    \item[$(\subseteq)$] let 
$x\in \bigcap_{i=1}^nF_i$. For every $\varphi\in \bigcup_{i=1}^nS_i$, there is $j\in\{1,\dots, n\}$ such that $\varphi\in S_j=\Gal(E/F_j)$, so $\varphi(x)=x$ since $x\in \bigcap_{i=1}^nF_i\subseteq F_j$.
Hence $\varphi(x)=x$ also for every $\varphi\in S=\left\langle\bigcup_{i=1}^nS_i\right\rangle$.
    \item[$(\supseteq)$] Vice versa, $S_i\leq S$, for every $i\in\{1\dots,n\}$. So, since the Galois correspondence reverses inclusions,
${}^{S_i}E\supseteq{}^SE$. But, by the correspondence, we also know that ${}^{S_i}E=F_i$.
Therefore $F_i\supseteq {}^HE$ for all $i\in\{1,\dots,n\}$, hence $\bigcap_{i=1}^nF_i \supseteq {}^SE$.
\end{itemize}
So we proved that $\bigcap_{i=1}^nF_i={}^S E$, so, using again the Galois correspondence,
we get that $$\Gal\left(E/\bigcap_{i=1}^nF_i\right)=\Gal\left(E/{}^H E\right)=S=\left\langle\bigcup_{i=1}^nS_i\right\rangle.$$

Let $F=\prod_{i=1}^nF_i$,
we want to prove that $\Gal(E/F)=\bigcap_{i=1}^nS_i$.
\begin{itemize}
    \item[$(\subseteq)$]  Let $\sigma \in \Gal(E/F)$, then $\sigma|_{F_i}=id$ for all $i$, so $\sigma\in \Gal(E/F_i)=S_i$ for all $i$.
Therefore $\bigcap_{i=1}^nS_i\supseteq \Gal(E/F)$.
\item[$(\supseteq)$] Since $F_i\subseteq \prod_{i=1}^nF_i=F$ for all  $i$,
then $S_i=\Gal(E/F_i)\geq \Gal(E/F)$
for all $i$.
So $\bigcap_{i=1}^nS_i\subseteq \Gal(E/F)$.\qedhere
\end{itemize}

\end{enumerate}
    
\end{sol}


\begin{ex}
\label{8.2}
For infinite extensions, the Galois correspondence is not always true (more precisely $\beta$ is not always injective).

Let $p_1,p_2,\dots$ be the ordered list of prime numbers, let
$S_k=\{\sqrt{p_i}\mid 1\leq i\leq k\}$ and 
$S=\cup_{k\in\N} S_k$.
Consider the field 
extensions $E_k=\Q(S_k)$ 
and $E=\Q(S)$,
let $G=\Gal(E/\Q)$ and for every $\sigma\in G$ 
denote $m(\sigma)=\{x\in S\mid \sigma(x)\neq x\}$.

Let $H=\{\sigma \in G\mid m(\sigma)\text{ is finite}\}$.
Prove that
\begin{enumerate}[label=(\roman*)]
        \item  
        For every $n$ such that $\gcd(n,p_i)=1$ for all $1\leq i\leq k$, 
        we have $\sqrt{n}\notin E_k$.

        \noindent\textit{Hint:} Use induction on $k$.
        \item $H$ is a proper subgroup of $G$.
        \item ${}^HE=\Q$.
    \end{enumerate}

\end{ex}
\begin{sol}~
    \begin{enumerate}[label=(\roman*)]
    \item If $k=1$, $E_k=\Q(\sqrt{p_1})$ and if
    $\sqrt{n}\in E_k$, then we can write
    $\sqrt{n}=a+b\sqrt{p_1}$, for some $a,b\in \Q$.
    So $n=a^2+b^2p_1+2ab\sqrt{p_1}$
    which would imply that $\sqrt{p_1}\in\Q$, 
    which is not the case.

    Assume now that the thesis is true for $k\geq 1$
    and consider integer 
    $n$ coprime with every $p_i$ for $1\leq i\leq k+1$.
    By inductive hypothesis, we know that
    $p_{k+1}\notin E_k$, so $E_{k+1}=E_k(\sqrt{p_{k+1}})$
    is an extension of degree 2 of $E_k$.
    If $\sqrt{n}\in E_{k+1}$, then we can write 
    $\sqrt{n}=a+b\sqrt{p_{k+1}}$, for some $a,b\in E_k$.
    So $n=a^2+b^2p_1+2ab\sqrt{p_{k+1}}$
    which would imply that $\sqrt{p_{k+1}}\in E_k$, 
    which is not the case.
    Therefore, we have the thesis.
    \item To prove that $H$ is a proper subgroup
    we can show that there is an element $\sigma$ in $G$ 
    such that $\sigma(\sqrt{p_i})=-\sqrt{p_i}$
    for all $i\geq 1$.
    In order to construct this element $\sigma$ we can proceed again by induction showing that there is 
    $\sigma_k\in\Hom(E_k/\Q,E_k/\Q)$ such that
    $\sigma_k(\sqrt{p_i})=-\sqrt{p_i}$
    for all $1\leq i\leq k$.

    If $k=1$ this is clear. Let now $k\geq 1$.
    By the previous part of the 
    exercise $E_{k+1}=E_k(\sqrt{p_k})$ is an extension 
    of degree 2 of $E_k$.
    Hence, given $\sigma_k$, we know that there exists an extension $\sigma_{k+1}$ of $\sigma_k$
    to $\Hom(E_{k+1}/\Q,\overline{\Q}/\Q)$ such that
    $\sigma_{k+1}(\sqrt{p_{k+1}})=-\sqrt{p_{k+1}}$.
    So we can construct the desired $\sigma\in G\setminus H$.
    \item Obviously $\Q\subseteq {}^HE$.
    On the other hand, let $x\in {}^HE$.
    In particular $x\in E=\Q(S)$, but $x$ can be written
    as a finite $\Q$-combination of elements in $S$.
    If we assume that $x\notin\Q$ then there exists a positive integer $k$ 
    such that $x\in E_k\setminus E_{k-1}$.
    Thanks to the first part of this exercise,
    can then consider the homomorphism
    $\tau:E_k\to E_k$ that extends the identity
    of $E_k$ and such that
    $\tau(\sqrt{p_{k}})=-\sqrt{p_{k}}$.
    Clearly, this extends again to an element of $G$
    such that
     $$\overline{\tau}\left(\sqrt{p_j}\right)=\begin{cases}
        \sqrt{p_j} &\text{ if } j\neq k\\
        -\sqrt{p_i} &\text{ if } j=k
    \end{cases}.$$ 
    But actually $\overline{\tau}\in H$, so 
    $\tau(x)=x$, which implies that $x\in E_{k-1}$, a contradiction.\qedhere
    \end{enumerate}
\end{sol}

\begin{ex}
\label{8.3}
Let $n \geq 1$ and let $\Sym_n$ be the symmetric group over $\{1, 2, \ldots, n \}$. For $1 \leq i \leq n$, define the subgroups
\[
S_i = \{ \pi \in \Sym_n : \pi(i) = i \}.
\]
\begin{enumerate}[label=(\roman*)]
\item For $\pi \in \Sym_n$, and $1 \leq i \leq n$ prove that $S_{\pi(i)} = \pi S_i\pi^{-1}$. Show also that
\[
\bigcap_{\pi \in \Sym_n} \pi S_1\pi^{-1} = \{ \mathrm{id} \}.
\]
\item Let $L/K$ be a finite Galois extension with $\Gal(L/K) \cong \Sym_n$. Prove that there is an intermediate extension $E/K$ with $[E:K] = n$ and $\prod_{\pi \in \Sym_n} \pi(E) = L$.
\item Let $L/K$ and $E/K$ be as in the previous item. There is an element $x \in E$ such that $E = K(x)$. Let $f = f(x,K)$ be its minimal polynomial. Prove that $L$ is the decomposition field of $f$ over $K$. In particular, every $\Sym_n$-extension of $K$ is the decomposition field of a polynomial of degree $n$ over $K$.
\end{enumerate}

\end{ex}

\begin{sol}
    See \cref{9.1}.
\end{sol}

\begin{ex}
\label{8.4}
    Let $p$ be a prime and let $L$ be a field with $p^n$ elements. We know that $L$ contains $K = \Z/p$.
\begin{enumerate}[label=(\roman*)]
\item Prove that every element in $L$ is a root of the polynomial $X^{p^n} - X$. Deduce that $L$ is separable over $K$ and that the Frobenius endomorphism $\Phi(x) = x^p$ is an automorphism.
\item Recall that $L^{\ast}$ is a cyclic group of order $p^n-1$. Show that 
$$\Gal(L/K) = \{ \Phi^i : 0 \leq i \leq n-1 \} \cong C_n.$$
\item Let $|L| = p^6$. Use the previous exercise to count the number of elements $x \in L$ with $K(x) = L$.

\noindent \textit{Hint:} Use the Galois correspondence to determine the number of elements that are not contained in a proper subfield of $L$.
\end{enumerate} 
\end{ex}

\begin{sol}
    See \cref{9.2}.
\end{sol}


\newpage

\section{Week 9}

\begin{ex}
\label{9.1}
Let $n \geq 1$ and let $\Sym_n$ be the symmetric group over $\{1, 2, \ldots, n \}$. For $1 \leq i \leq n$, define the subgroups
\[
S_i = \{ \pi \in \Sym_n : \pi(i) = i \}.
\]
\begin{enumerate}[label=(\roman*)]
\item For $\pi \in \Sym_n$, and $1 \leq i \leq n$ prove that $S_{\pi(i)} = \pi S_i\pi^{-1}$. Show also that
\[
\bigcap_{\pi \in \Sym_n} \pi S_1\pi^{-1} = \{ \mathrm{id} \}.
\]
\item Let $L/K$ be a finite Galois extension with $\Gal(L/K) \cong \Sym_n$. Prove that there is an intermediate extension $E/K$ with $[E:K] = n$ and $\prod_{\pi \in \Sym_n} \pi(E) = L$.
\item Let $L/K$ and $E/K$ be as in the previous item. There is an element $x \in E$ such that $E = K(x)$. Let $f = f(x,K)$ be its minimal polynomial. Prove that $L$ is the decomposition field of $f$ over $K$. In particular, every $\Sym_n$-extension of $K$ is the decomposition field of a polynomial of degree $n$ over $K$.
\end{enumerate}

\end{ex}
\begin{sol}
\begin{enumerate}[label=(\roman*)]
We calculate
\begin{align*}
    S_i & = \{ \rho \in \Sym_n : \rho(i) = i \} \\
    & = \{ \rho \in \Sym_n : (\rho\pi^{-1})(i) = \pi^{-1}(i) \} \\
    & = \{ \rho \in \Sym_n : (\pi\rho \pi^{-1})(i) = i \}.
\end{align*}
This shows that $\pi S_i \pi^{-1} = S_{\pi(i)}$. Using this, we see that
\[
\bigcap_{\pi \in \Sym_n} \pi S_1 \pi^{-1}  = \bigcap_{\pi \in \Sym_n} S_{\pi(1)} = \bigcap_{i=1}^n S_i
= \{ \rho \in \Sym_n: \forall 1 \leq i \leq n : \rho(i) = i \} = \{ \mathrm{id} \}.
\]
\item It is easy to see that $S_1 \cong \Sym_{n-1}$, therefore $|S_1| = (n-1)!$ and $[\Sym_n:S_1] = n$.

Combining this knowledge with that gained from the previous exercise, we see that $\Gal(L/K)$ contains a subgroup $H$ with $[\Gal(L/K):H] = n$ and $\bigcap_{\pi \in \Gal(L/K)} \pi H \pi^{-1} = 1$. Let $E = {}^HL$ be the corresponding intermediate field. By the Galois correspondence, we see that
\[
[E:K] = [\Gal(L/K):H] = n
\]
and
\[
    \Gal \left( L/ \prod_{\pi \in \Gal(L/K)} \pi(E) \right) = \bigcap_{\pi \in \Sym_n} \pi \Gal(L/E) \pi^{-1}
    = \bigcap_{\pi \in \Sym_n} \pi H \pi^{-1} = 1.
\]
This implies $\prod_{\pi \in \Gal(L/K)} \pi(E) = L$.

\item We know that $\mathcal{O}_{\Gal(L/K)}(x)$ is the set of roots of $f(x,K)$. If $y = \pi(x) \in \mathcal{O}_{\Gal(L/K)}(x)$, then $K(y) = \pi(K(x))$. Let $M$ be the decomposition field of $f$, then
\begin{align*}
    M & = K\left( y : y \in \mathcal{O}_{\Gal(L/K)}(x) \right) \\
    & = \prod_{y \in \mathcal{O}_{\Gal(L/K)}(x)}K(y) \\
    & = \prod_{\pi \in \Gal(L/K)} \pi(K(x)) \\
    & = \prod_{\pi \in \Gal(L/K)} \pi(E) = L.
\end{align*}
\end{enumerate}
\end{sol}

\begin{ex}
\label{9.2}
    Let $L/K$ be an extension of finite fields where $|K| = q$ and $[L:K] = n$.
\begin{enumerate}[label=(\roman*)]
\item Let $\sigma(x) = x^q$ be the Frobenius automorphism. Prove that $\sigma$ has order $n$ and that $\Gal(L/K) = \left\langle \sigma \right\rangle \cong C_n$.
\item Let $n \in \{3,4,6\}$. Count the number of elements $x \in L$ with $K(x) = L$. What does this tell you about the number of irreducible polynomials $f \in K[X]$ with $\deg f = n$?

\noindent \textit{Hint:} Use the Galois correspondence to determine the number of elements that are not contained in a proper subfield of $L$.
\end{enumerate} 
\end{ex}

\begin{sol}
    \begin{enumerate}[label=(\roman*)]
    \item We already know that $\sigma$ is an automorphism. We show that $\sigma$ has exactly order $n$:

    Note that $|L| = |K|^{[L:K]} = q^n$. By Lagrange's theorem, we see that for all $x \in L^{\times}$,
    \[
    1 = x^{|L^{\times}|} = x^{q^n-1} \Rightarrow x = x \cdot x^{q^n-1} = x^{q^n} = \sigma^n(x)
    \]
    which shows $\sigma^n = \mathrm{id}_L$.
    
    $L^{\times}$ is a cyclic group, therefore $L^{\times} \cong C_{q^n-1} = \left\langle x \right\rangle$ for some $x \in L^{\times}$ with $o(x) = q^n-1$. For $0 < i < p$, we see that for such an $x$,
    \[
    \sigma^i(x) = x^{q^i} \neq x
    \]
    because $1 < q^i < q^n-1$. Therefore $\sigma^i \neq \mathrm{id}_L$ for $0 < i < n$ which proves that $o(\sigma) = n$. Therefore,
    \[
    n = |\left\langle \sigma \right\rangle| \leq |\Gal(L/K)| \leq [L:K] = n.
    \]
    This shows that $\Gal(L/K) = \left\langle \sigma \right\rangle \cong C_n$. Furthermore, as the inequalities force the equality $|\Gal(L/K)| = [L:K]$, we conclude that $L/K$ is a Galois extension.
    \item Note that a cyclic group $C_n$ has a unique subgroup of index $d$ for each positive $d|n$. By the Galois correspondence, this shows that $L/K$ has a unique intermediate extension $L_d/K$ for any $d|[L:K] = n$. Observe that $L_n = L$ and $L_1 = K$.

    We first determine the number of elements in $L$ that are not contained in any intermediate field of $L/K$:

    \textit{$n=3$:} We have to calculate $|L_3 \setminus L_1| = q^3 - q^1 = q^3-q$.

    \textit{$n=4$:} Note that $L_1 \subseteq L_2$, therefore the desired quantity is
    \[
    |L_4 \setminus L_2 | = q^4 - q^2.
    \]

    \textit{$n=6$:} We first determine
    \[
    |L_2 \cup L_3| = |L_3| + |L_2| - |L_2 \cap L_3| = |L_3| + |L_2| - |L_1| = q^2 + q^3 - q.
    \]
    As $L_1 \subseteq L_2,L_3$, the number of elements not contained in any intermediate fields is given by
    \[
    |L_6 \setminus (L_2 \cup L_3)| = q^6 - q^3 - q^2 + q.
    \]

    We know that each root $\alpha$ of an irreducible polynomial $f \in K[X]$ with $\deg f = n$ generates an extension $M/K$ with $[M:K] = n$ and therefore satisfies $\alpha^{q^n} = \alpha$. It follows that $f | X^{q^n} - X$.

    As $X^{q^n}-X$ has $q^n$ distinct roots in $L$, we conclude that each irreducible polynomial $f$ has $n$ distinct roots in $M$. These roots can not lie in a proper intermediate field of $L/K$. On the other hand, each $\alpha \in L$ that does not lie in a proper intermediate field of $L/K$, satisfies $K(\alpha) = L$ and therefore $\deg f(\alpha,K) = [L:K]= n$.

    Therefore, the number $\mathcal{P}_{q,n}$ of irreducible (monic) $f \in K[X]$ with $\deg f = n$ is $\mathcal{P}_{q,n} = \frac{\mathcal{A}_{q,n}}{n}$ where we denote by $\mathcal{A}_{q,n}$ the number of elements in an extension $L/K$ with $[L:K]$ that are not in a proper intermediate field. We conclude that the number of irreducible monic polynomials in the considered cases is
    \begin{align*}
        \mathcal{P}_{q,3} & = \frac{q^3 - q}{3},\\
        \mathcal{P}_{q,4} & = \frac{q^4 - q^2}{4}, \\
        \mathcal{P}_{q,6} & = \frac{q^6 - q^3 - q^2 + q}{6} .
    \end{align*}

    \medskip

    \noindent \textit{Remark:} For $|K| = q$, the general formula for the number of irreducible monic polynomials $f \in K[X]$ with $\deg f = n$ is
    \[
    \mathcal{P}_{q,n} = \frac{\sum_{d|n}\mu(n/d)q^d}{n}
    \]
    where $\mu$ is the \emph{Moebius function}.
    \end{enumerate}
\end{sol}

\begin{ex}
\label{9.3}
    Give an explicit description of $\mathrm{norm}_{L/K}$ and $\mathrm{trace}_{L/K}$ for the following field extensions:
    \begin{enumerate}[label=(\roman*)]
        \item $\C/\R$,
        \item $\Q[\sqrt{d}]/\Q$, $d$ squarefree,
        \item $\Q[\sqrt[3]{2}]/\Q$

        \noindent \textit{Hint:} Represent an element of $\Q[\sqrt[3]{2}]$ as $x = a + b \sqrt[3]{2} + c \sqrt[3]{4}$, argue that
        \begin{align*}
        \mathrm{norm}_{\Q[\sqrt[3]{2}]/\Q}(x) & = (a + b \sqrt[3]{2} + c \sqrt[3]{4}) \cdot (a + b\xi \sqrt[3]{2} + c\xi^2 \sqrt[3]{4}) \\
        & \quad \cdot (a + b\xi^2 \sqrt[3]{2} + c \xi\sqrt[3]{4})
        \end{align*}
        where $\xi \neq 1$ is a third root of unity. Multiplying this out would give you $27$ summands - you only need those which do not contain a rational power of $\sqrt[3]{2}$ (why?). Then sum everything up by using the fact that $\xi + \xi^2 = -1$.
        \item $K(t)/K(t^p)$, where $\mathrm{char}\ K = p > 0$.
    \end{enumerate}
\end{ex}
\begin{sol}
    See \cref{10.3}.
\end{sol}

\begin{ex}
\label{9.4}~

    \begin{enumerate}[label=(\roman*)]
        \item Let $L/K$ be a finite extension of fields. Prove that $\mathrm{trace}_{L/K} \neq 0$ if and only if $L/K$ is separable.

    \noindent \textit{Hint:} Theorem 5.13.
        \item Let $L/K$ be a cyclic field extension and let $\Gal(L/K) = \left\langle \tau \right\rangle$. For an element $a \in L$, prove the equivalence
        \[
        \mathrm{trace}_{L/K}(a)=0 \Leftrightarrow \exists c \in L: a = c - \tau c.
        \]
        \noindent \textit{Hint:} Determine $\dim_K \ker (c \mapsto c- \tau c)$ and $\dim_K \ker (\mathrm{trace}_{L/K})$ in order to prove that $\im(c \mapsto c - \tau c) = \ker (\mathrm{trace}_{L/K})$.
    \end{enumerate}
\end{ex}
\begin{sol}
    See \cref{10.4}.
\end{sol}


\begin{ex}
\label{9.5}
    Let $L/K$ be an extension with $[L:K] = 2$ and $\mathrm{char}\ K \neq 2$. Express $\mathrm{norm}_{L/K}(x)$ in terms of $\mathrm{trace}_{L/K}(x)$ and $\mathrm{trace}_{L/K}(x^2)$.
\end{ex}
\begin{sol}
    See \cref{10.5}.
\end{sol}

\newpage

\section{Week 10}

\begin{ex}
\label{10.1}
    Let $f(X)=X^4-5X^2+6$.
    \begin{enumerate}[label=(\roman*)]
        \item Let $E$ be the decomposition field of $f$ over $\Q$.
        Compute the Galois group $\Gal(E/\Q)$.
        \item Find all the intermediate extensions of $E/K$.
        \item Prove that for every prime number $p$,
        the polynomial $g(X)=f(X)(X^2-6)$ has always 
        a root in $\Z/p$.                
    \end{enumerate}
\end{ex}
\begin{sol}~
    \begin{enumerate}[label=(\roman*)]
    \item Observe that $f(X)=(X^2-2)(X^2-3)$,
    so its solutions are $\pm \sqrt{2},\pm\sqrt{3}$.
    Hence $E=\Q(\sqrt{2},\sqrt{3})$ 
    and we have the following situation:

\hspace{-2cm}
    \begin{minipage}[c]{0.57\textwidth}
        \begin{tikzcd}
	& {E=\Q(\sqrt{2},\sqrt{3})}\arrow[rd,no head]\\
	{\Q(\sqrt{2})}\arrow[ru,no head,"\leq 2"]&& {\Q(\sqrt{3})}\arrow[ld,no head,"2"]  \\
	& {\Q}\arrow[lu,no head,"2"] 
 \end{tikzcd}
    \end{minipage}
    \begin{minipage}[l]{0.4\textwidth}
        But $\sqrt{3}\notin \Q(\sqrt{2})$,
        Otherwise $\sqrt{3}=a+b\sqrt{2}$,
        for some $a,b\in\Q$ and so
        $3=a^2+2b^2+2ab\sqrt{2}$,
        which would mean that $\sqrt{2}\in\Q$,
        a contradiction.
        Therefore $[E:\Q(\sqrt{2})]=2$ and 
        $[E:\Q]=4$, so
$\Q(\sqrt{2})\cap\Q(\sqrt{3})=\Q$.
    \end{minipage}

    \begin{comment}
    \begin{multicols}{2}
\adjustbox{scale=.80,center}{%
\begin{tikzcd}
	& {E=\Q(\sqrt{2},\sqrt{3})}\arrow[rd,no head]\\
	{\Q(\sqrt{2})}\arrow[ru,no head,"\leq 2"]&& {\Q(\sqrt{3})}\arrow[ld,no head,"2"]  \\
	& {\Q}\arrow[lu,no head,"2"] 
 \end{tikzcd}
 }
 \vfill\null
 \columnbreak
But it is easy to see that $\sqrt{3}\notin \Q(\sqrt{2})$.
Hence $[E:\Q]=4$.
Moreover $\Q(\sqrt{2})\cap\Q(\sqrt{3})=\Q$, hence 
$\Gal(E/K)\cong\Gal(\Q(\sqrt{2})/\Q)\times \Gal(\Q(\sqrt{2})/\Q)\cong\Z/2\times\Z/2$.
 \end{multicols}
 \end{comment}
Hence
\[
\Gal(E/K)\cong\Gal(E/\Q(\sqrt{2}))\times \Gal(E/\Q(\sqrt{3}))\cong\Z/2\times\Z/2.
\]
 Every $\sigma\in\Gal(E/K)$ is completely determined by $\sigma(\sqrt{2})$ and $\sigma(\sqrt{3})$.
 Moreover $\sigma(\sqrt{2})$ has to be a root of
 the minimal polynomial $f(\sqrt{2},\Q)=X^2-2$,
 which are $\pm\sqrt{2}$.
 Similarly $\sigma(\sqrt{3})=\pm\sqrt{3}$.
 Therefore the 4 elements of $\Gal(E/K)$ are
 precisely $\sigma_0=id$ and
 \[
 \sigma_1:\begin{cases}
     \sqrt{2}&\mapsto -\sqrt{2}\\ \sqrt{3}&\mapsto \sqrt{3}
 \end{cases},\quad
 \sigma_2:\begin{cases}
     \sqrt{2}&\mapsto -\sqrt{2},\\ \sqrt{3}&\mapsto \sqrt{3}
 \end{cases},\quad \sigma_3:\begin{cases}
     \sqrt{2}&\mapsto -\sqrt{2},\\ \sqrt{3}&\mapsto -\sqrt{3}
 \end{cases}.
 \]
 \item $G=\Gal(E/K)\cong\Z/2\times\Z/2$ has 5 subgroups: \[
 \{id\}, H_1=\langle\sigma_1\rangle,H_2=\langle\sigma_2\rangle, H_3=\langle\sigma_3\rangle \text{ and }G.
 \]
 The corresponding fixed fields are
 
 \[
 {}^{\{id\}}E=E,{}^{H_1}E=\Q(\sqrt{3}),{}^{H_2}E=\Q(\sqrt{2}),{}^{H_3}E=\Q(\sqrt{6})\text{ and }{}^{G}E=\Q.
 \]
 \item For $p=2$ and $p=3$ it is easy to see that $0$ is a root of $g$.

 Suppose now that $p>3$ and that $f$ has no roots in $\Z/p$.
 Then $2$ and $3$ are not squares in $\Z/p$.
 We need to prove that $X^2-6$ has a root in $\Z/p$,
 which means that $6$ is a square in $\Z/p$.
 Take the subgroup
 $S=\{x^2\mid x\in (\Z/p)^{\times}\}$
 of squares of $(\Z/p)^{\times}$.
 Recall that $(\Z/p)^{\times}$ is the group of units of a field.
 Hence it is a cyclic group, 
 say $(\Z/p)^{\times}=\langle \gamma \rangle$ and
 $S=\langle \gamma^2\rangle$.
 Since $p-1$ is even, the order of $\gamma^2$ is $\frac{p-1}{2}$ and $S$ is a subgroup of $(\Z/p)^{\times}$ of index 2.
 Therefore if $a,b\notin S$ then, since $S$ $ $
 $ab\in S$,
 otherwise $ $
 
 a product of two elements of $(\Z/p)^{\times}$
 that are not in $Sq$ has to be an element of $S$.\qedhere
 \end{enumerate}
\end{sol}


\begin{ex}
\label{10.2}
Let $K$ be a field, $n \geq 1$ and let $E = K(x_1,\ldots,x_n)$ be the rational function field in $n$ indeterminates.
    \begin{enumerate}[label=(\roman*)]
        \item Let $G$ be the subgroup of $\mathrm{Aut}(E/K)$ given by automorphisms of the form
        \[
        \frac{f(x_1,\ldots,x_n)}{g(x_1,\ldots,x_n)} \mapsto \frac{f(x_{\pi(1)},\ldots,x_{\pi(n)})}{g(x_{\pi(1)},\ldots,x_{\pi(n)})}
        \]
        for some $\pi \in \Sym_n$. Let $F = {}^GE$. Prove that $\Gal(E/F)=G \cong \Sym_n$.

        \item Recall Cayley's theorem. Use it, together with the previous part of the exercise, to prove that every finite group is the Galois group of some Galois extension.

        Can you give an explicit description of such an extension for every finite group?
    \end{enumerate}
\end{ex}
\begin{sol}
    ~
    \begin{enumerate}[label=(\roman*)]
    \item Let $\sigma:S_n\to G$ the map $\sigma(\pi)=\sigma_\pi$ such that 
    \[\sigma_\pi(q(x_1,\ldots,x_n)\mapsto q(x_{\pi(1)},\ldots,x_{\pi(n)}),
        \]
        for all $q(x_1,\ldots, x_n)\in K(x_1,\ldots,x_n)$.
    First of all, we want to prove that $\sigma$ is a 
    group homomorphism:
    \begin{align*}
        \sigma_\rho\sigma_\pi(q(x_1,\ldots,x_n))=
        &\sigma_\rho(q(x_{\pi(1)},\ldots,x_{\pi(n)}))=\\
        &q(x_{\rho\pi(1)},\ldots,x_{\rho\pi(n)})=\sigma_{\rho\pi}(q(x_1,\ldots,x_n))
    \end{align*}
    Moreover, by definition of $G$, $\sigma$ is surjective.
    Finally, $\sigma$ is also injective, i.e $\ker\sigma=\{id\}$.
    Indeed if $\pi\neq id$, there exists $i\in\{1,\dots,n\}$
    such that $\sigma(i)\neq i$.
    Hence $\sigma_\pi(x_i)=x_{\pi(i)}\neq x_i$.

    Now, by Artin's theorem, we know that 
    $\Gal(E/F)=G\cong S_n$,
    in particular, $E/F$ is a Galois extension.
    
    \item Cayley's Theorem states that each finite group is isomorphic to a subgroup of $S_n$ for some $n$.
    So take a finite group $H$, we can assume that $H$
    is a subgroup of $S_n$ for some $n$.
    By the previous part of the exercise we have that
    $\Gal(E/F)\cong S_n$.
    By the Galois correspondence, $H$ corresponds to 
    the intermediate field ${}^HE$ and $\Gal(E/{}^HE)\cong H$.

    More precisely, the Cayley embedding is given by
    \[
    \lambda: G\to S_G\quad \lambda(g)=\lambda_g:h\mapsto gh.
    \]
    We can then write $E=K((x_g)_{g\in G})$
    and $\sigma:G\to \Aut(E/K)$ given by $\sigma(h)=\sigma_h:q((x_g)_{g\in G})\mapsto q((x_{hg})_{g\in G})$.\qedhere
    \end{enumerate}
\end{sol}



\begin{ex}
\label{10.3}
    Give an explicit description of $\mathrm{norm}_{L/K}$ and $\mathrm{trace}_{L/K}$ for the following field extensions:
    \begin{enumerate}[label=(\roman*)]
        \item $\C/\R$,
        \item $\Q[\sqrt{d}]/\Q$, $d$ squarefree,
        \item $\Q[\sqrt[3]{2}]/\Q$

        \noindent \textit{Hint:} Represent an element of $\Q[\sqrt[3]{2}]$ as $x = a + b \sqrt[3]{2} + c \sqrt[3]{4}$, argue that
        \begin{align*}
        \mathrm{norm}_{\Q[\sqrt[3]{2}]/\Q}(x) & = (a + b \sqrt[3]{2} + c \sqrt[3]{4}) \cdot (a + b\xi \sqrt[3]{2} + c\xi^2 \sqrt[3]{4}) \\
        & \quad \cdot (a + b\xi^2 \sqrt[3]{2} + c \xi\sqrt[3]{4})
        \end{align*}
        where $\xi \neq 1$ is a third root of unity. Multiplying this out would give you $27$ summands - you only need those which do not contain a rational power of $\sqrt[3]{2}$ (why?). Then sum everything up by using the fact that $\xi + \xi^2 = -1$.
        \item $K(t)/K(t^p)$, where $\mathrm{char}\ K = p > 0$.
    \end{enumerate}
\end{ex}
\begin{sol}
    ~
    \begin{enumerate}[label=(\roman*)]
     \item $\Hom(\C/\R,\C/\R)=\{id, \sigma\}$,
     where $\sigma(a+ib)=a-ib$.
     Hence 
     \[
     \mathrm{norm}_{\C/\R}(a+ib)=(a+ib)\sigma(a+ib)=(a+ib)(a-ib)=a^2+b^2=|a+ib|^2.
     \]
     \[
     \mathrm{trace}_{\C/\R}(a+ib)=(a+ib)+\sigma(a+ib)=(a+ib)+(a-ib)=2a=2Re(a+ib).
     \]
     \item $\Hom(\Q[\sqrt{d}]/\Q,\overline{\Q}/\Q)=\{id, \sigma\}$,
     where $\sigma(a+\sqrt{d}b)=a-\sqrt{d}b$.
     Hence
     \[
     \mathrm{norm}_{\Q[\sqrt{d}]/\Q}(a+ib)=(a+\sqrt{d}b)\sigma(a+\sqrt{d}b)=(a+\sqrt{d}b)(a-\sqrt{d}b)=a^2-db^2.
     \]
     \[\mathrm{trace}_{\Q[\sqrt{d}]/\Q}(a+\sqrt{d}b)=(a+\sqrt{d}b)+\sigma(a+\sqrt{d}b)=(a+ib)+(a-\sqrt{d}b)=2a.
     \]
     \item The minimal polynomial of $\sqrt[3]{2}$ over
     $\Q$ is $f(\sqrt[3]{2},\Q)=X^3-2$, as it is 
     irreducible (using Eisenstein for $p=2$).
     Moreover, his roots in $\C$ are 
     \[
     \sqrt[3]{2},\sqrt[3]{2}\xi\text{ and }\sqrt[3]{2}\xi^2,
     \]
     where $\xi$ is a primitive third root of unit 
     over $\Q$.
     So if $\tau\in\Hom(\Q[\sqrt[3]{2}]/\Q, \overline{\Q}/\Q)$,
     then $\tau(\sqrt[3]{2})=\sqrt[3]{2}\xi^i$ for 
     some $i\in\{0,1,2\}$.
     More precisely,
     \[
     \Hom(\Q[\sqrt[3]{2}]/\Q,\overline{\Q}/\Q)=\{id, \sigma, \sigma^2\},
     \text{ where }
     \sigma:\sqrt[3]{2}\mapsto \sqrt[3]{2}\xi.
     \]
     Now we want to prove that
     $\{1,\sqrt[3]{2},\sqrt[3]{4}\}$ are linearly independent over $\Q(\xi)$.
     To prove this it is enough to show that 
     $[\Q(\sqrt[3]{2},\xi):\Q(\xi)]=3$.
     So we have the following situation:
     \begin{center}
    \begin{tikzcd}
	& {E=\Q(\sqrt[3]{2},\xi)}\arrow[rd,no head]\\
	{\Q(\sqrt[3]{2})}\arrow[ru,no head,"\leq 2"]&& {\Q(\xi)}\arrow[ld,no head,"2"]  \\
	& {\Q}\arrow[lu,no head,"3"] 
    \end{tikzcd}
    \end{center}
    And we know that $[\Q(\xi):\Q]=2$ because 
    $\xi\notin\Q$ and is a root of
    $$X^3-1=(X-1)(X^2+x=1),$$
    so $\xi$ has degree 2 over $\Q$.

    Therefore, denoting $E=\Q(\sqrt[3]{2},\xi)$,
    \[
    [E:\Q]=[E:\Q(\xi)][\Q(\xi):\Q]=[E:\Q(\xi)]2
    \]
    but also
    \[
    [E:\Q]=[E:\Q(\sqrt[3]{2})][\Q(\sqrt[3]{2}):\Q]=[E:\Q(\sqrt[3]{2})]3.
    \]
    So the only possibility is that $[E:\Q]=6$ and 
    $[E:\Q(\sqrt[3]{2})]=3$.
     
     In particular $\{1,\sqrt[3]{2},\sqrt[3]{4}\}$ is a
     basis of $E$ over $\Q(\xi)$.
     Hence
     \begin{align*}
     \mathrm{trace}_{\Q[\sqrt[3]{2}]/\Q}&(a+b\sqrt[3]{2}+c\sqrt[3]{4})=\\
     &(a+b\sqrt[3]{2}+c\sqrt[3]{4})+\sigma(a+b\sqrt[3]{2}+c\sqrt[3]{4})+\sigma^2(a+b\sqrt[3]{2}+c\sqrt[3]{4})=\\
     &(a+b\sqrt[3]{2}+c\sqrt[3]{4})+(a+b\xi\sqrt[3]{2}+c\xi^2\sqrt[3]{4})+(a+b\xi^2\sqrt[3]{2}+c\xi^4\sqrt[3]{4})=\\
      &3a+b\sqrt[3]{2}(1+\xi+\xi^2)+c\sqrt[3]{4}(1+\xi+\xi^2)
     \end{align*}
    and       
     \begin{align*}
         \mathrm{norm}_{\Q[\sqrt[3]{2}]/\Q}&(a+b\sqrt[3]{2}+c\sqrt[3]{4}) \\
      = &(a+b\sqrt[3]{2}+c\sqrt[3]{4})\cdot \sigma(a+b\sqrt[3]{2}+c\sqrt[3]{4}) \cdot \sigma^2(a+b\sqrt[3]{2}+c\sqrt[3]{4}) \\
      = & (a+b\sqrt[3]{2}+c\sqrt[3]{4}) \cdot (a+b\xi\sqrt[3]{2}+c\xi^2\sqrt[3]{4}) \cdot (a+b\xi^2\sqrt[3]{2}+c\xi\sqrt[3]{4}) = \ldots
     \end{align*}

     As   $\{1,\sqrt[3]{2},\sqrt[3]{4}\}$ are linearly independent over $\Q(\xi)$ and the norm of an element in $\Q[\sqrt[3]{2}]$ is in $\Q$, we can ignore products that involve a rational multiple of $\sqrt[3]{2}$ resp. $\sqrt[3]{4}$. Note also that $\xi + \xi^2 = -1$.

     \begin{align*}
         \ldots  = & \ a^3 + 2b^2 + 4c^3 + 2abc(\xi \cdot \xi + \xi^2 \cdot \xi^2 + \xi + \xi^2 + \xi + \xi^2) \\
         = & \ a^3 + 2b^2 + 4c^3 -6abc.
     \end{align*}

     \item The extension $K(t)/K(t^p)$ is generated by the primitive element $t$, i.e. $K(t) = K(t^p)(t)$. Furthermore, $f(t,K(t^p)) = X^p - t^p = (X-t)^p$. This proves that $K(t)/K(t^p)$ is purely inseparable. In particular, there is only one embedding into an algebraic closure of $K(t^p)$ which can be taken to be the identity of $K(t)$.

    Therefore,
    \[
    \mathrm{trace}_{K(t)/K(t^p)}(x) = [K(t):K(t^p)]_{\mathrm{ins}} \cdot x = px = 0.
    \]
    Furthermore,
    \[
    \mathrm{norm}_{K(t)/K(t^p)}(x) = x^{[K(t):K(t^p)]_{\mathrm{ins}}} = x^p
    \]
    is the Frobenius endomorphism of $K(t)$.\qedhere
\end{enumerate}
\end{sol}

\begin{ex}
\label{10.4}~
    \begin{enumerate}[label=(\roman*)]
        \item Let $L/K$ be a finite extension of fields. Prove that $\mathrm{trace}_{L/K} \neq 0$ if and only if $L/K$ is separable.

    \noindent \textit{Hint:} Theorem 5.13.
        \item Let $L/K$ be a cyclic field extension and let $\Gal(L/K) = \left\langle \tau \right\rangle$. For an element $a \in L$, prove the equivalence
        \[
        \mathrm{trace}_{L/K}(a) \Leftrightarrow \exists c \in L: a = c - \tau c.
        \]
        \noindent \textit{Hint:} Determine $\dim_K \ker (c \mapsto c- \tau c)$ and $\dim_K \ker (\mathrm{trace}_{L/K})$ in order to prove that $\im(c \mapsto c - \tau c) = \ker (\mathrm{trace}_{L/K})$.
    \end{enumerate}
\end{ex}
\begin{sol}
    \begin{enumerate}[label=(\roman*)]
        \item Suppose that $L/K$ is inseparable. Then $K$ has characteristic $p$ for some prime $p$ and and $[L:K]_{\mathrm{ins}} = p^n$ for some $n > 0$. But then, for all $x \in L$,
        \begin{align*}
            \mathrm{trace}_{L/K}(x) & = [L:K]_{\mathrm{ins}} \cdot \sum_{\sigma \in \Hom(L/K,\Bar{L}/K)} \sigma(x) \\
            & = \underbrace{p^n}_{=0} \cdot \sum_{\sigma \in \Hom(L/K,\Bar{L}/K)} \sigma(x) = 0.
        \end{align*}

        On the other hand, suppose that $L/K$ is separable. Then, by Theorem 5.13, the set $\Hom(L/K,\Bar{L}/K)$ is linearly independent. Furthermore, as $L/K$ is separable, $[L:K]_{\mathrm{ins}} = 1$. This implies that
        \[
        \mathrm{trace}_{L/K} = \sum_{\sigma \in \Hom(L/K,\Bar{L}/K)} \sigma \not \equiv 0.
        \]

        \item Let $\gamma: L \to L$; $c \mapsto c - \tau c$. Our goal is to show that $\im\ \gamma = \ker \mathrm{trace}_{L/K}$.

        First of all, $\im \gamma \subseteq  \ker \mathrm{trace}_{L/K}$, as for all $x \in L$,
        \begin{align*}
        \mathrm{trace}_{L/K}(\gamma(x)) & = \mathrm{trace}_{L/K}(x - \tau x) \\
        & = \sum_{\sigma \in \Gal(L/K)} \sigma(x) - \tau(x) \\
        & =  \sum_{\sigma \in \Gal(L/K)} \sigma(x) - \sum_{\sigma \in \Gal(L/K)} \sigma \tau(x) \\
        & = \sum_{\sigma \in \Gal(L/K)} \sigma(x) - \sum_{\sigma^{\prime} \in \Gal(L/K)} \sigma^{\prime}(x) \quad (\sigma^{\prime} = \sigma \tau ) \\
        & = 0.
        \end{align*}

        Let us now have a look at the dimensions! We know that
        \[
        \ker \gamma = \{ x \in L: \gamma x = 0 \} = \{ x \in L : \tau x = x \} = {}^{\left\langle \tau \right\rangle} L = {}^{\Gal(L/K)}L = K. 
        \]
        Therefore $\dim_K \ker \gamma = 1$ which implies that
        \[
        \dim \im\ \gamma = \dim_KL - \dim_K \ker \gamma = \dim_KL - 1.
        \]

        On the other hand, we have shown that $\mathrm{trace}_{L/K} \neq 0$, therefore $\im\  \mathrm{trace}_{L/K} = K$, i.e. $\dim_K \mathrm{trace}_{L/K} = 1$ which tells us that
        \[
        \dim_K \ker \mathrm{trace}_{L/K} = \dim_KL - \dim_K \im \ \mathrm{trace}_{L/K} = \dim_KL - 1.
        \]

        To conclude, we have shown that
        \[
        \dim \im\ \gamma  = \dim_KL - 1 = \dim_K \ker \mathrm{trace}_{L/K}.
        \]
        As $\im\ \gamma \subseteq \ker \mathrm{trace}_{L/K}$, this shows that $\im\ \gamma = \ker \mathrm{trace}_{L/K}$.\qedhere
    \end{enumerate}
\end{sol}


\begin{ex}
\label{10.5}
    Let $L/K$ be an extension with $[L:K] = 2$ and $\mathrm{char}\ K \neq 2$. Express $\mathrm{norm}_{L/K}(x)$ in terms of $\mathrm{trace}_{L/K}(x)$ and $\mathrm{trace}_{L/K}(x^2)$.
\end{ex}
\begin{sol}
    Note first that $L/K$ is separable, as $2 \nmid [L:K]$. Therefore $\Hom(L/K,\overline{L}/K)$ has two elements, call them $\sigma,\tau$. Then
    \begin{align*}
        \mathrm{norm}_{L/K}(x) & = \sigma(x)\tau(x), \\
        \mathrm{trace}_{L/K}(x) & = \sigma(x)+ \tau(x) \\
         \mathrm{trace}_{L/K}(x^2) & = \sigma(x^2) + \tau(x^2) \\
         & = \sigma(x)^2 + \tau(x)^2.
    \end{align*}
    From this we derive
    \begin{align*}
        (\mathrm{trace}_{L/K}(x))^2 & = \sigma(x)^2 + \tau(x)^2 + 2 \sigma(x)\tau(x) \\
        & = \mathrm{trace}_{L/K}(x^2) + 2 \mathrm{norm}_{L/K}(x) \\
       \Rightarrow \ (\mathrm{trace}_{L/K}(x))^2 - \mathrm{trace}_{L/K}(x^2) & = 2 \mathrm{norm}_{L/K}(x) \\
       \Rightarrow \ \frac{1}{2} \left( (\mathrm{trace}_{L/K}(x))^2 - \mathrm{trace}_{L/K}(x^2) \right) & =  \mathrm{norm}_{L/K}(x).\qedhere
    \end{align*}
\end{sol}

\newpage

\section{Week 11}

\begin{ex}
\label{11.1}
    Prove that $f(X)=X^4-aX^2+b\in \Q[X]$ is solvable by radicals
    and determine a radical tower for $R/\Q$,
    where $R$ contains a decomposition field of $f$ over $\Q$.
\end{ex}

\begin{sol}
    We first solve the equation $f(X) = 0$ by substituting $Y = X^2$. The resulting equation is
    \[
    Y^2 - aY + b = 0
    \]
    whose solutions are $y_{1,2} = \frac{a}{2} \pm \sqrt{\left(\frac{a}{2}\right) - b}$.
    
    Let $K_0 = \Q$. We see that
    \[
    K_1 = K_0(y_1,y_2) = \Q\left( \sqrt{\left(\frac{a}{2}\right)^2 - b} \right).
    \]

    Note that the solutions of $f(X) = 0$ are the solutions of $X^2 = y_1$ resp. $X^2 = y_2$. Let $x_{1,2}$ be the solutions of $X^2 = y_1$, then $x_{1,2} = \pm \sqrt{\frac{a}{2} + \sqrt{\left(\frac{a}{2}\right)^2 - b}}$. Similarly, the solutions of $X^2 = y_2$ are $ \pm \sqrt{\frac{a}{2} - \sqrt{\left(\frac{a}{2}\right)^2 - b}}$.

    We therefore get
    \[
    K_2 = K_1(x_1,x_2) = K_1\left( \sqrt{\frac{a}{2} + \sqrt{\left(\frac{a}{2}\right)^2 - b}} \right)
    \]
    and
    \[
    K_3 = K_2(x_3,x_4) = K_2\left( \sqrt{\frac{a}{2} - \sqrt{\left(\frac{a}{2}\right)^2 - b}} \right).
    \]
    It is clear from the construction that $K_0 \subseteq K_1 \subseteq \ldots \subseteq K_3$ is a radical tower and that $K(x_1,\ldots,x_4) \subseteq K_3$.  
\end{sol}



\begin{ex}
\label{11.2}
    Let $p$ be prime and suppose that $a \in \Q$ is \emph{not} a $p$-th power in $\Q$. Let $E$ be the decomposition field of $f = X^p-a \in \Q[X]$. Describe $\Gal(E/\Q)$ - does it look familiar to you?

    \noindent \textit{Hint:} Observe first that $E = \Q(\omega,\sqrt[p]{a})$ where $\omega$ is a $p$-th root of unity. An element $\varphi \in \Gal(E/\Q)$ can now be constructed by extending an element in $\Q(\omega)/\Q \cong (\Z/p)^{\times}$ by assigning a value for $\varphi(\sqrt[p]{a})$.
\end{ex}

\begin{sol}
    Note that the solutions of $f(X) = X^p - a = 0$ are given by $\omega^i \sqrt[p]{a}$ where $\omega$ is a primitive $p$-th root of unity. It is easily checked that the decomposition field of $f$ is $E = \Q(\omega,\sqrt[p]{a})$.

    We first show that $[\Q(\sqrt[p]{a}):\Q] = p$. If $p = 2$, this is clear from the fact that $a$ is not a square in $\Q$, so let us assume $p > 2$. We have to show that $f(\sqrt[p]{a},\Q) = X^p-a$ which amounts to showing that $X^p-a$ is irreducible. Let $h \in \Q[X]$ be a polynomial with $h|X^p-a$, then
    \[
    h = \prod_{j=0}^{\deg h -1} (X - \omega^{i_j}\sqrt[p]{a})
    \]
    for some - distinct - indices $0 \leq i_j \leq p-1$. Then
    \[
    |h(0)| = \prod_{j=0}^{\deg h -1} |-  \omega^{i_j}\sqrt[p]{a}| = \sqrt[p]{|a|}^{\deg h} =: c \in \Q.
    \]
    In particular, $c^p = |a|^{\deg h}$ is a $p$-th power in $\Q$. Suppose that $0 < \deg h < p$, then we can choose integers $m,n$ such that $m\deg h + np = 1$. Therefore $|a|^{(\deg h) \cdot m} \cdot |a|^{pn} = |a|$ would also be a $p$-th power in $\Q$ which, if $p > 2$, contradicts our assumption that $a$ is not a $p$-th power in $\Q$. We have therefore shown that $f(\sqrt[p]{a}) = X^p-a$ and $[\Q(\sqrt[p]{a}):\Q] = p$.

    Note that $\Q(\omega)/\Q$ is a normal extension. We claim that $f(\omega,\Q) = g = \frac{X^p-1}{X-1}$. We calculate
    \[
    g(\omega) = \frac{\omega^p-1}{\omega-1} = \frac{1-1}{\omega-1} = 0
    \]
    but we also have to show that $g$ is irreducible. This is the case if and only if $g(X+1)$ is irreducible. We calculate
    \[
    g(X+1) = \frac{(X+1)^p-1}{X} = \frac{\sum_{k=1}^p\binom{p}{k}X^k}{X} \overset{l = k-1}{=} X^p + \sum_{l=0}^{p-2} \binom{p}{l+1}X^l.
    \]
    Note that $g(X+1)$ is monic, and $p \mid \binom{p}{l+1}$ ($0 \leq l \leq p-2$). Its constant coefficient is $\binom{p}{1} = p$ which, however, is not divisible by $p^2$. By the Eisenstein criterion for the prime $p$, $g(X+1)$ is irreducible and so is $g$.
    
    Therefore, $\deg f(\omega,\Q) = p-1 = \varphi(p-1)$ which implies that $\Gal(\Q(\omega)/\Q) \cong \mathcal{U}(\Z/p) = (\Z/p)^{\times}$. We make this action more explicit: if $\alpha \in (\Z/p)^{\times}$ then the element $\varphi_{\alpha} \in \Gal(\Q(\omega)/\Q)$ acts by $\varphi_{\alpha}(\omega) = \omega^{\alpha}$. More generally, $\varphi_{\alpha}(\omega^i) = \omega^{\alpha i}$, where in the exponents, we calculate mod $p$.

    Note that $[\Q(\omega):\Q] = p-1$ and $[\Q(\sqrt[p]{a}):\Q] = p$. As $E$ is the compositum of $\Q(\omega)$ and $\Q(\sqrt[p]{a})$, we deduce that $[E:\Q] \leq p(p-1)$. On the other hand, $[\Q(\omega):\Q]$ and $[\Q(\sqrt[p]{a}):\Q]$ are coprime and have to divide $[E:\Q]$, so $p(p-1) \leq [E:\Q]$. It follows that $[E:\Q] = p(p-1)$.

    If $\varphi \in \Gal(E/\Q)$, then $\varphi(\omega) = \omega^{\alpha}$ for some $\alpha \in (\Z/p)^{\times}$. Furthermore $\varphi(\sqrt[p]{a}) = \omega^{\beta} \sqrt[p]{a}$ for some $\beta \in \Z/p$. There are $p$ possible values for $\varphi(\sqrt[p]{a})$ and $p-1$ possible values for $\varphi(\omega)$. As there are $p(p-1)$ possible combinations and $|\Gal(E/\Q)| = [E:\Q] = p(p-1)$, we conclude that each combination defines an element of $\Gal(E/\Q)$. We can therefore say that
    \[
    \Gal(E/\Q) = \{ \varphi_{\alpha,\beta} : \alpha \in (\Z/p)^{\times}, \beta \in \Z/p\}
    \]
    where $\varphi_{\alpha,\beta} = \omega^{\alpha}$, $\varphi_{\alpha,\beta}(\sqrt[p]{a}) = \omega^{\beta} \sqrt[p]{a}$. Let now $\varphi_{\alpha,\beta},\varphi_{\gamma,\delta} \in \Gal(E/\Q)$, then
    \begin{align*}
        (\varphi_{\alpha,\beta}\varphi_{\gamma,\delta})(\omega) & = \varphi_{\alpha,\beta}(\omega^{\gamma}) = \omega^{\alpha\gamma}, \\
        (\varphi_{\alpha,\beta}\varphi_{\gamma,\delta})(\varphi_{\alpha,\beta}\varphi_{\gamma,\delta})(\sqrt[p]{a}) & = \varphi_{\alpha,\beta} (\omega^{\delta}\sqrt[p]{a}) \\
        & = \omega^{\alpha\delta} \cdot \omega^{\beta} \sqrt[p]{a} \\
        & = \omega^{\alpha \delta + \beta} \sqrt[p]{a}.
    \end{align*}
    This implies that $\varphi_{\alpha,\beta}\varphi_{\gamma,\delta} = \varphi_{\alpha\gamma, \alpha\delta + \beta}$ which shows that
    \[
    \Gal(E/\Q) \cong \left\{ \begin{pmatrix} \alpha & \beta \\ 0 & 1 \end{pmatrix} : \alpha \in (\Z/p)^{\times}, \beta \in \Z/p \right\} \cong \Z/p \rtimes (\Z/p)^{\times} \cong \mathrm{AGL}(p,1),
    \]
    the latter being the group of affine transformations of the $\Z/p$-vector space $\Z/p$.
\end{sol}


\begin{ex}
\label{11.3}
Let $p > 2$ be a prime and $\omega$ a $p$-th root of unity. In this exercise, we consider the cyclotomic field extension $\Q(\omega)/\Q$. 
\begin{enumerate}[label=(\roman*)]
    \item What is $\Gal(\Q(\omega)/\Q)$?
    \item Prove that there is a unique quadratic intermediate extension $L/\Q = \Q(\sqrt{m})/\Q$ for some $m \in \Q$. When is $L/\Q \subseteq \R$?

    \noindent \textit{Hint:} $-1$ is a square in $(\Z/p)^{\times}$ if and only if $p = 4k+1$ for some $k$.
    \item Define the element
    \[
    \gamma = \sum_{k = 0}^{p-1}\omega^{k^2} \in \Q(\omega).
    \]
    Compute $|\gamma|^2 = \gamma \cdot \Bar{\gamma}$.

    \noindent \textit{Hint:} Note that $\overline{\omega^l} = \omega^{-l}$ and multiply everything out.
    \item Prove that $\gamma \in L$ and $\gamma^2 \in \Z$. Use this information to determine $L$!
\end{enumerate}
\end{ex}

\begin{sol}
    \begin{enumerate}[label=(\roman*)]
    \item We first show that $\deg(f(\omega,\Q)) = p-1$. Let $f = X^{p-1} + \ldots + X + 1 = \frac{X^p-1}{X-1}$, then $f(\omega) = \frac{\omega^p-1}{\omega -1} = \frac{0}{\omega - 1} = 0$, therefore $g =\frac{X^p-1}{X-1} | f(\omega,\Q) $. We show that $g$ is irreducible by showing that $g(X+1)$ is irreducible. We calculate
    \begin{align*}
        g(X+1) & = \frac{(X+1)^p-1}{(X+1)-1} \\
        & = \frac{\left(\sum_{k=0}^p \binom{p}{k}X^k\right)-1}{X} \\
        & = \frac{\sum_{k=1}^p\binom{p}{k}X^k}{X} \\
        & = X^{p-1} + \sum_{l=0}^{p-2} \binom{p}{l+1}X^l \quad (l = k-1) \\
        & = X^{p-1} + \left(\sum_{l=1}^{p-2} \binom{p}{l+1}X^l \right) + p.
    \end{align*}
    As $p \mid \binom{p}{l}$ for $1 \leq k \leq k$ and $p^2 \nmid p$, we see by the Eisenstein criterion that $g(X+1)$ is irreducible. Therefore, $g$ is also irreducible.

    As $\deg(f(\omega,\Q)) = \varphi(p-1)$, we see that $\Gal(\Q(\omega)/\Q) \cong \mathcal{U}(\Z/p) = (\Z/p)^{\times}$ where $\alpha \in (\Z/p)^{\times}$ acts as the element $\varphi_{\alpha} \in \Gal(\Q(\omega)/\Q)$ with $\varphi_{\alpha}(\omega) = \omega^{\alpha}$.

    \item $\Z/p$ is a field, therefore $(\Z/p)^{\times}$ is cyclic of order $p-1$. As $2\mid p-1$, we see that $(\Z/p)^{\times} \cong \Gal(\Q(\omega)/\Q)$ has a unique subgroup of index $2$ which corresponds to the unique quadratic intermediate field $L/\Q$ of $\Q(\omega)/\Q$. This one is given by $\{ \varphi^2 : \varphi \in \Gal(\Q(\omega)/\Q) \} \cong ((\Z/p)^{\times})^2$. We note that, by the Galois correspondence,
    \[
    L = \{ x \in \Q(\omega): \forall \alpha \in (\Z/p)^{\times}: \varphi_{\alpha^2}(x) = x \}
    \]

    Let $\psi \in \Gal(\Q(\omega)/\Q$ be defined by $\psi(x) = \overline{x}$, the complex conjugation. This is well-defined since $\Q(\omega)/\Q$ is normal. As $\psi(\omega) = \overline{\omega} = \omega^{-1}$, we conclude that $\psi = \varphi_{-1}$.

    We conclude that
    \begin{align*}
        L \subseteq \R & \Leftrightarrow \forall x \in L: \psi(x) = x \\
        & \Leftrightarrow \forall x \in L: \varphi_{-1}(x) = x \\
        & \Leftrightarrow \varphi_{-1} \in \Gal(\Q(\omega)/\Q) \\
        & \Leftrightarrow -1 \in ((\Z/p)^{\times})^2 \\
        & \Leftrightarrow p \equiv 1 \ (4),
    \end{align*}
    the last step being justified by the hint.

    \item Note that $\omega^{-1} = \Bar{\omega}$. Furthermore, the number $\omega^k$ only depends on the class of $k$ modulo $p$, so we might assume that $k$ comes from $\Z/p$. Using this, we calculate
    \begin{align*}
        |\gamma|^2 = \gamma \cdot \Bar{\gamma} & = \left( \sum_{k \in \Z/p} \omega^{k^2} \right) \cdot \left( \overline{\sum_{l \in \Z/p} \omega^{l^2}} \right) \\
        & = \left( \sum_{k \in \Z/p} \omega^{k^2} \right) \cdot \left( \sum_{l \in \Z/p} \omega^{-l^2} \right) \\
        & = \sum_{k,l \in \Z/p} \omega^{k^2 -l^2} \\
        & = \sum_{k,l \in \Z/p} \omega^{(k+l)(k-l)}
    \end{align*}
    We now substitute $k = \frac{a+b}{2}$ and $l = \frac{a-b}{2}$ ($a,b \in \Z/p$). This is possible since $p$ is odd.
    \begin{align*}
        \ldots,& = \sum_{a,b \in \Z/p}\omega^{ab} \\
        & = \sum_{a,b \in \Z/p;\  ab = 0} \omega^0 + \sum_{a,b \in \Z/p;\ ab \neq 0} \omega^{ab} \\
        & = |\{ (a,b) \in \Z/p^2 : ab = 0 \}| + \sum_{c \in (\Z/p)^{\times}} \sum_{a,b \in \Z/p^2;\ ab = c} \omega^c  \\
        & = |\{ (a,b) \in \Z/p^2 : a = 0 \vee b = 0 \}| + \sum_{c \in (\Z/p)^{\times}} |\{(a,b) \in \Z/p^2: ab = c \}|\cdot \omega^c \\
        & = 2p - 1 +  \sum_{c \in (\Z/p)^{\times}} |\{(a,a^{-1}c) \in \Z/p^2: a \neq 0 \}|\cdot \omega^c \\
        & = p + p-1 + \sum_{c \in (\Z/p)^{\times}} (p-1) \omega^c \\
        & = p + (p-1) \cdot \sum_{c \in \Z/p} \omega^c \\
        & = p + (p-1) \cdot \sum_{m=0}^{p-1}\omega^m \\
        & = p + (p-1) \cdot \frac{1-\omega^p}{1-\omega} \\
        & = p + (p-1) \cdot 0 = p.
    \end{align*}
    \item We first show that $\gamma \subseteq L$. $L$ is the field fixed by all $\varphi_{\alpha^2}$ ($\alpha \in (\Z/p)^{\times}$), so we check that $\gamma$ is indeed fixed by all these elements:
    \begin{align*}
        \varphi_{\alpha^2}(\gamma) & = \varphi_{\alpha^2}\left( \sum_{a \in \Z/p} \omega^{a^2} \right) \\
        & = \sum_{a \in \Z/p} \omega^{(\alpha a)^2} \\
        & \overset{b = \alpha a}{=} \sum_{b \in \Z/p} \omega^{b^2} = \gamma.
    \end{align*}
    Therefore, $\gamma \in L$.

    If $p \equiv 1 \ (4)$, we have seen in part (ii) that $L \subseteq \R$. Therefore, $\gamma \in \R$ which shows that
    \[
    p = |\gamma|^2 = \gamma^2 \Rightarrow \gamma \in \{ \pm \sqrt{p} \}.
    \]
    Therefore $\Q(\sqrt{p}) = \Q(\gamma) \subseteq L$. As $[\Q(\sqrt{p}):\Q] = [L:\Q] = 2$, this shows $L = \Q(\sqrt{p})$.

    The case $p \equiv 1 \ (3)$ is slightly harder. Write $Q = ((\Z/p)^{\times})^2$, then $-1 \not \in Q$ and, as $[(\Z/p)^{\times}:Q] = 2$, we see that $(\Z/p)^{\times} = Q \sqcup -Q$. Furthermore,
    \[
    \gamma = 1 + 2 \cdot \sum_{a \in Q} \omega^a,
    \]
    as each nonzero square in $(\Z/p)^{\times}$ is representable as a square of exactly two different elements. As $\overline{\gamma} = \varphi_{-1}(\gamma)$, we see that
    \[
    \overline{\gamma} = 1 + 2 \cdot \sum_{a \in Q} \omega^{-a} = 1 + 2 \cdot \sum_{a^{\prime} \in -Q} \omega^{\prime}.
    \]
    Now we calculate
    \begin{align*}
    \gamma + \overline{\gamma} & = 1 + 2 \cdot \sum_{a \in Q} \omega^a + 1 + 2 \cdot \sum_{a \in -Q} \omega^a \\
    & = 2 + 2 \cdot \sum_{a \in Q \sqcup -Q} \omega^a \\
    & = 2 \omega^0 + 2 \cdot \sum_{a \in (\Z/p)^{\times}} \omega^a \\
    & = 2 \cdot \sum_{a \in \Z/p} \omega^a \\
    & = 2 \cdot \sum_{m = 0}^{p-1} \omega^m = 0
    \end{align*}
    where the last step is the evaluation of the same geometric sum as in the previous part of the exercise. We conclude that $\overline{\gamma} = - \gamma$. Therefore,
    \[
    p = |\gamma|^2 = \gamma \cdot \overline{\gamma} = \gamma \cdot (-\gamma) = -\gamma^2.
    \]
    This implies $\gamma \in \{ \pm i \sqrt{p} \}$. As in the other case, we conclude that $L = \Q(i \sqrt{p})$.

    \medskip

    \noindent \textit{Remark:} getting the correct sign of $\gamma$ is extremely (!) difficult. However, for this exercise, it is sufficient to determine $\gamma$ up to a potential sign. One can prove, for example using complex analysis, that
    \[
    \gamma = \begin{cases}
    \sqrt{p} & p \equiv 1 \ (4) \\
    i\sqrt{p} & p \equiv 3 \ (4).
    \end{cases}\qedhere
    \]
    \end{enumerate}
\end{sol}

\begin{ex}
\label{11.4}
    Let $\omega$ be an $8$-th root of unity.
    \begin{enumerate}[label=(\roman*)]
    \item What is $\Gal(\Q(\omega)/\Q)$?
    
    \item Prove that $\Q(\omega)/\Q$ has exactly \emph{three} quadratic intermediate extensions. Determine them!

    \noindent \textit{Hint:} Compute $\omega^2$ and $(\omega + \omega^{-1})^2$.
\end{enumerate}
\end{ex}
\begin{sol}
\begin{enumerate}[label=(\roman*)]
\item We first show that $\deg f(\omega,\Q) = 4 = \varphi(8)$. As $\omega$ is a primitive $8$-th root of unity, we see that $(\omega^4)^2 = \omega^8 = 1$ but $\omega^4 \neq 1$, therefore $\omega^4 = -1$ which shows that $f(\omega,\Q) | X^4 + 1$. Now, $g = X^4+1$ is irreducible if $g(X+1)$ is irreducible. We calculate
\[
g(X+1) = (X +1)^4 +1 = X^4 + 4X^3 + 6X^2 + 4X + 1 + 1 = X^4 + 4X^3 + 6X^2 + 4X + 2.
\]
Using the Eisenstein criterion with $p=2$ proves the irreducibility of $g(X+1)$ and therefore of $g$. $f(\omega,\Q) = X^4 + 1$ has degree $4 = \varphi(8)$ which implies $\Gal(\Q(\omega)/\Q) \cong \mathcal{U}(\Z/8) = \{ \Bar{1},\Bar{3}, \Bar{5}, \Bar{7}\}$.

It is easily checked that $x^2 = \Bar{1}$ for all $x \in \mathcal{U}(\Z/8)$. Therefore, $\Gal(\Q(\omega)/\Q)$ is a Klein Four Group!
\item The subgroups of $\mathcal{U}(\Z/8)$ are $\{ \Bar{1} \}$, $\{ \Bar{1}, \Bar{3} \}$, $\{ \Bar{1}, \Bar{5} \}$, $\{ \Bar{1}, \Bar{7} \}$ and $\mathcal{U}(\Z/8)$ itself. Those subgroups of index $2$ correspond two quadratic intermediate extensions. As there are three subgroups of index $2$, there are three intermediate extensions.

As $\omega^4 = - 1$ (see the previous part), we see that $(\omega^2)^2 = -1$. Therefore, $\omega^2 = \pm i$, depending on the choice of $\omega$. This shows that $\Q(i) \subseteq \Q(\omega)$.

On the other hand,
\[
(\omega + \omega^{-1})^2 = \omega^2 + \omega^{-2} + 2\omega \omega^{-1} = i + i^{-1} + 2 = i - i + 2 = 2,
\]
which shows that $\omega + \omega^{-1} = \pm \sqrt{2}$, again depending on the choice of $\omega$. Thus, $\Q(\sqrt{2}) \subseteq \Q(\omega)$. As $i, \sqrt{2} \in \Q(\omega)$, we also find that $i \sqrt{2} \in \Q(\omega)$ which proves that $\Q(i \sqrt{2}) \subseteq \Q(\omega)$.\qedhere
\end{enumerate}
\end{sol}


\newpage

\section{Week 12}

\begin{ex}
\label{12.1}
    Let $a\in\Z_{>0}$ and $K=\Q(i\sqrt{a})$.
    Then $K/\Q$ cannot be an intermediate extension
    of a cyclic extension of degree multiple of 4. 

    \noindent \textit{Hint:} Suppose there was a cyclic Galois extension $L/\Q$ with $K \subseteq L$ and $[L:K] = 4d$. Use the Galois correspondence to restrict the problem to an intermediate extension $L^{\prime}/\Q$ with $[L^{\prime}:\Q] = 4$ and $K \subseteq L^{\prime}$. Now show that complex conjugation restricts to a nontrivial automorphism of $L^{\prime}/\Q$. What can you say about its fixed field?
\end{ex}

\begin{sol}
    Assume that there exists a cyclic extension
    $L/\Q$ of degree $4d$ for some $d\geq 1$.
    Then $G=\Gal(L/\Q)$ is a cyclic group of order
    $4d$. So there exists a unique subgroup $H$
    $G$ of index 4.
    Since the subgroup of a cyclic group are ordered by divisibility of the order, we get
    $H\subseteq \Gal(K/\Q)$.
    Hence, using Galois correspondence,
    $L'={}^H\supseteq K$ and $L'/\Q$ is a cyclic 
    extension of degree 4.

    Consider now the complex conjugation 
    $\sigma: \Bar{\Q}\to \Bar{\Q}$.
    It's restriction to $L'$ is an element of
    $G'=\Gal(L'/\Q)$ which is not the identity,
    as $\sigma(i\sqrt{a})=-i\sqrt{a}$.

Therefore we have a subgroup of $G'$
$J=\langle \sigma|_{L'}\rangle$ of order 2.
Thus, by uniqueness of subgroup of order 2 in
a cyclic group, $J=\Gal(L'/K)$.
On the other hand, the fixed field of 
$\Gal(L'/K)$ is $K=\Q(i\sqrt{a})$, while the one fixed by $J$ is
$${}^J L'=\{z\in L'\mid \Bar{z}=z\}=L'\cap\R.$$
But $K$ is not contained in $\R$ since 
$i\sqrt{a}$ is not real.
\end{sol}

\begin{ex}
\label{12.2}
    Compute the Galois group of $f(X)=X^{104}-1$
    over $\Z/{13}$.

    \noindent \textit{Hint:} $104$ is divisible by $13$.
\end{ex}
\begin{sol}
    First of all note that $104=13\cdot 8$,
so
$$X^{104}-1=(X^8)^{13}-1=(X^8-1)^{13}.$$
Thus the Galois group of $f$ is the 
same as the Galois group of $g=X^8-1$.

Note that $K$ is the decomposition field of $g=X^8-1$ if and only if $K$ is as small as possible while containing $8$ elements such that $\alpha^8 = 1$. For $n > 0$, let $F_n$ be the field with $13^n$ elements, then $F_n^{\times}$ is cyclic of order $13^n - 1$. Therefore, $F_n^{\times}$ contains a subgroup of size $8$ if and only if $8|13^n-1$. One quickly finds that $n=2$ is the smallest $n > 0$ with that property. Therefore, $K = F_2$. As this implies that $[K:\Z/{13}] = 2$, we get that $\Gal(K/\Z/{13} \cong C_2$. The generator of $\Gal(K/\Z/{13})$ is the Frobenius automorphism $\Phi(x) = x^{13}$.

Another way to determine the degree is the following: first decompose
\[
X^8 - 1 = (X^4 + 1) (X^2 + 1)(X+1)(X-1).
\]
Clearly, the factors $X+1$ and $X-1$ do not contribute new elements and therefore can be discarded. Note that the element $- 1 \in \Z/{13}$ is a square as $-1 = 5^2 = 8^2$. Therefore, $X^2 + 1 = (X-5)(X-8)$ can also be discarded! Furthermore, this allows us to factorize
\[
X^4 +1 = (X^2 -5)(X^2-8).
\]
Trying out elements, we find that $-5$ is \emph{not} a square in $\Z/{13}$. Let $\alpha \not\in \Z/{13}$ be an element with $\alpha^2 = 5$, then $(5\alpha)^2 = 5^2 \alpha^2 = -5 = 8$. Therefore, we can factorize
\[
X^4 + 1 = (X - \alpha)(X + \alpha)(X - 5\alpha)(X + 5\alpha).
\]
It turns out that $K = \Z/{13}(\alpha)$ with $f(\alpha,\Z/{13}) = X^2 - 5$. Therefore, $[K:\Z/{13}] = 2$ and we can finish as before.
\end{sol}


\begin{ex}
\label{12.3}
Let $f=X^5-5\in \Q[X]$, $E$ be a decomposition field of $f$ over $\Q$ and $G=\Gal(E/\Q)$.
\begin{enumerate}[label=(\roman*)]
    \item Determine $[E:\Q]$.
    \item Prove (using Sylow's Theorems)
    that $G$ has a unique subgroup of 
    order 5 and that $G\cong \Z/5\rtimes \Z/4$.
    \item Find all intermediate extensions of $E/\Q$,
    specifying which ones are normal.
\end{enumerate} 
\end{ex}

\begin{sol}
    \begin{enumerate}[label=(\roman*)]

        \item It is easy to see that $E = \Q(\omega, \sqrt[5]{5})$ where $\omega$ is a primitive fifth root of unity. We have already shown on the last sheet that $[\Q(\omega):\Q] = 4$ with $\Gal(\Q(\omega):\Q) \cong \Z/5^{\times}$. Furthermore, $f(\sqrt[5]{5},\Q) = X^5 -5$, as the latter is irreducible as an Eisenstein polynomial for $p=5$. This shows $[\Q(\sqrt[5]{5}):\Q]=5$.

        Note that $E = \Q(\omega)\Q(\sqrt[5]{5})$, therefore
        \[
        [E:\Q] \leq [\Q(\omega):\Q][\Q(\sqrt[5]{5}):\Q] = 5\cdot 4 = 20.
        \]
        We also have $4 = [\Q(\omega):\Q] | [E:\Q]$ and $5 = [\Q(\sqrt[5]{5}):\Q] | [E:\Q]$. Therefore, $20 = \lcm(4,5) | [E:\Q]$ which proves $[E:\Q= 20$.

        \item We have shown in the last session that
        \[
        \Gal([E:\Q]) = \{ \varphi_{a,b}: a \in \Z/p, b \in (\Z/p)^{\times} \} \cong \Z/5 \rtimes \Z/5^{\times} \cong \Z/5  \rtimes \Z/4.
        \]
        Here, $\varphi_{a,b}(\omega) = \omega^b$, $\varphi_{a,b}(\sqrt[5]{5}) = \omega^a \sqrt[5]{5}$.

        Furthermore, note that the multiplication rule on $\Z/5 \rtimes \Z/5^{\times}$ is given by
        \[
        (a,b)(c,d) = (a + bc, bd).
        \]

        One can show the unicity of the subgroup of index $5$ as follows:

        Let $n_5$ be the number of $5$-Sylow subgroups of $\Gal(E/\Q)$ which is of order $20$. Then the Sylow theorems tell us that 
        \begin{align*}
            n_5 & \geq 1, \\
            n_5 & | 20, \\
            n_5 & \equiv 1 \mod 5.
        \end{align*}
    This proves that $n_5 = 1$, i.e. there is a unique $5$-Sylow subgroup in $\Gal(E/\Q)$ (this can also be derived from the fact that $\Z/5$ is a normal $5$-Sylow subgroup in $\Z/5 \rtimes \Z/5^{\times}$ and therefore the only $5$-Sylow subgroup).

    \item Abbreviate $G = \Gal(E/\Q)$, then the possible subgroups are of order $1$, $2$, $4$, $5$, $10$, $20$. Those of order $1$ and $20$ are obviously $1$ and $G$ itself. These correspond to the intermediate fields $E$ resp. $\Q$.
    
    Furthermore, we have seen that there is a unique subgroup of size $5$ which is $\left\langle \varphi_{1,1} \right\rangle$. The corresponding intermediate field is the unique extension of degree $4$, i.e. $\Q(\omega)$.

    A subgroup of order $10$ has to contain $\left\langle \varphi_{1,1} \right\rangle$, the unique subgroup of size $5$. Furthermore, it must contain one of the elements of order $2$, say $\varphi_{0,-1}$. But then it contains $\varphi_{1,1}^a \varphi_{0,-1} = \varphi_{a,-1}$ as well. Therefore, there is a unique subgroup of order $10$ which is given by $\left\langle \varphi_{1,1}, \varphi_{0,-1} \right\rangle$. The Galois correspondent is the unique subfield of degree $2$ in $\Q(\omega)$ which is $\Q(\sqrt{5})$, as we have seen in the last session.

    The $2$-Sylow subgroups are of size $4$ and one of them is $\left\langle \varphi_{0,2} \right\rangle$ which is cyclic. The other $2$-Sylows of $G$ are conjugates of this group which are $\left\langle \varphi_{a,2} \right\rangle$ ($a \in \Z/5$). The $5$ corresponding intermediate fields are of degree $5$ and are given by $\Q(\omega^a\sqrt[5]{5})$ ($a \in \in \Z/5$) (note that the $a$'s do not necessarily correspond in subgroups and subfields)

    Squaring the generators $\varphi_{a,2}$, one finds that the elements of order $2$ in $G$ are the elements $\varphi_{a,2}^2 = \varphi_{(a,2)^2} = \varphi_{3a,-1}$ which can also also be parametrized as $\varphi_{a,-1}$ ($a \in \Z/p$). Therefore, the subgroups of order $2$ are $\left\langle \varphi_{a,-1} \right\rangle$ ($a \in \Z/p$). The $5$ Galois correspondents have degree $10$ and are given by $\Q(\sqrt{5},\omega^a\sqrt[5]{5})$ ($a \in \Z/p$)

    We list the subgroups and subfields:

\begin{center}
    \begin{tabular}{c|c| c | c}
        Order &  Subgroup(s) & Degree & Subfield(s) \\ \hline
        1 & $1$ & 20 & $E$\\
        2 & $\left\langle \varphi_{a,-1} \right\rangle$ ($a \in \Z/p$) & 10 & $\Q(\sqrt{5},\omega^a\sqrt[5]{5})$ ($a \in \Z/p$) \\
        4 & $\left\langle \varphi_{a,2} \right\rangle$ ($a \in \Z/5$)  & 5 &  $\Q(\omega^a\sqrt[5]{5})$ ($a \in \Z/5$) \\
        5 & $\left\langle \varphi_{1,1} \right\rangle$ & 4 & $\Q(\omega)$ \\
        10 & $\left\langle \varphi_{1,1}, \varphi_{0,-1} \right\rangle$  & 2 & $\Q(\sqrt{5})$ \\
        20 & $E$ & 1 & $\Q$
        \end{tabular}

\end{center}
    \end{enumerate}
\end{sol}

\begin{ex}
\label{12.4}Let $G$ be a finite group and $M$ a $G$-module. Prove the following statements:
    \begin{enumerate}[label=(\roman*)]
    \item If $M$ is finite and $\gcd(|M|,|G|) = 1$ then $H^1(G,M) = 0$.
    \item If $M$ is finitely generated then $H^1(G,M)$ is finite.
    \end{enumerate}
\end{ex}

\begin{sol}
    For both parts of the exercise, we will use the fact that $|G| \cdot H^1(G,M) = 0$ (Theorem 12.14).

    \begin{enumerate}[label=(\roman*)]

    \item For each $x \in M$, we have $|M| \cdot x = 0$. By construction of $H^1(G,M)$, it becomes clear that also $|M| \cdot \alpha = 0$ for each $\alpha \in H^1(G,M)$. As $\gcd(|M|,|G|) = 1$, we can write $1 = a|M| + b|G|$, therefore for all $\alpha \in H^1(G,M)$:
    \[
    \alpha = 1 \cdot \alpha = a \cdot |M| \cdot \alpha + b \cdot |G| \cdot \alpha = 0 + 0 = 0.
    \]
    Therefore, $H^1(G,M) = 0$.
    \item By construction, $H^1(G,M)$ is finitely generated. Let $\alpha_1, \ldots, \alpha_k$ be generators. As $|G| \cdot \alpha_i = 0$ for all $1 \leq i \leq k$, we deduce that every element of $H^1(G,M)$ is of the form $\sum_{i=1}^kn_i \alpha_i$ with $0 \leq n_i < |G|$ for all $i$. Therefore $|H^1(G,M)| \leq |G|^k$. In particular, $H^1(G,M)$ is finite.\qedhere
    \end{enumerate}
    
\end{sol}

\begin{ex}
\label{12.5}
    Let $K$ be a field that contains all $n$-th roots of unity, i.e. the set $\mu_n = \{ \zeta \in K: \zeta^n = 1 \}$ has $n$ elements. Furthermore, let $L/K$ be a Galois extension.
    \begin{enumerate}[label=(\roman*)]
    \item Denote by $(L^{\times})^n$ the group of $n$-th powers in $L^{\times}$. Check the exactness of the sequence
    \[
    1 \longrightarrow \mu_n \overset{\iota}{\longrightarrow} L^{\times} \overset{\varepsilon}{\longrightarrow} (L^{\times})^n \to 1
    \]
    where $\iota(\zeta) = \zeta$ and $\varepsilon(x) = x^n$. Furthermore, check that this is an exact sequence of $G$-modules.
    \item Write down the exact sequence from Theorem 12.3 and prove the existence of an isomorphism
    \[
    ((L^{\times})^n \cap K)/(K^{\times})^n \cong \Hom(\Gal(L/K),\mu_n).
    \]
    \item Suppose that $n=p$ is prime. What does the previous part of the exercise tell you about intermediate Galois extensions $M/K$ of degree $[M:K] = p$?
    \end{enumerate}
\end{ex}

\begin{sol}
    \begin{enumerate}[label=(\roman*)]

    \item Exactness at $\mu_n$ is clear: as $\iota$ is an injection: therefore, $\ker \iota = 1 = \im\ (1 \to \mu_n)$.

    By construction, $\varepsilon$ is a surjection, which implies $\im\ \varepsilon = (L^{\times})^n = \ker ( (L^{\times})^n \to 1)$. It follows that the sequence is exact at $(L^{\times})^n$.

    Finally,
    \[
    \ker \varepsilon = \{ x \in L^{\times}: x^n = 1 \} = \mu_n = \im\ \iota
    \]
    as all $n$-th roots of unity are already contained in $K$. Therefore, the given sequence is indeed exact.

    We have to show compatibility with the Galois action: let $\sigma \in \Gal(L/K)$ and $\zeta \in \mu_n$, then
    \[
    \iota(\sigma(\zeta)) = \sigma(\zeta)= \sigma(\iota(\zeta)),
    \]
    therefore $\iota$ is a homomorphism of $\Gal(L/K)$-modules. If $x \in L^{\times}$, then 
    \[
    \sigma(\varepsilon(x)) = \sigma(x^n) = \sigma(x)^n = \varepsilon(\sigma(x))
    \]
    which shows that also $\varepsilon$ is a homomorphism of $\Gal(L/K)$-modules.
    \item Write $G = \Gal(L/K)$. The exact cohomology sequence is given by
    \[
    1 \to H^0(G,\mu_n) \overset{\iota^0}{\to} H^0(G,L^{\times}) \overset{\varepsilon^0}{\to} H^0(G,(L^{\times})^n) \overset{\delta}{\to} H^1(G,\mu_n) \overset{\iota^1}{\to} H^1(G,L^{\times}) \overset{\varepsilon^1}{\to} H^1(G,(L^{\times})^n).
    \]
    We ignore the last term in this sequence. By Hilbert's theorem 90, we have $H^1(G,L^{\times}) = 1$. Furthermore, $H^0(G,-)$ associates with a $G$-module its fixed elements, which implies $H^0(G,M) = M \cap K$ for each subgroup $M \leq L^{\times}$. This shows that $H^0(G,\mu_n) = \mu_n$, $H^0(G,L^{\times}) = K^{\times}$ and $H^0(G,(L^{\times})^n) = (L^{\times})^n \cap K$.
    
    Furthermore, $\mu_n \leq K^{\times}$, therefore $\mu_n$ is a trivial $G$-module which implies $H^1(G,\mu_n) \cong \Hom(G,\mu_n)$ (see Example 12.10). The first terms of our sequence are therefore
    \[
    1 \to \mu_n \overset{\iota^0}{\to} K^{\times} \overset{\varepsilon^0}{\to} (L^{\times})^n \cap K \overset{\delta}{\to} \Hom(G,\mu_n) \to 1
    \]
    Where $\iota^0$, $\varepsilon^0$ are given by restriction of $\iota$ resp. $\varepsilon$.

    As $\im\ \varepsilon^0 = (K^{\times})^n$, the exactness of this sequence tells us that
    \[
    \Hom(G,\mu_n) \cong ((L^{\times})^n \cap K) / \im\ \varepsilon^0 = ((L^{\times})^n \cap K) / (K^{\times})^n.
    \]
    Using the construction of $\delta$ in Theorem 12.13, we can make this isomorphism explicit: take an element $[x] \in ((L^{\times})^n \cap K) / (K^{\times})^n$, represented by $x \in (L^{\times})^n \cap K$. Take an $\alpha \in L^{\times}$ with $x = \alpha^n$, then $\delta(x): G \to \mu_n$ is the map given by
    \[
    (\delta(x))(\sigma) = \frac{\sigma(\alpha)}{\alpha}.
    \]
    This induces the isomorphism $\Bar{\delta}: ((L^{\times})^n \cap K) / (K^{\times})^n \to \Hom(G,\mu_n)$; $[x] \mapsto \delta(x)$.

    \item Let $[x] \in ((L^{\times})^p \cap K) / (K^{\times})^p $ be a nontrivial class, then $\alpha^p = x \in K$ for some $x \in L^{\times}$. Consider the extension $[K(\alpha):K]$ which is of degree $p$. We remark that
    \begin{align*}
    \Gal(L/K(\alpha)) & = \{ \sigma \in \Gal(L/K): \sigma(\alpha) = \alpha \} \\
    & = \{ \sigma \in \Gal(L/K): \frac{\sigma(\alpha)}{\alpha} \} \\
    & = \ker \Bar{\delta}([x]).
    \end{align*}
    Each normal subgroup $N \unlhd \Gal(L/K)$ with $[\Gal(L/K):N] = p$ is the kernel of some homomorphism $\varphi: \Gal(L/K) \to \mu_p$. The surjectivity of $\Bar{\delta}$, together with the previous calculation, shows that each such subgroup $N$ is of the form $\Gal(L/K(\alpha))$ for some $\alpha \in L^{\times}$ with $\alpha^n \in K$. This is Lemma 11.12.

    A more careful inspection shows that there is a $1-1$ correspondence between subgroups of $H \leq ((L^{\times})^p \cap K) / (K^{\times})^p$ with $|H| = p$ and subgroups of $J \leq \Hom(\Gal(L/K),\mu_p)$ with $|J| = p$.\qedhere
    \end{enumerate}
\end{sol}




\end{document}
