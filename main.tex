\documentclass[a4paper,10pt,reqno]{amsart}

\input{Exstuff}

\newtheorem{ex}{Exercise}[section]
\newenvironment{sol}
  {\renewcommand\qedsymbol{$\blacksquare$}\begin{proof}[Solution]}
  {\end{proof}}

  \setlength\parindent{0pt}

\usepackage{multicol}
\usepackage{adjustbox}




\title{Exercises Galois Theory and some solutions}

\author{Carsten Dietzel, Silvia Properzi}
%\date{February 2024}
\address[Carsten Dietzel]{Department of Mathematics and Data Science, Vrije Universiteit Brussel, Pleinlaan 2, 1050 Brussel, Belgium}
\email{Carsten.Dietzel@vub.be}


\address[Silvia Properzi]{Department of Mathematics and Data Science, Vrije Universiteit Brussel, Pleinlaan 2, 1050 Brussel, Belgium}
\email{Silvia.Properzi@vub.be}

\begin{document}

\maketitle
\section{Week 1}


\begin{ex}
    Let $\sigma : K\to L$ be a field homomorphism, prove that $\sigma$ is injective.
\end{ex}
\begin{sol}[\textit{using ring theory}]
    We know that $\ker\sigma$ is an ideal of $K$.
    But $K$ is a field so the only possibilities are that $\ker\sigma=\{0\}$ or $\ker\sigma=K$.
    As $\sigma(1)=1\neq 0$, we know that $1\not\in \ker\sigma$, hence $\ker\sigma=\{0\}$ which means that $\sigma$ is injective.
\end{sol}
\begin{sol}[\textit{by hands}]
    Let $a,b\in K$ such that $\sigma(a)=\sigma(b)$.
    Then 
    \[
    \sigma(a-b)=\sigma(a)-\sigma(b)=0.
    \]
    If we assume that $a\neq b$, then $a-b$ is a non-zero element in a field, hence it is invertible.
    Then 
    \[
    1=\sigma(1)=\sigma((a-b)(a-b)^{-1})=
    \sigma(a-b)\sigma((a-b)^{-1})=0,
    \]
    which is a contradiction.
    Therefore $a=b$ and so $\sigma$ is injective.
\end{sol}

\begin{ex}
    Let $K$ be a field, $K_0$ be its prime field and $\sigma:K\to K$ be a field homomorphism. Prove that $\sigma\in\Hom(K/K_0,K/K_0)$.
\end{ex}


\begin{ex}
Let $\Q[i]=\{a+ib\mid a,b\in \Q\}$, $\Q[\sqrt{2}]=\{a+\sqrt{2}b\mid a,b\in\Q\}$ and $\Q(i)=\{\frac{a+ib}{c+di}\mid a,b,c,d\in \Q\}$, $\Q(\sqrt{2})=\{\frac{a+\sqrt{2}b}{c+\sqrt{2}d}\mid a,b,c,d\in\Q\}$.

\begin{enumerate}[label=(\roman*)]
    \item Prove that $\Q(\sqrt{2})=\Q[\sqrt{2}]$ and $\Q(i)=\Q[i]$.
    \item Prove that $\Q(i)$ and $\Q(\sqrt{2})$ are not isomorphic.
\end{enumerate}
\end{ex}

\begin{ex}~
    \begin{enumerate}[label=(\roman*)]
        \item Let $a=\sqrt{2}$ and $b=\sqrt[3]{3}$. Prove that $ab$ is algebraic over $\Q$.
        \item Show that $\sqrt{2} + i$ is algebraic over $\Q$ by finding a nonzero polynomial $f \in \Q[X]$ with $\deg(f) = 4$ such that $f(\sqrt{2} + i) = 0$. What are the other roots of $f$?
    \end{enumerate}
\end{ex}

\begin{sol}~
    \begin{enumerate}[label=(\roman*)]
        \item Observe that $ab^6=\sqrt{2}^6\sqrt[3]{3}^6=8\cdot 9=72$.
        Therefore $ab$ is a root of the polynomial $X^6-72\in\Q[X]$.
        \item let $\alpha=\sqrt{2} + i$. 
        Then $\alpha^2=2+2\sqrt{2}i-1=1+2\sqrt{2}i$ and $\alpha^2-1=2\sqrt{2}i$.
        Squaring both sides of the last equality we have that
        \[
         \alpha^4-2\alpha^2+1=(\alpha^2-1)^2=(2\sqrt{2}i)^2=-8.
        \]
        Therefore the polynomial 
        \[
        f(X)=X^4-2X^2+1+8=X^4-2X^2+9\in \Q[X]
        \]
        is such that $f(\alpha)=0$.

        Looking for other roots of $f$ means to find all $x\in C$
        such that $f(x)=0$, i.e. $x^4-2x^2+9=0$. 
        Going back to how we construct $f$, this equation 
        can also be written as $(x^2-1)^2=(2\sqrt{2}i)^2$.
        Therefore $x^2-1=2\sqrt{2}i$ or $x^2-1=-2\sqrt{2}i$.
        So 
        \[
        x^2=2\sqrt{2}i+1=\alpha^2\qquad\text{ or } \qquad x^2=-2\sqrt{2}i+1=\overline{\alpha^2}=\overline{\alpha}^2,
        \]
        where $\overline{\alpha}$ indicates the complex conjugate of 
        $\alpha$.
        Thus $x=\pm \alpha$ or $x=\pm \overline{\alpha}$ and the 4 roots of $f$ are
        \[
        x_1=\alpha=\sqrt{2} + i, x_2=-\alpha=-\sqrt{2} - i,
        \]
        \[
        x_3=\overline{\alpha}=\sqrt{2} - i, x_4=-\overline{\alpha}=-\sqrt{2} + i.
        \]
    \end{enumerate}
\end{sol}

\begin{ex} Let $p$ be a prime number. Denote by $\binom{n}{k}$ the binomial coefficient ``$n$ over $k$".
    \begin{enumerate}[label=(\roman*)]
    \item Prove that $p$ divides $\binom{p}{k}$ for $1 \leq k \leq p-1$.
    \item Let $K$ be a field of characteristic $p$. Show that the map $\Phi: K \to K$; $x \mapsto x^p$ is a field endomorphism. This map is called the \emph{Frobenius endomorphism} of $K$.
\end{enumerate}
\end{ex}

\begin{ex}
    Let $L/K$ be a field extension and let $M$ be a subring of $L$ that contains $K$. Suppose that $\dim_KM < \infty$.
    \begin{enumerate}[label=(\roman*)]
        \item Prove that for any $\alpha \in M$, there is a nonzero $f \in K[X]$ with $f(\alpha) = 0$.

        \noindent \textit{Hint:} The elements $\alpha^n$ ($n \geq 0$) are linearly dependent.
        
        \item Prove that $M$ is a field.
        
        \noindent \textit{Hint:} For any $0 \neq \alpha \in K$, let $f \in \Q[X]$ be a nonzero polynomial with $f(\alpha) = 0$ whose degree is as small as possible. If $f = \sum_{i=0}^{\infty} a_iX^i$, prove that $a_0 \neq 0$ and use this to construct an inverse of $\alpha$ that lies in $M$.
    \end{enumerate}
\end{ex}

\begin{ex}
Let $K$ be field.
\begin{enumerate}[label=(\roman*)]
    \item Prove that $K[X]$ is a PID.
    \item Let $I\neq \{0\}$ be an ideal of $K[X]$, then there exists a unique monic polynomial that generates $I$ as an ideal.
\end{enumerate}
    
\end{ex}
\newpage
\section{Week 2}

\begin{ex}
    Let $L/K$ be a field extension and let $\alpha, \beta\in L$ such that 
    \[
    [K(\alpha):K]=[K(\beta):K]=2.
    \]
    Assume that the characteristic of $K$ is not 2.
    \begin{enumerate}[label=(\roman*)]
    \item Prove that there is an $\alpha' \in L$ such that $K(\alpha') = K(\alpha)$ and $\alpha'^2 \in K$.
    \item Assume that $\alpha,\beta \in L$ satisfy $\alpha^2,\beta^2 \in K$. Prove that $K(\alpha)=K(\beta)$ if and only if $\frac{\alpha^2}{\beta^2}$ is a square in $K$.
    \item Prove that there is a bijective map 
    \[
    K^\times/(K^\times)^2\longrightarrow \{L\mid L/K \text{ is a field extension with } [L:K] \leq 2\}.
    \]
\end{enumerate}    
\end{ex}
\begin{sol}~
    \begin{enumerate}[label=(\roman*)]
        \item Since $[K(\alpha):K]=2$, $\alpha\not\in K$ and the 3 
        elements $1,\alpha,\alpha^2$ are linearly dependent over $K$.
        Thus there exist $a_0,a_1,a_2\in K$ not all zero such that 
        \[
        a_0+a_1\alpha+a_2\alpha^2=0.
        \]
        If $a_2=0$, then $a_0+a_1\alpha=0$, hence either $a_1=0$ or 
        $\alpha=-a_0/a_1$. 
        If $a_1=0$, the also $a_0=0$. But this is not possible 
        because $(a_0,a_1,a_2)\neq (0,0,0)$.
        On the other hand, if $\alpha=-a_0/a_1$, then $\alpha\in K$, a contradiction.

        So we have that $a_2\neq 0$ and we can divide by $a_2$,
        obtaining
        \[        b+a\alpha+\alpha^2=0, \text{ i.e. } \alpha^2+a\alpha=-b,
        \]
        where $a=a_2^{-1}a_1\in K$ and $b=a_2^{-1}a_0\in K$. 
        Since we assumed that $K$ has not characteristic two,
        we can also complete the square:
        \[
        (\alpha+a/2)^2=\alpha^2+a\alpha+a^2/4=-b+a^2/4\in K.
        \]
        Therefore $\alpha'=\alpha+a/2$
        is such that $\alpha'^2=-b+a^2/4\in K$.
        Moreover $\alpha'=\alpha+a/2\in K(\alpha)$, so $K(\alpha')\subseteq K(\alpha)$ and $\alpha=\alpha'-a/2\in K(\alpha')$ so $K(\alpha)\subseteq K(\alpha.)$. Therefore $K(\alpha)=K(\alpha')$ and $\alpha'^2\in K$.
        \item Assume that $\frac{\alpha^2}{\beta^2}$ is a square in $K$,
        i.e. there is a $k\in K$ such that
        $\frac{\alpha^2}{\beta^2}=k^2$.
        (Note that $k\neq 0$ otherwise $\alpha=0\in K$
        and $[K(\alpha):K]=[K:K]=1$.)

        Then $\alpha=k^2\beta^2$, hence $\alpha=\pm k\beta\in K(\beta)$. So $K(\alpha)\subseteq K(\beta)$.
        On the other hand, $\beta^2=\frac{\alpha^2}{k^2}$, hence $\beta=\pm \frac{\alpha}{k}\in K(\alpha)$. So $K(\beta)\subseteq K(\alpha)$.
        Having proved both inclusions we deduce that 
        $K(\alpha)=K(\beta)$.

        Vice versa, assume that 
        $K(\alpha)=K(\beta)$.
        Knowing that $[K(\alpha):K]=2$, 
        we have that $\{1,\alpha\}$ 
        is a generating set of the $K$-vector space $K(\alpha)=K(\beta)$.
        Therefore there exist $a,b\in K$ such that
        $\beta=a+b\alpha$.
        Squaring both sides of the equality, we get
        $\beta^2=a^2+2ab\alpha+\alpha^2$.
        So
        $2ab\alpha=\beta^2-a^2-\alpha^2$ is a sum of elements in $K$.
        Hence $2ab\alpha\in K$, but $\alpha\not\in K$.
        Therefore the only possibility is that $2ab=0$, i.e. (since we are not in characteristic 2)
        $ab=0$. But $b\neq 0$, otherwise $\beta=a\in K$, which is not possible (otherwise
        and $[K(\beta):K]=[K:K]=1$).
        Thus $a=0$, i.e. $\beta=b\alpha$ and so $\frac{\beta^2}{\alpha^2}=a^2$ is a square in $K$.
        \item Let $L/K$ be a field extension with $[L:K]=2$. 
        Take $\alpha\in L\setminus K$, then
        \[
        1<[K(\alpha):K]\leq [L:K]=2.
        \]
        Hence $L=K(\alpha)$.
        Moreover, the first part of this exercise allows us to choose $\alpha$ such that $\alpha^2\in K$.
        Now we can define the following map
        \[\psi:\{L\mid L/K \text{ is a field extension with } [L:K] \leq 2\}\to K^\times/(K^\times)^2\]
        as $\psi(K)=[1]$ and for
        $[L:K]=2$ as $\psi(L)=[\alpha^2]\in K^\times/(K^\times)^2$, for $\alpha\in L\setminus K$ (so $L=K(\alpha)$) such that $\alpha^2\in K$.     
        
        This map is well-defined:
        take $L/K$ of degree 2 and
        $\alpha,\beta\in L\setminus K$ such that $\alpha^2\in K$.
        Then, by the remark made at the beginning, $L=K(\alpha)=K(\beta)$ 
        As shown in the second part of this exercise, this is equivalent to $\frac{\alpha^2}{\beta^2}\in (K^\times)^2$, so $[\alpha^2]=[\beta^2]\in K^\times/(K^\times)^2$.
        (Note also that $[\alpha^2]=[1]$  if and only if $\alpha^2 =k^2\in K^2$,
        so $\alpha=\pm k\in K$ and $K(\alpha)=L$.)

        Let now prove that $\psi$ is injective.
        Assume that we have 
        two extension $L/K$ and 
        $L'/K$ of degree $\leq 2$, such that
        $[\alpha^2]=\psi(L)=\psi(L')=[\beta^2]$.
        
        If $[\beta^2]=[\alpha^2]=[1]$,
        then $L=K(\alpha)=K=K(\beta)=L'$.
        
        Otherwise, $\beta^2/\alpha^2\in (K^\times)^2$,
        so, by the previous part of this exercise,
        $L=K(\alpha)=K(\beta)=L'$.     

        Finally, for the subjectivity, let $x\in K^\times$.
        If $x\in (K^\times)^2$, then $x=\alpha^2$ for some $\alpha\in K^\times$
        and so $L=K(\alpha)$ is an extension of $K$ of degree $\leq 2$ 
        and $\psi(L)=[\alpha]$.

        If $x\not\in (K^\times)^2$, then we can find and extension $L=K(\alpha)$ such that
        $\alpha^2=x\in K^\times$ and so $\psi(L)=[x]\in K^\times/(K^\times)^2$.
        To construct this extension consider the polynomial $f(X)=X^2-x\in K[X]$.
        It has to be irreducible, otherwise $X^2-x=(aX+b)(cX+d)=acX^2+(ad+bc)X+bd$, for some
        $a,c\in K^\times$ and $b,d\in K$.
        So $1=ac$, $0=ad+bc$ and $-x=bd$, i.e. $c=a^{-1}$, $d=-a^{-1}bc=bc^2$ and $-x=bd=b^2c^2\in (K^\times)^2$, a contradiction to the assumption $x\not\in (K^\times)^2$.
        Then we can consider the field $L=K[X]/(X^2-x)$ that contains $K$
        (it is a field because the ideal $(X^2-x)$ is maximal,
        since it is generated by an irreducible polynomial).
        Defining $\alpha$ as the class of $X$ in $L$ 
        we get that $L=K(\alpha)$ and $\alpha^2=x\in K$.        
    \end{enumerate}    
\end{sol}

\begin{ex}
    Let $E/K$ be a field extension and 
    $a$ and $b$ be algebraic over $K$.

    \begin{enumerate}
        \item Assume that $[K(a):K] = m$, $[K(b):K] = n$. Prove that $K[a,b] \subseteq E$ is generated, as a vector space over $K$, by the elements $a^ib^j$ ($1 \leq i \leq m$, $1 \leq j \leq n$).
        \item Prove that $a+b$ and $ab$ are algebraic over $K$. Can you estimate the quantities $[K(a+b):K]$ and $[K(ab):K]$?
        \item Find a polynomial $f \in \Q[X]$ such that $\deg(f) \leq 6$ and $f(\sqrt[3]{3} + \sqrt{5})$.
    \end{enumerate}
\end{ex}

The following lemma can be used, without proof, in the following exercise.

\begin{lem*}[Gauss' Lemma]
    Let $A$ be a unique factorization domain and $K$ be its fraction field.
    A non-constant polynomial $f\in A[X]$ is irreducible if and only if is primitive and is irreducible in $K[X]$.
\end{lem*}

\begin{ex} [Eisenstein's irreducibility criterion]
    Let $A$ be a unique factorization domain and $K$ be its fraction field.
    Let $f=\sum_{i=0}^n a_iX^i\in K[X]$ be a polynomial of degree $n>0$. 
    Assume that there exists a prime element $p\in A$ such that
    $p\mid a_i$ for all $i\in\{0,1,\dots,n-1\}$, $p\nmid a_n$ and
    $p^2\nmid a_0$. Then $f$ is irreducible in $K[X]$. 
\end{ex}


\begin{ex}
Let $\zeta\in\C$ be a primitive cubic root of one.
Set $E=\Q[\sqrt[3]{2}]$, $F=\Q(\zeta)$ and $L=\Q[i]$.
\begin{enumerate}[label=(\roman*)]
    \item Prove that $[E:\Q]=3$ and $[F:\Q]=2$ and compute the minimal polynomial of $\zeta$ over $\Q$ and over $L$.
    \item Prove that $EF=\Q(\sqrt[3]{2},\zeta)$.
    \item Compute $[EF:\Q]$ and $[E\cap F:\Q]$.
\end{enumerate}  
\end{ex}

\begin{ex}
    Let $E/K$ be a field extension and let $L/K$ and $M/K$ be subextensions.
    \begin{enumerate}[label=(\roman*)]
    \item Prove that $[LM : K ] \cdot [L \cap M : K] \leq [L:K] \cdot [M:K]$.
    \item Can you find examples where $[LM : K ] \cdot [L \cap M : K] < [L:K] \cdot [M:K]$?
    
    \noindent \textit{Hint:} Use two different roots of the polynomial $X^3 -2$.
    \end{enumerate}  
\end{ex}

\begin{ex}
    Let $E/K$ be a field extension and let $f \in K[X]$ be a polynomial such that $f$ factorizes in $E[X]$ as $f = \prod_{i=1}^n(x - \alpha_i)$. Prove by induction that $[K(\alpha_1,\alpha_2, \ldots, \alpha_n):K] \leq n!$.
\end{ex}

\newpage

\section{Week 3}


\begin{ex}
    
    For every polynomial $p(X)=\sum_{i=0}^n a_iX^i\in K[X]$ of degree $n$,
    define its \emph{reciprocal polynomial} as
    \[
    \widehat{p}(X)=\sum_{i=0}^n a_{n-i}X^i.
    \]
    Let $p(X),q(X)\in K[X]$ be polynomials of degree $n$ and $m$ respectively such that $p(0)\neq 0$ and $q(0)\neq 0$. Prove that
    \begin{enumerate}[label=(\roman*)]
    \item $\widehat{p}(X)=X^np(1/X)$ in $K(X)$,
    \item $\widehat{ \widehat{p}}(X)=p(X)$,
    \item $\widehat{pq}(X)=\widehat{p}(X)\widehat{q}(X)$,
    \item $\widehat{p}(X)$ is irreducible if and only if $p(X)$ is irreducible.
    \end{enumerate}
\end{ex}

\begin{sol}
Suppose that $p(X)=\sum_{i=0}^n a_iX^i$ has degree $n$ and $q(X)=\sum_{i=0}^m b_iX^i$ has degree $m$.
    \begin{enumerate}[label=(\roman*)]
    \item 
    $X^np(1/X)=X^n\sum_{i=0}^n a_i(1/X)^i=\sum_{i=0}^n a_iX^{n-i}=\sum_{j=0}^n a_{n-j}X^j=\widehat{p}(X)$
    \item Since $p(0)\neq 0$, $a_0\neq 0$, therefore $\deg(\widehat{p})=\deg(p)=n$. Hence, using the previous part of this exercise,
    \[
    \widehat{\widehat{p}}(X)=X^n\widehat{p}(1/X)=X^n(1/X)^np(1/(1/X))=p(X).
    \]    
    \item Since $p(0)\neq 0$ and $q(0)\neq 0$ we deduce that also $pq(0)\neq 0$.
    Hence we can apply the previous results for $p,q$ and $pq$. 
    So, using also that $pq(1/X)=p(1/X)q(1/X)$ and that $\deg{pq}=n+m$, we have that
    \[
    \widehat{pq}(X)=X^{n+m}pq(1/X)=X^np(1/X)X^mq(1/X)=\widehat{p}(X)\widehat{q}(X).
    \]
    \item Suppose $\widehat{p}(X)$ is irreducible and let $p=q_1q_2$ for
    $q_1(X),q_2(X)\in K[X]$.
    Note that $0\neq p(0)=q_1(0)q_2(0)$ implies that both $q_1(0)\neq 0$ and $q_2(0)\neq 0$.
    Considering now the reciprocal polynomials and using the previous properties,
    $\widehat{p}(X)=\widehat{q_1}(X)\widehat{q_2}(X)$.
    Since $\widehat{p}(X)$ is irreducible, there is $i\in\{1,2\}$ such that $\widehat{q_i}$ 
    is a constant, i.e. $\deg{q_i}=\deg{\widehat{q_i}}=0$.
    Thus, $q_i$ is a constant too, and therefore $p(X)$ is irreducible.

    Vice versa, assume that $p(X)$ is irreducible. Since we know that $p(X)=\widehat{\widehat{p}}$, by the property just proved, we obtain that $\widehat{p}$ 
    is also irreducible.
    \end{enumerate}
\end{sol}



\begin{ex}
    Let $E/K$ be a field extension and $x\in E$ be an algebraic element and let $f=f(x,K)$ is the minimal polynomial of $x$ over $K$ of degree $\deg(f)=n$.
    \begin{enumerate}[label=(\roman*)]
    \item Prove that $[K(x):K]=n$.
    \item Prove that $\frac{1}{f(0)}\widehat{f}$ is the minimal polynomial of $1/x$ over $K$.
    \item Write $f(X)=\sum_{i=0}^na_iX^i=p(X^2)+Xd(X^2)$, where 
    \[
    p(X)=\sum_{j=0}^{\lfloor n/2\rfloor } a_{2j}X^{j}\text { and }
    d(X)=\sum_{j=0}^{\lfloor n/2\rfloor } a_{2j+1}X^j.
    \]
    Let $g(X)=p(X)^2-Xd(X)^2$ and prove that
    \begin{itemize}
        \item if $d(x^2)=0$, then the minimal polynomial of $x^2$ over $K$ is $p(X)$,
        \item if $d(x^2)\neq 0$, then the minimal polynomial of $x^2$ over $K$ is $(-1)^ng(X)$.
    \end{itemize}
    \noindent
    \end{enumerate}
\end{ex}
\begin{sol}~
    \begin{enumerate}[label=(\roman*)]
    \item We will prove that $B=\{1,x,\dots, x^{n-1}\}$ is a basis of $K(x)$ as a $K$ vector space which implies that $[K(x):K]=\dim_K(K(x))=|B|=n$.
    \begin{itemize}
        \item[-] \textit{$B$ is a generating set for $K(x)$ as a $K$ vector space.}
        
        To prove this recall that $K(x)=K[x]$, since $x$ is algebraic over $K$.
        
        (Similar proof to $1)\Rightarrow 2)$ of Theorem 2.7 of the lecture notes.)

        Let $z\in K(x)=K[x]$, say $z=h(x)$ for some $h\in K[X]$. 
        Divide $h$ by $f$ to obtain polynomials $q,r\in K[X]$ 
        such that $h=fq+r$, where $r=0$ or $\deg r<\deg f=n$. This implies that
	\[
		z=h(x)=f(x)q(x)+r(x)=r(x).
	\]
	Moreover we can write
        $r=\sum_{i=0}^{n-1}c_iX^i$ for some $c_0,\dots,c_{n-1}\in K$. 
        Thus $z=\sum_{i=0}^{n-1}c_ix^i\in \langle 1,x,\dots,x^{n-1}\rangle$
        and hence $K[x]$ is generated by $\{1,x,\dots,x^{n-1}\}$ as a $K$-vector space.
        \item[-] \textit{$B$ is linearly independent over $K$.}

        If $B$ is linearly dependent over $K$ then there exists a linear combination 
        $0=\sum_{i=0}^{n-1}c_ix^i$ over $K$, with not all $c_i$ equal to 0.
        Then the polynomial $h(X)=\sum_{i=0}^{n-1}c_iX^i$ is in $K[X]\setminus\{0\}$
        and has $x$ as a root.
        So 
        \[
        n-1=\deg(h)\leq \deg(f)=n,
        \]
     a contradiction.    
    \end{itemize}
    \item Fist of all we note that $f(0)\neq 0$. Otherwise we can write $f(X)=Xg(X)$ for some $g(X)\in K[X]$, but $f$ is monic and irreducible in $K[X]$, hence $g(X)=1$ and $f(X)=X$.
    Evaluating $f$ in $x$ we obtain $0=f(x)=x$, a contradiction with the hypothesis $x\neq 0$.

    Since $f(0)\neq 0$, we can use the previous exercise and obtain that $\widehat{f}$ is also irreducible and $\widehat{f}=X^nf(1/X)$.
    Hence
    \[
    \frac{1}{f(0)}\widehat{f}\left(1/x\right)=\frac{1}{f(0)}\left(1/x\right)^nf\left(\frac{1}{1/x}\right)=\frac{1}{f(0)x^n}f(x)=0.
    \]
    So we have that $\frac{1}{f(0)}\widehat{f}$ is an irreducible polynomial in $K[x]$ 
    with $1/x$ as a root.
    To prove that $\frac{1}{f(0)}\widehat{f}$ is 
    the minimal polynomial of $1/x$ over $K$ it remains to prove that it is monic.
    Looking at the definition of $\widehat{f}$ we see that its leading coefficient
    is the constant term of $f$, i.e. $f(0)$.
    Therefore the leading coefficient of $\frac{1}{f(0)}\widehat{f}$ is $\frac{1}{f(0)}f(0)=1$, hence $\frac{1}{f(0)}\widehat{f}$ is monic.
    \item 
    First of all note that 
    \begin{equation}
           0=f(x)=p(x^2)+xd(x^2), \label{eq: f(x)=0}
    \end{equation} therefore
    \[
    g(x^2)=p(x^2)^2-x^2d(x^2)^2=(p(x^2)+xd(x^2))(p(x^2)-xd(x^2))=0.
    \]
    So $x^2$ is a root of $g$ and the degree of $g$ is 
    \[
   \begin{cases}
        2\deg(p) &\text{ if }n\text{ is even}\\
        1+2\deg(d)&\text{ if }n\text{ is odd}
    \end{cases} =\begin{cases}
        2\left\lfloor \frac{n}{2}\right\rfloor &\text{ if }n\text{ is even}\\
        1+2\left\lfloor \frac{n}{2}\right\rfloor&\text{ if }n\text{ is odd}
    \end{cases} = n.
    \]
    Moreover, by the first point of this exercise and the fact that the degree is multiplicative
    \[
    n=\deg(f)=[K(x):K]=[K(x):K(x^2)][K(x^2):K].
    \]
    Hence, to compute $[K(x^2):K]$ (which is also the degree of $f(x^2,K)$,
    the minimal polynomial of $x^2$ over $K$), we need to know $[K(x):K(x^2)]$.
    Observe that $X^2-x^2$ is a polynomial in $K(x^2)[X]$ which has $x$ has a root. Thus 
    \[
    [K(x^2):K]=\deg(x,K(x^2))\leq \deg(X^2-x^2)=2
    \]
    Therefore 
    \begin{equation}\label{eq: deg of x^2}
        [K(x^2):K]=\frac{[K(x):K]}{[K(x):K(x^2)]}\in\left\{n,\frac{n}{2}\right\}.
    \end{equation}
    \begin{itemize}
        \item If $d(x^2)=0$, by Equation \eqref{eq: f(x)=0}, also $p(x^2)=0$ and $\deg(p)=\lfloor \frac{n}{2}\rfloor< n$.
        So 
        \[
        [K(x^2):K]=\deg(f(x^2,K))\leq \deg(p)< n,
        \]
        thus, by \eqref{eq: deg of x^2}, $[K(x^2):K]=\frac{n}{2}$, which implies that $n$ has to be even, $p(X)$ monic and 
        \[
        [K(x^2):K]=\frac{n}{2}=\left\lfloor \frac{n}{2}\right\rfloor=\deg(p).
        \]
        Therefore $p(X)$ is a monic polynomial in $K[X]$
        which as $x^2$ as a root and of degree $[K(x^2):K]=\deg(f(x^2,K))$,
        hence it is $\deg(f(x^2,K))$,
        the minimal polynomial of $x^2$ over $K$.
        \item If $d(x^2)\neq0$, then, by Equation \eqref{eq: f(x)=0}, 
        \[
        x=-\frac{p(x^2)}{d(x^2)}\in K(x^2).
        \]
        Therefore $K(x)\subseteq K(x^2)\subseteq K(x)$ and so $K(x^2)=K(x)$,
        which means $[K(x):K(x^2)]=1$ and, by \eqref{eq: deg of x^2} ,
        \[
        [K(x^2):K]=\frac{[K(x):K]}{[K(x):K(x^2)]}=[K(x):K]=n=\deg(g).
        \]
        Moreover the leading coefficient of $g(X)$ is
        \[
        \begin{cases}
        a_n &\text{ if }n\text{ is even}\\
        -a_n&\text{ if }n\text{ is odd}
        \end{cases}=(-1)^na_n=(-1)^n .
        \]
        Therefore we have the monic polynomial  
        $(-1)^ng(X)\in K[X]$ that vanishes in $x^2$,
        of degree $n=[K(x^2):K]=\deg(f(x^2,K))$. Hence $(-1)^ng(X)=f(x^2,K)$.
    \end{itemize}
    \end{enumerate}
\end{sol}



The following lemma can be used, without proof, in the following exercise.

\begin{lem*}[Gauss' Lemma]
    Let $A$ be a unique factorization domain and $K$ be its fraction field.
    A non-constant polynomial $f\in A[X]$ is irreducible if and only if it is primitive and irreducible in $K[X]$.
\end{lem*}

\begin{ex} [Eisenstein's irreducibility criterion]
    Let $f=\sum_{i=0}^n a_iX^i\in \Z[X]$ be a polynomial of degree $n>0$. 
    Assume that there exists a prime $p$ such that
    $p\mid a_i$ for all $i\in\{0,1,\dots,n-1\}$, $p\nmid a_n$ and
    $p^2\nmid a_0$. Then $f$ is irreducible in $\Q[X]$.

    \vspace{.5cm}
    
    \textit{More general version}
    
    Let $A$ be a unique factorization domain and $K$ be its fraction field.
    Let $f=\sum_{i=0}^n a_iX^i\in A[X]$ be a polynomial of degree $n>0$. 
    Assume that there exists a prime element $p\in A$ such that
    $p\mid a_i$ for all $i\in\{0,1,\dots,n-1\}$, $p\nmid a_n$ and
    $p^2\nmid a_0$. Then $f$ is irreducible in $K[X]$. 
\end{ex}




\begin{ex}
Let $\zeta\in\C$ be a primitive cubic root of one.
Set $E=\Q[\sqrt[3]{2}]$, $F=\Q(\zeta)$ and $L=\Q[i]$.
\begin{enumerate}[label=(\roman*)]
    \item Prove that $[E:\Q]=3$ and $[F:\Q]=2$ and compute the minimal polynomial of $\zeta$ over $\Q$ and over $L$.
    \item Prove that $EF=\Q(\sqrt[3]{2},\zeta)$.
    \item Compute $[EF:\Q]$ and $[E\cap F:\Q]$.
\end{enumerate}  
\end{ex}

\begin{ex}
    Let $E/K$ be a field extension and let $L/K$ and $M/K$ be subextensions.
    \begin{enumerate}[label=(\roman*)]
    \item Prove that $[LM : K ] \cdot [L \cap M : K] \leq [L:K] \cdot [M:K]$.
    \item Can you find examples where $[LM : K ] \cdot [L \cap M : K] < [L:K] \cdot [M:K]$?
    
    \noindent \textit{Hint:} Use two different roots of the polynomial $X^3 -2$.
    \end{enumerate}  
\end{ex}

\begin{ex}
    Let $E/K$ be a field extension and let $f \in K[X]$ be a polynomial such that $f$ factorizes in $E[X]$ as $f = \prod_{i=1}^n(x - \alpha_i)$. Prove by induction that $[K(\alpha_1,\alpha_2, \ldots, \alpha_n):K] \leq n!$.
\end{ex}

\newpage
\section{Week 4}


\begin{ex}
    Let $K$ be a field. For a polynomial $f = \sum_{k=0}^na_kX^k \in K[X]$, we define the \emph{derivative} by
    \[
    f' = \sum_{k=1}^n ka_kX^{k-1}.
    \]
    \begin{enumerate}[label=(\roman*)]
        \item Let $\alpha \in K$ and $f,g \in K[X]$. Prove the following properties of the derivative
        \begin{enumerate}
            \item $(f + g)' = f' + g'$,
            \item $(\alpha \cdot f)' = \alpha \cdot f'$,
            \item $(f\cdot g)' = f'\cdot g + f \cdot g'$.
        \end{enumerate}
        \item Let $f \in K[X]$ be a polynomial that factorizes as $f = \prod_{i=1}^n (X - \alpha_i)$. Prove that the roots $\alpha_1, \ldots, \alpha_n$ are pairwise different if and only if $\gcd(f,f') = 1$.
        \item Let $C$ be an algebraic closure of $K$ and let $f \in K[X]$ be a polynomial with $\deg(f) \geq 1$ that is irreducible over $K$. Prove that $f$ has repeated roots in $C$ if and only if $f' = 0$.  In particular, show that 
        having such a polynomial $f$
        implies that $\mathrm{char}(K) = p > 0$ and $f(X) = g(X^p)$ for some irreducible polynomial $g \in K[X]$.
    \end{enumerate}
\end{ex}

\begin{sol}~
 \begin{enumerate}[label=(\roman*)]
        \item For sake of simplicity, write $f = \sum_{k=0}^{\infty} a_k X^k$ and $g = \sum_{k=0}^{\infty} b_kX^k$. Then
        %\begin{enumerate}
            %\item 
            \begin{align*}
            (f + g)' & 
            =  \left(\sum_{k=0}^{\infty} (a_k + b_k) X^k \right)' 
            = \sum_{k=1}^{\infty} k (a_k+b_k)X^{k-1} \\
            & = \sum_{k=1}^{\infty} ka_kX^{k-1} + \sum_{k=1}^{\infty} kb_kX^{k-1} = f' + g'.
        \end{align*}
        Hence we have proved (a).
        We can now prove also (b) as:
        %\item
        \[
        (\alpha \cdot f)'
        = \left(\sum_{k=0}^{\infty} \alpha a_k X^k \right)
        = \sum_{k=1}^{\infty} k \alpha a_kX^{k-1} 
        = \alpha \cdot \sum_{k=1}^{\infty} k a_kX^{k-1}
        = \alpha \cdot f'.
        \]
        % \begin{align*}
        %     (\alpha \cdot f)' & = \left(\sum_{k=0}^{\infty} \alpha a_k X^k \right) \\
        %     & = \sum_{k=1}^{\infty} k \alpha a_kX^{k-1} \\
        %     & = \alpha \cdot \sum_{k=1}^{\infty} k a_kX^{k-1} \\
        %     & = \alpha \cdot f'.
        % \end{align*}
        %\item 
        To prove (c), we first check the equality for $f = X^k$ and  $g = X^l$:
        \[
        (f \cdot g)' = (X^{k+l})' = (k+l)X^{k+l-1} = kX^{k-1}X^l + X^klX^{l-1} = f' \cdot g + f \cdot g'.
        \]
        Using the already-established $K$-linearity (i.e. (a) and (b) of this exercise), we can now calculate
        \begin{align*}
            (f \cdot g)' & 
            = \left( \sum_{k,l \geq 0} a_kb_l X^{k+l} \right)' 
            = \sum_{k,l \geq 0} a_kb_l (X^{k+l})'\\
            &= \sum_{k,l \geq 0} a_kb_l ((X^k)'X^l + X^k(X^l)') \\
            & = \sum_{k,l \geq 0} a_kb_l (X^k)'X^l  + \sum_{k,l \geq 0} a_kb_l X^k(X^l)' \\
            & = \left( \sum_{k=0}^{\infty} a_kX^k \right) \cdot \left( \sum_{l=0}^{\infty} b_l(X^l)' \right) + \left( \sum_{k=0}^{\infty} a_k(X^k)' \right) \cdot \left( \sum_{l=0}^{\infty} b_lX^l \right) \\
            & = f' \cdot g + f \cdot g'.
        \end{align*}
        %\end{enumerate}

        \item Let $\alpha_i$ be one of the roots of $f$. Write $f = (X - \alpha_i) \cdot g$.
        By the (i) of this exercise,
        \[
        f' = (X - \alpha_i)' \cdot g + (X - \alpha_i) \cdot g' = g + (X - \alpha_i) \cdot g'.
        \]
        Therefore, $f'(\alpha_i) = g(\alpha_i) = \prod_{j \neq i} (\alpha_i - \alpha_j)$. This implies that $f'(\alpha_i) = 0$ if and only if $\alpha_i$ is a repeated root of $f$ which is the case if and only if $(X - \alpha_i)$ is a common divisor of $f$ and $f'$.

        As the linear factors $(X-\alpha_i)$ are prime elements of $K[X]$, it follows that $f$ and $f'$ have common divisors if and only if $f$ has repeated roots.

        \item Let $f$ be irreducible in $K[X]$ with repeated roots in $C[X]$. By the previous exercise, $\gcd(f,f') \neq 1$. As $f$ is irreducible and $\gcd(f,f') \in K[X]$, this implies $\gcd(f,f') = f$. Therefore, $f$ divides $f'$. As $\deg(f') < \deg(f)$, this implies $f' = 0$.

        In case that $f' = 0$, we write $f = \sum_{k=0}^{\infty}a_kX^k$ and consider that
        \[
        f' = \sum_{k=1}^{\infty} ka_kX^{k-1} = 0.
        \]
        Therefore, for all $k \geq 0$, we have $k = 0$ or $a_k = 0$. As there is at least one $k \geq 1$ with $a_k$ we conclude that $k = 0$ holds in $K$ for some nonzero $k$. This implies that  $\mathrm{char}(K) = p > 0$ and that $a_k = 0$ whenever $p \nmid k$. We can therefore write
        \[
        f = \sum_{l = 0}^{\infty} a_{pl}X^{pl} = g(X^p)
        \]
        with $g = \sum_{l=0}^{\infty} a_{pl}X^l$. For a decomposition $g = g_1g_2$, we  also get a decomposition $f(X) = g(X^p) = g_1(X^p)g_2(X^p)$ which implies that either $g_1(X^p)$ or $g_2(X^p)$ is in $K$, which amounts to saying that $g_1$ or $g_2$ is in $K$. Therefore, $g$ has to be irreducible.
    \end{enumerate}
    
\end{sol}

\begin{ex}
    Let $L$ be a finite field.
    \begin{enumerate}[label=(\roman*)]
    \item Show that $L$ is not algebraically closed.
    
    \noindent \textit{Hint:} Consider the polynomial $f(X)=1+\prod_{l\in L} (X-l)\in L[X]$.
    \item Show that $L$ contains a 
    subfield $K$ isomorphic to $\Z/p$
    (its ring of integers) and that $|L|=p^m$, where $m=[L:K]$.
    \end{enumerate}
    Assume now that $K=\Z/p$ and let $f(X)=X^{p^m}-X\in K[X]$.
    Let $C$ be an algebraic closure of $K$ and set 
    $L=\{\alpha\in C\mid f(\alpha)=0\}$.
    Prove that
    \begin{enumerate}
        \item[(iii)] $|L|=p^m$

        \noindent\textit{Hint:} Use the previous exercise.
        \item[(iv)] Recall that, since $K$ has characteristic $p$, $\Phi: K \to K$; $x \mapsto x^p$ is a field endomorphism (the Frobenius endomorphism).
        Prove that $L$ is a field and $K\subseteq L\subseteq C$.
    \end{enumerate}
\end{ex}
% \begin{sol}~
% \begin{enumerate}[label=(\roman*)]
%     \item The polynomial $f(X)\in L[X]$ doesn't have roots in $L$.
%     In fact consider any $a\in L$, then 
%     \[
%     f(a)=1+\prod_{l\in L} (a-l)=1+(a-a)\prod_{l\in L\setminus \{a\}} (X-l)=1+0=1\neq 0.
%     \]
%     \item Let $K$ be the ring of integers of $L$. 
%     Since $L$ is finite, $K$ is finite too. 
%     Hence $K$ is isomorphic to $\Z/p$ for some prime $p$.
%     $L$ is a vector space of dimension $m$ over $K\cong\Z_p$.
%     Let $\{x_1,x_2,\dots,x_m\}$ is a 
%     basis of $L$ over $K$,
%     then every element of $L$ can be 
%     written in a unique way as a linear 
%     combination $\sum_{i=1}^m a_ix_i$, 
%     with $a_i\in K$.
%     So the number of elements of $L$ is
%     equal to the number of tuples
%     $(a_1,\dots,a_m)\in K^m$.
%     Hence $|L|=|K|^m=p^m$.
%     \item If $f=X^{p^m}-X$, then $f'=p^mX^{p^m-1}-1=-1$. 
%     So, $ \gcd(f,f')=1$ and, by the previous exercise, we can conclude 
%     that $f$ has $p^m$ pairwise different roots in $C$, i.e. $|L|=p^m$.
%     \item Using the Frobenius endomorphism $\Phi $ of $K$, we can see that
%     \[
%     L=\{\alpha\in C\mid \alpha^{p^m}=\alpha\}=\{\alpha\in C\mid \Phi^m(\alpha)=\alpha\}.
%     \]
%     Since $\Phi$ is an endomorphism, $\Phi^m$ is also an endomorphism.
%     Then for all $\alpha,\beta\in L$,
%     \[
%     \Phi^m(\alpha+\beta)=\Phi^m(\alpha)+\Phi^m(\beta)=\alpha+\beta
%     \]
%     \[
%     \Phi^m(-\alpha)=-\Phi(\alpha)=-\alpha
%     \]
%     \[
%     \Phi^m(\alpha\beta)=\Phi^m(\alpha)\Phi^m(\beta)=\alpha\beta.
%     \]
%     \[
%     \Phi^m(\alpha^{-1})=\big(\Phi^m(\alpha)\big)^{-1}=\alpha^{-1}.
%     \]
%     So $\alpha+\beta,\-\alpha,\alpha\beta,\alpha^{-1}\in L$, i.e.
%     $L$ is a field.
% \end{enumerate}
% \end{sol}

\begin{ex}
Let $C/K$ be an algebraic field extension.
Show that the following are equivalent:
\begin{enumerate}[label=(\roman*)]
    \item $C$ is an algebraic closure of $K$.
    \item For every algebraic extension $L/K$ there is an extension homomorphism $\varphi\in \Hom(L/K,C/K)$.
\end{enumerate}  
\noindent \textit{Hint:} For (i)$\Rightarrow$(ii) use Proposition 3.6 in the notes.
\end{ex}





\begin{ex}
    Let $f\in K[X]$ be a polynomial of degree $n$.
    Let $C$ be an algebraic closure of $K$ and $A=\{\alpha_1,\dots\alpha_k\}\subseteq C$ be the distinct roots of $f$ in $C.$
    We know that $E=K(\alpha_1,\dots,\alpha_k)$ is the decomposition field of $f$ over $K$.
    \begin{enumerate}[label=(\roman*)]
    \item Prove that $[E:K]\leq n! $.
    \item Prove that there is an injective homomorphism $\Gal(E/K)\longrightarrow \Sym_{A}\cong \Sym_k$ 
    
    \noindent\textit{Hint: } Prove that $\sigma(A)=A$ for every $\sigma \in \Gal(E/K)$.
    \item For $K=\Z/3$ and $f=X^3-X-1$, compute $[E:K]$ and $\Gal(E/K)$.
    (Observe that in this case $[E:K]<n!$.)
    \end{enumerate}
\end{ex}
% \begin{sol}
% \begin{enumerate}[label=(\roman*)]
%     \item If $n = 0$, the polynomial $f$ is a nonzero constant. Therefore, $A = \emptyset$ and $E = K$. In this case, $[E:K] = [K:K] = 1 = 0!$.

%     Suppose that we have proven that $[F:L] \leq n!$ whenever $F$ is the decomposition field of a polynomial $g \in L[X]$ with $\deg(g) = n$.
    
%     We assume now that $f \in K[X]$ has $\deg(f) = n+1$. Denote the decomposition field of $f$ over $K$ by $E$. Let $\alpha$ be a root of $f$.

%     As $f(\alpha) = 0$ we know that $f(\alpha,K)| f$. Therefore,
%     \[
%     [K(\alpha):K] = \deg(f(\alpha,K)) \leq \deg(f) = n+1.
%     \]
%     As $\alpha \in K(\alpha)$, we conclude that $g = \frac{f}{X-\alpha} \in K(\alpha)[X]$. Furthermore, $E$ is the decomposition field of $g$ over $K$: if $A$ is the set of roots of $g$, then $A \cup \{\alpha \}$ is the set of roots of $f$. Therefore, $E = K(A \cup \{ \alpha \}) = K(\alpha)(A)$.
    
%     As $\deg(g) = n$, we can apply the inductive hypothesis and infer that
%     \[
%     [E:K] = [E:K(\alpha)] \cdot [K(\alpha):K] \leq (n+1) \cdot n! = (n+1)!.
%     \]
%     \item Let $\alpha \in A$ and $\sigma \in \Gal(E/K)$, then
%     \[
%     f(\sigma(\alpha)) = \overline{\sigma}(f) (\sigma(\alpha)) = \sigma(f(\alpha)) = \sigma(0) = 0.
%     \]
%     Therefore, $\sigma(\alpha) \in A$. As a consequence, $\sigma(A) = A$ for all $\sigma \in \Gal(E/K)$.
%     Therefore the restriction
%     \begin{align*}
%     \gamma: \Gal(E/K) & \to \Sym_A \\
%     \sigma & \mapsto \sigma |_A
%     \end{align*}
%     is well-defined and, as the restriction of a group action, indeed a homomorphism. As $E$ is generated by $A$ over $K$, an automorphism $\sigma \in \Gal(E/K)$ is uniquely determined by its action on $A$. We conclude that $\gamma$ is injective.
%     \item Let $\alpha$ be a root of $f$. As $f = X(X-1)(X+1) -1$, the fact that $\mathrm{char}(K) = 3$ implies that $\alpha + k$ is a root of $f$ for any $k \in \Z/3$. Looking at the degree, this implies that these are in fact all roots of $f$.

%     Note that $f(k) = -1$ for all $k \in K$ which implies that $f$ has no roots in $K$. A reducible polynomial of degree $3$ always has roots, therefore $f$ has to be irreducible.

%     It follows that $f = f(\alpha,K)$ and $[K(\alpha):K] = 3$. As all roots $\alpha + k \in K(\alpha)$ for $k \in K$, we conclude that $E = K(\alpha)$. Therefore $[E:K] = 3$.

%     By the same argument as in the proof of Theorem 4.10, there is for each $k \in K$ a unique $\phi \in \Hom(E/K,E/K)$ with $\phi(\alpha) = \phi(\alpha+k)$. This shows that $\Gal(E/K) \cong \Z/3$.
%     \end{enumerate}
    
% \end{sol}



\begin{ex}
    Let $f=X^4-5X^2+5\in \Q[X]$ and $E$ be a decomposition field of $f$ over $\Q$. 
    Prove that $[E:\Q]=4$. 

    \textit{Hint:}
    Given $\alpha,\beta\in \C$
    two solutions of $f$ such that $\beta\neq-\alpha$,
    compute $\alpha\beta$.
    Prove also that $E=\Q(\alpha)$.
\end{ex}
% \begin{sol}
% Note that, since $f$ is an even polynomial
% if $\alpha\in \C$ is a root of $f$,
% then also $-\alpha$ is a root of $f$.
% Hence, given two roots $\alpha,\beta\in \C$
% such that $\beta\neq-\alpha$,
% we have that $E=\Q(\alpha,-\alpha,\beta,-\beta)$.
% But $-\alpha,-\beta\in \Q(\alpha,\beta)\subseteq E$
% and so 
% \[
% E=\Q(\alpha,-\alpha,\beta,-\beta)\subseteq \Q(\alpha,\beta)\subseteq\Q(\alpha,-\alpha,\beta,-\beta)=E,
% \]
% which means that $E=\Q(\alpha,\beta)$.
% Moreover we can decompose $f$ in $\C[X]$ as 
% \[
% (X-\alpha)(X+\alpha)(X-\beta)(X+\beta)=(X^2-\alpha^2)(X^2-\beta^2)=X^4-(\alpha^2+\beta^2)X^2+\alpha^2\beta^2.
% \]
% This implies in particular that $\alpha^2\beta^2=5$, hence $\beta=\pm \frac{\sqrt{5}}{\alpha}\in \Q(\alpha,\sqrt{5})$.
% Therefore $E=\Q(\alpha,\beta)\subseteq \Q(\alpha,\sqrt{5})$.
% On the other hand $\sqrt{5}=\pm\alpha\beta\in\Q(\alpha,\beta)$,
% hence $\Q(\alpha,\sqrt{5})\subseteq\Q(\alpha,\beta)=E$.
% So we can conclude that $E=\Q(\alpha,\sqrt{5})$.
% Using the multiplicative of the degree of finite extension we get that
% \[
% [E:\Q]=[E:\Q(\alpha)][\Q(\alpha):\Q].
% \]
% But $[\Q(\alpha):\Q]$ is equal to the degree of the minimal
% polynomial of $\alpha$ over $\Q$.
% Using Eisenstein criterion with $p=5$, we have that $f$ is irreducible (and monic), so it is the minimal polynomial of $\alpha$ over $\Q$.
% Thus $[\Q(\alpha):\Q]=\deg f=4$.
% It remains to compute $[E:\Q(\alpha)]=[\Q(\alpha,\sqrt{5}):\Q(\alpha)]$.
% We have the following situation:
% \begin{multicols}{2}
% \adjustbox{scale=.98,center}{%
% \begin{tikzcd}
% 	& {E=\Q(\alpha,\sqrt{5})}\arrow[rd,no head]\\
% 	{\Q(\alpha)}\arrow[ru,no head,"\leq 2"]&& {\Q(\sqrt{5})}\arrow[ld,no head,"2"]  \\
% 	& {\Q}\arrow[lu,no head,"4"] 
% \end{tikzcd}
% }
% \vfill\null
% \columnbreak
% Observe that $\Q(\alpha,\sqrt{5})$ is equal to the composite of $\Q(\alpha)$ and $\Q(\sqrt{5})$.
% We can use the property of composite extension,
% $[LF:L]\leq[F:K]$, to deduce that 
% \[
% [\Q(\alpha,\sqrt{5}):\Q(\alpha)]\leq [\Q(\sqrt{5}):\Q]=2.
% \]
% The last equality is because $X^2-5$ is 
% the minimal polynomial of $\sqrt{5}$ over $\Q$,
% as it is monic has $\sqrt{5}$ as a root 
% and it's irreducible due to Eisenstein's criterion.
% \end{multicols}
% Finally, we want to understand whether $[\Q(\alpha,\sqrt{5}):\Q(\alpha)]$ is 1 or 2.
% For this, we need to understand the relation between $\alpha$ and $\sqrt{5}$.
% Note that $\alpha^4-5\alpha^2+5=0$, so we can solve the equation for
% $\alpha^2$ as it is a root of $X^2-5X+5$, i.e.
% \[
% \alpha^2=\frac{5\pm \sqrt{25-20}}{2}=\frac{5\pm \sqrt{5}}{2},
% \]
% hence $\sqrt{5}=\pm (2\alpha^2-5)\in\Q(\alpha)$.
% So $\Q(\alpha,\sqrt{5})\subseteq\Q(\alpha)\subseteq \Q(\alpha,\sqrt{5})$, 
% which means that $E=\Q(\alpha)$ and
% $[E:\Q]=[\Q(\alpha):\Q]=4$.
% \end{sol}









\end{document}